\chapter{Retardo mental}
\section*{Notas clínicas}
\section*{Encare}
\subsection*{En suma}
Analfabetismo Pensión por incapacidad Laboral - núcleo familiar AF: Retraso Mental - Alcoholismo. AP: Escuela de Recuperación, INAU.
\subsection*{Agrupación sindromática}
Según motivo de ingreso:
\subsubsection*{Síndrome conductual}
Crisis de excitación psicomotriz (conversiva). IAE en contexto de impulsividad. Heteroagresividad. Fugas.
\begin{itemize}
	\item Actual Incapacidad en previsión de actos y sus consecuencias: alteración del juicio.
	\item Curso de vida Pauta de heteroagresividad que demuestra baja tolerancia a las frustraciones y un pobre control de los impulsos. Conductas antisociales. Consumo de sustancias.
	\item Pragmatismos y conductas basales.
\end{itemize}
\subsubsection*{Síndrome delirante}
Temática pobre, ideas y supersticiones ingenuas, alta incidencia sociocultural.
\subsubsection{Síndrome deficitario intelectual (congénito o precozmente adquirido)}

* Escolaridad: deficiente, destacar analfabetismo.
* Pensamiento: pobre, pueril, concreto, anecdótico. No puede dar cuenta de situaciones importantes de su vida. baja capacidad de abstracción.
* Logros: incapacidad adaptativa social, falta de habilidad en resolver problemas cotidianos que lo llevan a una disminución de la autonomía y responsabilidad esperadas para su edad evidenciado en:
** Logros sociales escasos.
** Escasos logros a nivel laboral, familiar, etc.
** Dificultades en manejo del dinero.

A confirmar por tests psicométricos.
\subsubsection*{Otros}
Epilepsia.
\subsection*{Personalidad y nivel}
Yo débil:
\begin{itemize}
\item Baja tolerancia a las frustraciones
\item Incapacidad de adaptación a nuevas situaciones.
\item Existencia acotada, dependiente, círculo restringido de intereses.
\item Dificultad para solucionar conflictos que lleva a:
\item Inhibición, oposición, desafío, terquedad, mitomanía.
\item Descargas bruscas de tensión emocional.
\item Alteración del juicio: no existe previsión de consecuencias de sus actos.
\item Rechazo de imperativos morales (o sumisión a los mismos).
\item Egocentrismo.
\end{itemize}
Personalidad: teñida por el déficit intelectual. Personalidad defectuosa.
\subsection*{Diagnóstico positivo}
\subsubsection*{Nosografía clásica}
.Oligofrenia o Retraso Mental.

Diagnóstico clínico presuntivo por:

* Déficit escolar importante
* Incapacidad adaptativa social
* Inicio en etapa de desarrollo (< 18 años)

A confirmar por tests psicométricos (CI<70). Dadas las alteraciones del pensamiento ... .... inferimos que clínicamente se halla por debajo de los parámetros normales.

.Grado de déficit

Diagnóstico clínico-psicométrico. Del punto de vista clínico:

* Leve: puede acceder a algún nivel de enseñanza. Entrenable y educable (puede hablar, leer y escribir bien).
* Moderado: no educable, entrenable (... lo que valoraremos en sucesivas entrevistas y una vez superado el cuadro actual).
* Grave: se acompaña de trastornos motores importantes, sobre todo neurológicos, no existe autonomía en habilidades elementales. Necesita supervisión continua. No educable, no entrenable.

.Tipo

Disarmónico por la inestabilidad afectiva, con reacciones emotivas frecuentes bajo la forma de reacciones explosivas de auto/heteroagresividad (y otros trastornos de conducta). Armónico: buena adaptación, docilidad, colaboración, pasividad y obediencia (retraso intelectual simple).

.Complicado

Con crisis convulsiva, conversiva, IAE, aumento del monto de impulsividad, síndrome delirante, etc.

.Causa de descompensación

Biopsicosocial.
\subsubsection*{CIE-10 - DSM-IV}
Al codificar en Eje II, es compatible con diagnósticos del eje I (el eje I descompensa el eje II). F70 Retraso mental leve F71 Retraso mental moderado F72 Retraso mental grave F73 Retraso mental profundo F78 Otro retraso mental F79 Retraso mental sin especificación Cuarto carácter para especificar la extensión del deterioro añadido del comportamiento: • F7x.0 con deterioro del comportamiento mínimo o ausente • F7x.1 con deterioro del comportamiento importante que requiere atención o trata-miento • F7x.8 con otros deterioros del comporta-miento • F7x.9 sin alusión al deterioro del comporta-miento
\subsection*{Diagnósticos diferenciales}
1. Epilepsia generalizada tipo Gran mal: • Descartar en base a diferencias con crisis conversivas. • Dada la frecuente comorbilidad, deben realizarse estudios paraclínicos. • En caso de haber alteraciones conductuales (IAE, heteroagresividad, fuga) y es epiléptico conocido, se puede plantear DD con: • Estado crepuscular postictal. • Crisis parcial compleja. • (ambos se descartan porque no existe trastorno de la conciencia en la comisión del acto).
2. Crisis de adolescencia patológica: en el caso del RM la alteración se da en el curso longitudinal y está centrada en el déficit escolar y adaptativo social, hecho que excede el DD planteado. En la crisis de adolescencia existe: • menor duración • historia previa sin alteraciones deficitarias • generalmente hay causa desencadenante.
3. Según HC puede plantearse DD con: Neurosis, Trastorno de la Personalidad. Neurosis: necesita una personalidad más con-formada (el oligofrénico puede usar mecanismos de defensa seudoneuróticos).
\subsection*{Diagnóstico etiopatogénico}
Multifactorial.

.Biológico

Factores pre, peri, postnatales: metabólicos, complicación de embarazo, infecciones neonatales, traumatismos obstétricos. Importa descartar: consanguinidad de padres, edad de la madre, alcoholismo paterno ( Frecuencia, citar si existe).

.Psicosocial

Actuando sobre este terreno biológicamente o congénitamente alterado, existen elementos que nos hablan de: DEPRIVACION AMBIENTAL • alteraciones del medio familiar, violencia, alcoholismo • medio de poco estímulo • familia poco continente

\subsection*{Paraclínica}

Destinada a: • Evaluar déficit • Descartar diagnósticos diferenciales • Con vistas al tratamiento • Valoración general

.Biológico

• Lo somático que tenga • EEG en busca de signos focales, neurólogo. • Rutinas • Valoración para uso de Carbamazepina: Hemograma completo, Funcional y enzimograma hepático.

.Psicológico

• Test psicométrico específico: Bender y Weschler para evaluar CI y grado de déficit en su escala ejecutiva y verbal. • Test de personalidad proyectivos y no proyectivos.

.Social

Directamente o con Asistente Social: • Terceros dada la poca confiabilidad • Red de soporte social • Incidencia del medio en su patología y en la descompensación • Historia perinatal para orientación de etiología

\subsection*{Tratamiento}

Destinado a:

* Cuadro actual: tratamiento sintomático Bps
* Largo plazo: bPS, mantendremos fármacos de mantenimiento, pero será fundamental-mente psicosocial y estará destinado a favorecer inserción social y combatir complicaciones.

Se usará medicación en casos en que footnote:[National Institute for Health and Care Excellence. "Psychotropic medicines in people with learning disabilities whose behaviour challenges." (2017).]:
* Las intervenciones psicosociales solas no sean suficientes.
* Exista riesgo para sí o para terceros.

.Cuadro actual

Internación en Hospital General: fundamental-mente por continencia interna y externa con de descontrol por parte del paciente y aislamiento del foco conflictivo. Breve porque es mal tolera-da. Vigila fuga, IAE, heteroagresividad. En lo posible aislado al inicio. Equipo multidisciplinario.

Biológico

1. Sedaremos con Benzodiacepinas: Clonazepam, en su calidad de sedativo y su acción contra la irritación, impulsividad y disforia. Además otras benzodiacepinas, al tratarse de un cerebro disfuncional, presentan con > frecuencia el fenómeno de desinhibición. También antiepiléptico. Indicaremos 2 mg v/o H8, H14 y 4 mg VO H20 (2 2 4), que iremos según respuesta hasta llegar a 14-16 mg/día.
2. Indicaremos Carbamazepina como estabilizador del humor y por su acción sobre la irritabilidad y la disforia e impulsividad. Empezamos con 200 mg VO c/12 hs e iremos pudiendo llegar a 1200-1600 mg/día. Parecería que la dosis óptima corresponde a una concentración plasmática de 4-12 µg/ml. Realizaremos controles con hemograma (semanal el 1º mes, luego mensual o bi-mensual), funcional y enzimograma hepático (mensual el 1º trimestre, luego bimensual)., ya que como efecto secundario puede disminuir la fórmula leucocitaria con el consiguiente riesgo de infecciones graves y también provoca alteraciones en el FH (hepatotoxicidad).
3. Indicaremos Propericiazina (cerebro disfuncional > EPI y > EPS) NL que contribuye a la sedación, con acción sobre la impulsividad. Como efecto secundario baja el umbral convulsivo y existe la posibilidad de que nuestro paciente sea epiléptico. Se indica 25 mg VO H20, pudiendo llegar a 50 mg según la evolución.
4. Realizaremos Flunitrazepam 2 mg v/o ya que consideramos fundamental la regulación del sueño. Retiraremos al obtener mejoría. NOTA: considerar el uso de antipsicóticos atípicos, sobre todo Risperidona.

Psicológico

Entrevistas frecuentes para lograr buen vínculo, tranquilizar con respecto a la internación.

.A largo plazo

Estará destinado a incidir sobre la adaptación social, procurando la autonomía s/t con medidas psicosociales.

Biológico

Realizaremos controles al principio semanales, que iremos espaciando hasta llegar a mensuales. Procuraremos disminuir al mínimo efectivo los fármacos para facilitar el cumplimiento. Previo al alta, según valoración del perfil de cumpli-miento del paciente y la continentación del me-dio, en caso de ser dificultoso el cumplimiento indicaremos Palmitato de Pipotiazina 25-50 mg i/m c/4 semanas que nos asegura el cumplimien-to.

Psicosocial

Vincularemos a taller de rehabilitación que puede ser dificultoso por los trastornos conductuales. Realizaremos psicoeducación de familiares para mejorar continencia del medio, lograr con-troles clínicos periódicos y cumplimiento de la medicación. Vincularemos a la familia con ex-perto en terapia familiar para cambio de conductas de ésta que puedan incidir en las descompensaciones. Eventualmente vincularemos al pa-ciente con expertos en Retraso Mental. Vincula-remos con AS para que tenga acceso a beneficios sociales.

\subsection*{Evolución y pronóstico}
PPI: bueno con tratamiento instituido, supedita-do a trastornos conductuales con auto/hetero. PVI: bueno, supeditado al psiquiátrico. PVA: sujeto a lo orgánico que tenga, en cierto modo vinculado al PPA que pensamos reservado ya que está dificultado por: • Autonomía limitada • Conductas antisociales • Continentación social y recursos económicos • Bajo umbral de reactividad para psicosis Evolución que intentamos mitigar con las medidas efectuadas.
