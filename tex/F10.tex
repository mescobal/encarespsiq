\chapter{Trastornos por consumo de alcohol}
\section*{Notas clínicas}
Los trastornos vinculados al alcohol son varios, y pueden acumularse como diagnósticos. Una clasificación posible es dividir entre trastornos crónicos (abuso, dependencia) y agudos (intoxicación, abstinencia, etc). El encare clásico de "Delirio Alcohólico Subagudo" no da cuenta de todas las situaciones vinculadas al alcohol y la nomenclatura está en desuso. El DASA, nosográficamente, quedaría ubicado como "Delirium inducido por alcohol" (por la existencia de confusión mental). Los diferentes cuadros posibles son\cite{cie102000}:
\begin{itemize}
\item F10.0 Intoxicación alcohólica aguda (.03 con delirium)
\item F10.1 Consumo perjudicial / Trastorno por Abuso de Alcohol
\item F10.2 Dependencia alcohólica / Trastorno por dependencia de Alcohol
\item F10.3 Abstinencia alcohólica
\item F10.4 Delirium por abstinencia de Alcohol / Delirio alcohólico subagudo / Delirium Tremens.
\item F10.5 Trastorno psicótico inducido por alcohol: .51 = con ideas delirantes (Celotipia alcohólica), .52 = con alucinaciones (Alucinosis alcohólica)
\item F10.6 Trastorno amnésico inducido por alcohol
\item F10.7 Trasotrnos psicótico residual (.73 = Demencia inducida por alcohol).
\item F10.8 Otros trastornos (Trastornos de ansiedad y del estado de ánimo / Trastorno sexual inducido por alcohol / Trastorno del sueño inducido por alcohol).
\item F10.9 Trastorno relacionado con alcohol no especificado.
\end{itemize}
\subsection*{Trastornos mentales agudos y subagudos del alcoholismo crónico}
En común: estado de confusión con diferente profundidad\cite{ey1996}.

Agudo = Delirium Tremens

Subagudo = Delirio onírico alcohólico.

.Delirio alcohólico subagudo
Lasègue (1869: DASA), Magnan (1874: Delirio alcohólico simple), encefalosis alcohólica subaguda.

Comienzo: acceso confuso-onírico agitado, continuación de pesadillas. Duración variable, inicialmente puede ser intermitente, luego se hace continuo. Delirio actuado (pide socorro, amenaza o pega, se atrinchera, huye por la calle, ataca enemigos imaginarios, disparos).

Presentación: sudoración, olor podálico (Clérembault), agitado, delirio PAFAINVIF, gestos de defensa o ataque.

Delirio onírico: Lasègue "Le délire alcoolique est un rêve", Garnier, Magnan. Vivenciado, actuado, onirismo profesional (temática laboral). Zoopsias (como imágenes oníricas, ratas, serpientes, arañas, sapos). Escenas de terror (es atacado), visiones macabras, incendios. Raramente temas eróticos o de celos. Predominio de alucinaciones visuales (como en el sueño). Pueden haber alucinaciones auditivas (amenazantes, peyorativas), olfativas (gas, azufre), gustativas (veneno), alucinaciones de la sensibilidad general (pinchazos, viscosidad, gusanos). Características:

* Vivido
* Móvil,cambiante.
* Cargado de ansiedad, carácter penoso, peligro inminente
* Contexto de confusión mental: DOTE, distraibilidad. Sin trazas mnésicas excepto las ideas fijas postoníricas.

Síndrome somático: taquicardia, temperatura normal o levemente aumentada. Poca afectación del estado general (DD con DT). Sudoración, polipnea. Buscar subicericia. Temblor distal. Hiperalgia de masas musculares. Trastornos visuales: ambiolpía, discromatopsia, escotoma.

Formas clínicas: más frecuente forma confusoonírica. Excepcionalmente formas alucinatorias no oníricas o sin confusión (Marchand) o formas con automatismo mental (eco del pensamiento, comentario de actos, s/t alucinosis de bebedores de Wernicke).

Evolución: favorable (o a DT). Secuelas: ideas fijas PO o crónicas (psicosis alcohóicas crónicas).

Diagnóstico: S° confusional en alcoholista.

.Delirio alcohólico agudo: DT
Comienzo: más agudo, más grave. Sudoración profusa, temblor generalizado, agitación intensa, insomnio total.

Delirio onírico más marcado, intensamente alucinatorio, onirismo profesional, DOTE completa, zoopsias (microzoopsias en esquinas del cuarto). Liberación de movimientos anormales (masticación, succión, prensión). Temblor generalizado en todo el cuerpo. Fiebre (>39) persistente por días. Fiebre > 40 → delirio alcohólico hiperazoémico. Deshidratación. Diuresis suele ser (paradójicamente) suficiente.


TIP: No existe DT apirético


Examen: azoemia (puede ser normal, si está ↑ pronóstico grave, delirio agudo azoémico). Ionograma (buscar hipopotasemia).

Evolución: favorable (retorno del sueño, apirexia, retrocede confusión) o no (coma hipertérmico, convulsiones, síncope).

Anatomía patológica: encefalitis psicótica aguda. Meningitis crónica alcohólica.

Patogenia: DT → autointoxicación por disminución de capacidad funcional del hígado. DT: cada vez menos frecuentes por tratamiento precoz.

.Formas delirantes del alcoholismo crónico
. Secuelas postoníricas: Regis describe 3 tiempos del despertar de los estados oníricos: ausencia de crítica → dudas → rectificación. Cuando persiste la idea → idea fija postonírica (explica el delirio con detalle). Pueden haber IFPO permanentes.
. Estados de alucinosis alcohólica: psicosis alucinatoria o síndrome de automatismo mental subagudo con predominio de actividad alucinatoria (más auditivo que visual) sin desestructuración de la conciencia. BOTE, sin obnubilación, sin trastorno de memoria. Alucinosis de los bebedores (Wernicke). Inicio: ilusiones, s/t en la noche (ruidos, conversaciones) y rápidamente aparecen alucinaciones acusticoverbales con carácter de sensorialidad más acentuado. Contenido hostil (insultos, amenazas, oye hablar de él en tercera persona). Poco frecuente alucinaciones visuales (más ilusiones, formas amenazantes, sombras). Experiencia delirante (imaginación alucinatoria). Es una forma de delirio vivido solamente en los datos perceptivos. Evolución favorable en pocos días. Si se cronifica → F20 desencadenado por EPA OH.
. Delirios alcohólicos crónicos:
.. Delirios de interpretación: psicosis sistematizadas. Celotipia alcohólica. Personalidad neurótica predelirante, trastorno "procesual" de la personalidad (Jaspers). Puede estar precedido de una fase onírica. Aumenta el número de amantes a medida que progresa el delirio. Mezcla de temas hipocondríacos, homosexuales, incestuosos latentes.
.. Delirios alucinatorios: paranoia alucinatoria de los bebedores (Kraepelin). Pueden estar precedida de una alucinosis. Actividad alucinatoria y seudoalucinatoria muy viva. Esfera acusticoverbal o visual. Diálogos, susurros, comentario de actos, fotopsias. Poco ansiosa, alegría paradojal, intenta apartarse de las alucinaciones con distracción. Fabulación. Fantasía. Evolución a la indiferencia → demencia.
.. Delirios paranoides: evolución de tipo esquizofrénico → rediagnóstico.

.Síndromes anatomoclínicos de las encefalopatías alcohólicas
. Psicosis polineurítica alcohólica de Korsakoff. Más frecuente en mujer. Precedido de uno o más accesos subagudos. Cambios de humor, depresión o apatía y cefaleas. Trasotnros de memoria, amnesia de fijación, fabulación. Lesiones de cuerpos mamilares. Signos de polineuritis (dolores, paresteisas o abolición de ROT, atrofia muscualr, steppage). En agudo: encefalomielitis alcohólica aguda (confusión mental, onirismo, paraplejia fláccida, repercusión general, insomnio, deshidratación, incontinencia). Evolución fatal. Forma crónica: trastorno de memoria → clínica de cuadro demencial progresivo.
. Encefalopatía de Gayet-Wernicke: comienzo progresivo. Ansiedad, irritación, insomnio, indiferencia, inactividad, cefaleas, vértigos. Período de estado: torpor y somnolencia intercalado con agitación, delirio y alucinaciones. Patognomónico: trastorno oculares (parálisis de los movimientos de los globos, descenso de la agudeza visual, fotofobia, nistagmo, contractura de tipo meningítico). Evolución en 10-15 días al coma y muerte. Remite con tratamiento.
. Encefalopatía alcohólica portocava: F10 + cirrosis + trastornos de conciencia, trastornos del humor (apatía, irritabilidad, despreocupación pueril) + flapping tremor (batimiento de alas con flexión y extensión de los dedos), cierta hipotonía muscuilar. Duración breve, coma reversible. Hiperamoniemia (aumento de producción, disminución de la eliminación por insuficiencia hepática) → interrupción del ciclo de Krebs.

.Demencias alcohólicas
Predominio de apatía y degradación social. "Demencia ética". Puede haber regresión luego de tratamiento prolongado. Puede tomar forma de seudoparálisis general alcohólica (si se acompaña de temblor, disartria, anisocoria, indiferencia).
\section*{Encare}
\subsection*{Agrupación sindromática}
Va a depender del trastorno

A= Agudo, C= Crónico, R= Recurrente
\begin{center}
\begin{tabular}{|c|c|c|c|c|c|c|}
\hline
Diagnóstico & Conductual & Confusional & Abst & Delirante & Alt.Percep. & Def.Int.\\
Intoxicación & A+C &  & & &\\
Delirium & A+C & A & A & A\\
Abuso & C & & & &\\
Dependencia & C &  & A/R & & \\
Abstinencia & C & & A\\
Psicosis & C & & & C\\
Amnésico & C & & & C\\
Demencia & C & & & & & C\\
\hline
\end{tabular}
\end{center}
\subsubsection*{Síndrome confuso-onírico}
De instalación brusca, de X tiempo de evolución, con síntomas fluctuantes.
\paragraph{Síndrome confusional}
Donde destacamos las fluctuaciones en la sintomatología que pasamos a detallar dada por trastornos en (COMA): Conciencia: no presentifica (siendo incapaz de organizar el aquí y ahora), embotamiento, obnubilación, que muestra falta de lucidez y claridad del campo de la conciencia, con imposibilidad para efectuar una síntesis adecuada de los contenidos psíquicos, los cuales se confunden y aglutinan. A partir de este trastorno fundamental derivan los otros elementos del síndrome. Orientación: desorientación TE con autopsíquica generalmente conservada. Memoria: falsos reconocimientos, alteraciones en la memoria de fijación, evocación laboriosa, fabulación. Atención: deficiente tanto en su modalidad voluntaria como espontánea. El paciente en un esfuerzo por poner en orden su síntesis mental alterada hace intentos por salir del embotamiento y orientarse lo que se manifiesta por una perplejidad ansiosa. Esta confusión se acompaña de trastornos perceptivos típicos que configuran un delirio de características peculiares que pasamos a analizar. Predisponen a confusión: edad avanzada, lesión SNC (demencia, etc.), abstinencia de sustancia, que-maduras, cirugía, etc.
\paragraph{Síndrome onírico}
Dado en: Lo vivencial: por vivencias mórbidas, de instalación aguda, incompartibles, irreductibles a la lógica, que han perdido el juicio de realidad, que se instalan sobre esta incapacidad del paciente de reconocer lo externo. A temática: persecutoria, de daño y perjuicio, profesional, erótica, místicas, celos. A mecanismo: alucinatorio (visual: zoopsias), ilusiones (dismorfopsias, dismegalopsias). Mal sistematizadas: sus componentes no guardan una relación lógica entre sí, presentan movilidad, carácter cambiante y mínima organización. No presentan hilo argumental. Lo conductual: con conductas de deambulación, peleas con personas imaginarias, agresión, defenestración, huida. Definimos este delirio como onírico por las siguientes características: • Profusión de alucinaciones de tipo escenográfico semejante a sueños • Afectividad y psicomotricidad acompañan uniformemente en forma de agitación y ansiedad • Adhesión al delirio: actuado, ejecutado y vivido. • Fluctuante: se presenta en oleadas tomando el pensamiento del paciente en sacabocados, por momentos el paciente recobra su lucidez • Exacerbación nocturna con de agitación y ansiedad fases hipnagógicas ansiosas y atormentadoras.

.Fenómenos acompañantes
Excitación psicomotriz. Humor y afectividad. Lenguaje caótico, desordenado.

\paragraph{Síndrome de abstinencia}
Actual o retrospectivo. Definido por la aparición (en horas o días) de sintomatología luego de interrupción o disminución de ingesta alcohólica abundante previa, con (bastan 2) hiperactividad autonómica (sudoración, taquicardia), temblor distal de manos, insomnio, náuseas o vómitos, alucinaciones visuales / táctiles / auditivas o ilusiones, agitación psicomotora, ansiedad, crisis comiciales de gran mal. 3. Síndrome conductual 1. Cuadro actual: lo relacionado al MC, CB y pragmatismos. 2. Curso de vida: trastornos conductuales si existen. Alcoholista de larga data con pérdida del control e impulso a la embriaguez con elementos de abstinencia que calman con la ingesta (nombrarlos), con elementos que hablan de tolerancia (nombrarlos) o tolerancia inversa (nombrarlos), con consumo persistente a pesar de las consecuencias del mismo. 4. síndrome somático Sudoración, PNM (temblor), CV (central y periférico), toxiinfeccioso (fiebre, tos, expectoración), endócrino. Puede incluirse en un síndrome de abstinencia.

\subsection*{Personalidad y nivel}
Nivel: Cualquiera.

Personalidad: dependiente, paranoico. Dificultad en evaluación de rasgos por el alcoholismo.
\subsection*{Diagnóstico positivo}
\subsubsection*{Nosografía clásica}
En el caso de Delirio Alcohólico Subagudo: Psicosis \faArrowRight Aguda \faArrowRight Confuso\-onírica (por el síndrome confuso\-onírico analizado, es un diagnóstico inespecífico en lo nosográfico que reconoce una causa orgánica desencadenante) \faArrowRight de causa alcohólica por tratarse de un alcohólico crónico con dependencia severa al alcohol.

En contexto de:
\begin{itemize}
\item ingesta excesiva, mayor a usos dietéticos usuales, dado por cantidad y pauta (ej: diario y solitario).
\item de X años de evolución
\item con pérdida del control del consumo con incapacidad para abstenerse
\item con S. de abstinencia ante el cese o disminución del consumo (consume para evitarlo)
\item por presentar tolerancia: necesita ingesta para lograr los mismos efectos o alcanza la embriaguez con menores cantidades que antes
\item con trastornos mnésicos (black out, palimpsestos)
\item abandonando otras fuentes de placer
\item persiste con el consumo a pesar de consecuencias adversas (pragmatismos, orgánicas)
\end{itemize}
Por lo que decimos que se trata de un paciente con dependencia al alcohol (leve, moderada, severa) por la cantidad de síntomas y por el deterioro FA SE SO LA Corresponde a un alcoholismo: Tipo Jellineck Alonso\-Fernández Alfa Sintomático, secundario a otro trastorno psiquiátrico Beta Sin síndrome de abstinencia, sin tolerancia pero con repercusión orgánica extra cerebral Gamma Tolerancia, dependencia, abstinencia, falta de control, impulso a la embriaguez. Bebedor irregular, acoholómano. Delta Tolerancia, dependencia, abstinencia, incapacidad de abstención, no existe pérdida de control. Bebedor excesivo regular

Con estos elementos hacemos diagnóstico de DELIRIO ALCOHÓLICO SUBAGUDO por tratarse de un alcoholista crónico con modificaciones en la ingesta (aumento, suspensión brusca, mantenimiento) y por el Sº confuso\-onírico analizado, donde se destaca la presencia de temblores, sudoración y zoopsias (típicas alteraciones del onirismo alcohólico). Destacamos la existencia de factores de comorbilidad tales como: • adolescentes: intoxicación aguda, abstinencia de drogas, TEC, postQ. • adulto: cuadros MQ, intoxicación alcohólica/abstinencia, psicofármacos. • anciano: adulto + estresores previamente bien tolerados (postQ, EPOC, arritmia), enfermedad cerebrovascular (lo + frecuente).
\subsubsection{DSM IV}

Eje I. Caben varias posibilidades de codificación según la semiología presente: ver notas al inicio. Ejemplos: . Delirium por intoxicación por alcohol (DASA). . Delirium por abstinencia de alcohol (Del Alc Subag). . Trastorno psicótico inducido por alcohol, con ideas delirantes. . Trastorno psicótico inducido por alcohol, con alucinaciones. . Trastorno psicótico inducido por alcohol: con alucinaciones / con ideas delirantes.

\subsection*{Diagnósticos diferenciales}
\begin{itemize}
\item DELIRIUM TREMENS: no pensamos ya que en este cuadro confusional agudo existe: > gravedad con deshidratación, alteraciones HE, alteraciones NV con temperatura de 40º, taquicardia, sudoración, PA. • temblor importante • agitación intensa y agotadora • insomnio casi total
\item Otras causas de confusión mental: no pensamos: enfermedades médicas, otras sustancias.
\item Encefalopatía por derivación porto-cava: se trata de una descompensación de una hepatopatía crónica por hiperamoniemia, que se presenta como un trastorno de conciencia que por lo general agrega ictericia, ante un estrés físico grave (ej.: hemorragia digestiva).
\item Encefalopatía de Wernicke: en la cual se presenta confusión grave pero que agrega síntomas neurológicos: nistagmo, oftalmoplejia del III par, ataxia.
\item Alucinosis alcohólica: que también ocurre ante supresión/aumento de la ingesta alcohólica, pero en donde no existe confusión, puede haber una alteración leve de la conciencia, las alucinaciones son s/t Auditivo-verbales, de amenaza alucinatoria, hay cierta crítica a ellas y carece de correlato somático.
\item No pensamos que se trate de otras psicosis agudas (afectivas o delirantes) por el importante trastorno de conciencia y las características oníricas del delirio.
\end{itemize}
\subsection*{Diagnóstico etiopatogénico y psicopatológico}

.Del cuadro actual
* Abstinencia
* Causas intercurrentes (infecciones, cirugía, TEC)

El delirio comporta para Ey aspectos negativos y positivos, estando presente una desestructuración de la conciencia.

.De la intoxicación crónica

Biológico: dependencia biológica manifestada en el síndrome de abstinencia, predisposición hereditaria (padre).

Psicosocial: 1º social y luego reitera y aumenta la ingesta para evitar experiencias displacenteras.

* intolerancia a situaciones disfóricas
* búsqueda del placer
* tendencia a la satisfacción oral

Existirían rasgos que favorecen (terreno de personalidad predisponente): dependencia, intolerancia a las frustraciones, trastorno profundo de vínculos, abandono, mal manejo de la agresividad. También son predisponentes: cuadros afectivos previos, ansiedad. Marcar si existe:

* identificación con padre alcohólico
* conducta aprendida con pautas conductuales y modo de relación

Psicopatología, etiopatogenia

Considerar los siguientes factores:

* conductual: refuerzo positivo/negativo.
* social: refuerzo social, subculturas.
* genéticos: riesgo x 4 en hijos de alcohólicos, si hay AF: inicio precoz, más trastornos conductuales, peor pronóstico, formas más graves.
* biológicos: neurotransmisores.
* psicoanálisis: Superyo punitivo, fijación a etapa oral.
* comorbilidad: depresión, ansiedad, trastornos de la personalidad.
* historia infantil: trastorno por déficit de atención/hiperactividad.

\subsection*{Paraclínica}

Descartar comorbilidad:

* neurológico: TEC con HSC
* infeccioso: s/t renal y respiratorio
* medio interno: deshidratación

===== Biológico

Examen físico exhaustivo:

* PyM: ictericia, equimosis, anemia, hidratación, estigmas de alcoholismo (angiomas, telangiectasias, palmas y plantas hepáticas, ginecomastia, disposición ginoide del vello pubiano, atrofia testicular).
* CV: dilatación cardíaca.
* PP: concomitancia con EPOC (asociación lesional)
* ABD: hepatomegalia, esplenomegalia, circulación colateral (HT portal, ascitis).
* PNM: parálisis horizontal de la mirada, tono muscular y sensibilidad, parestesias (polineuropatía de MMII, velocidad de conducción). Flapping o aleteo, rueda dentada, hiperreflexia, Síndrome cerebeloso (marcha). Fondo de ojo.

Examenes complementarios:

De suma importancia para detección de comorbilidad (infeccioso, neurológico, medio interno) y para valoración del alcoholismo.

* hemograma completo: anemia carencial por déficit nutricional, leucocitosis y VES [infecciones: hay aumento de susceptibilidad]).
* ionograma: ver Zn y Mg (cofactores de vitamina B).
* crasis sanguínea: s/t tiempo de protrombina que disminuye al disminuir la capacidad funcional del hígado.
* funcional y enzimograma hepático.
* PEF: hipoalbuminemia
* Glicemia, azo, crea, orina
* RxTx (neumopatía por aspiración)
* TAC: hematoma subdural crónico, atrofia cortical (se desconoce su relación con OH).

Exámenes complementarios para detección de consumo:

* Alcoholemia: valores > 50 mg/100ml sugiere consumo de riesgo. Valores > 150 mg/100 ml sugieren existencia de tolerancia (y por lo tanto de dependencia).
* GGT (alta especificidad pero baja sensibilidad), dado que la ingestión aguda no modifica sus valores, es un indicador de consumo perjudicial habitual (consumo > a 40\-80 g/día en ausencia de hepatopatía). Cifras de GGT de Nx3 se consideran valores muy sugestivos de consumo perjudicial. Para monitorizar la abstinencia: las cifras disminuyen a un 50\% en 5-7 días y se normalizan a las 4-8 semanas del cese de la ingesta (vuelven a aumentar si se reanuda el consumo). Pueden haber valores elevados en hepatopatías no alcohólicas o x fármacos.
* Volumen corpuscular medio (alta especificidad, baja sensibilidad): aumentado en el 70\-90\% de pacientes alcohólicos, con consumos superiores a 60 g/día x períodos prolongados. Luego de la sus-pensión de la ingesta, disminuye a los 90 días aproximadamente. Aumenta nuevamente en caso de que se reinicie la ingesta.
* Otros: relación AST/ALT, Transferrina deficiente en hidratos de carbono.

===== Psicológico

Afianzar vínculo, obtener más datos.

===== Social

Familiar, datos anteriores, etc.

\subsection*{Tratamiento}

El tratamiento debe ser individualizado ajustando el enfoque a las características del paciente y del equipo tratante.

==== Cuadro Actual

Depende del diagnóstico

* F10.00 Intoxicación alcohólica aguda. Sedación con haloperidol 5 mg IM a repetir. Tiamina IM si hay que hacer hidratación con SGF. MdeC si es necesario.
* F10.03 Delirium por intoxicación por Alcohol. Haloperidol IM + Tiaprida IM + Hidratación.
* F10.1- Consumo perjudicial / Trastorno por Abuso de Alcohol: en caso de consumo con patrón compulsivo, plantearse uso de Topiramato en dosis progresivas, comenzando con 50 mg/día, aumentando 50 mg cada semana hasta 300 mg/día[kenna2009review].
* F10.2x Dependencia alcohólica / Trastorno por dependencia de Alcohol Naltrexona 50 mg/día, o Topiramato, en dosis progresivas hasta 300 mg/día, en 2 tomas.
* F10.3- Abstinencia alcohólica BZD de vida media larga v/o (si no hay hepatopatía): Diazepam 10\-20 mg/día v/o. En caso de toque hepático: Lorazepam.
* F10.4- Delirium por abstinencia de Alcohol / Delirio alcohólico subagudo. Haloperidol IM + Lorazepam IM + Vitaminoterapia IM + Hidratación.
* F10.51 Trastorno psicótico inducido por alcohol (con ideas delirantes)/ Celotipia alcohólica: Risperidona VO.
* F10.52 Trastorno psicótico inducido por alcohol (con alucinaciones) / Alucinosis alcohólica Según gravedad: Risperidona VO o Haloperidol IM.
* F10.6- Trastorno amnésico inducido por alcohol Vitaminoterapia (complejo B) + Nootrópicos a dosis altas.
* F10.73 Demencia inducida por alcohol. Vitaminoterapia + tratamiento de demencias.
* F10.8- Trastornos de ansiedad y del estado de ánimo en alcohólicos / Trastorno sexual inducido por alcohol / Trastorno del sueño inducido por alcohol: ver encares respectivos. Para ansiedad: primera línea = Buspirona (evitar BZD). Para depresión ISRS.

Fármacos específicos

Topiramato: es una molécula similar a la fructosa, con propiedades anticonvulsivantes, aumenta la actividad neuronal facilitada por GABA-A y simultáneamente antagoniza los receptores AMPA y kainato-glutamato lo que puede disminuir la liberación de dopamina inducida por alcohol en el núcleo accumbens. Teóricamente el aumento de la inhibición GABA de las neuronas dopaminérgicas del núcleo accumbens interferiría con el agonismo exitatorio glutamatérgico característico del alcoholismo crónico y atenuaría la actividad dopaminérgica mesolímbica. Esto atenuaría los efectos de recompensa de la ingesta de alcohol. Secundariamente tendría acción neuroprotectora sobre el aumentode la actividad glutamatérgica ocasionada por la ingesta crónica de alcohol. Por su acción anticonvulsivante, se comporta secundariamente como protector del umbral convulsivo en un proceso de abstinencia.

En todos los casos: mantener abstinencia alcohólica + intervenciones psicosociales + vitaminoterapia v/o.

Ejemplo de pauta de tratamiento: Delirio Alcohólico Subagudo DAS: Urgencia médica con riesgo vital. DT: CTI. Directivas: . inmediato: calmar agitación, yugular delirio, compensación del punto de vista general . largo plazo: tratamiento de la enfermedad de fondo Tratamiento de la confusión mental en general: . corregir agente causal . corrección de factores intercurrentes, funciones vitales, psiquiátrico sintomático Mantener la internación con controles diarios, monitoreo de síntomas. Equipo multidisciplinario. Medidas de sostén: control de signos vitales mantener abstinencia nutrición, reposo iluminación medidas de orientación de realidad (reloj, calendario, iluminación, acompañante a permanencia). En caso de riesgo de existencia de agitación marcada, intentos de quitarse la vía que instalaremos, au-to/heteroagresividad, fugas, instauraremos medidas de contención a cargo de personal entrenado (según normas del MSP).

Medidas específicas

Bajar la fiebre (si hay): Dipirona, medidas físicas.

Hidratación, abundantes líquidos v/o. Si hay deshidratación (fiebre, diaforesis, vómitos, diarreas, san-grados, pliegue perezoso, agitación) VVP + 1000 cc SGF c/8 hs (o hidratación rápida con 1 l en 2 hs., 1 l en 6 hs y 1 l hasta completar las 24 horas.) Previamente administraremos Tiamina ya que la glucosa aumenta los requerimientos de ésta en el SNC, pudiendo precipitar una encefalopatía de Wernicke.

Sedación: fundamental para: tratar la agitación, prevenir la progresión a DT, alivio sintomático, facilitar tratamiento. El fármaco a usar dependerá de si aparece durante la intoxicación o en abstinencia.
a. Intoxicación: no usar BZD por riesgo de agravar depresión respiratoria. Usar NLS (Levomepromazina 25 mg i/m a repetir, sabiendo de cierto riesgo dado que baja el umbral convulsivo) o Tiapridal 1 amp (100 mg) cada 6-8 horas. Luego que ceda la intoxicación, pasamos a benzodiacepinas para evitar el síndrome de abstinencia.
b. Abstinencia: Por ser el alcoholismo favorecedor de una malabsorción crónica, comenzaremos con vía intramuscular: Lorazepam 1 amp (4 mg) cada 6-8 hs. Tan pronto como sea posible usaremos la vía oral, teniendo 2 posibilidades:

* Si no hay elementos en contra: Diazepam 15-20 mg/día en 3 dosis v/o 5-5-10 mg a regular según evolución, tolerancia.
* En caso de: agitación intensa, anciano, FH alterado, alteración de tiempo protrombina, hipoalbuminemia usamos una benzodiacepina de vida media más corta y sin metabolitos activos como el Lorazepam 2 mg c/4\-6 hs v/o a regular por evolución (no afectado su metabolismo por 1º paso hepático, no tiene metabolismos activos, no tiene efecto acumulativo). En caso de que el síndrome de abstinencia sea intenso y domine el cuadro, existen pautas de tratamiento con benzodiacepinas, una de las más usadas es con Diazepam en un esquema de 4 días:
* Día 1: 20 mg cada 6 horas
* Día 2: 20 mg cada 8 horas
* Día 3: 20 mg cada 12 horas
* Día 4: 20 mg en 24 horas.
Alternativas: Clometiazol, Tetrabamato.

Haloperidol: NL incisivo con acción sobre el delirio, contribuyendo a la sedación. Alta potencia con poco efecto sobre: ritmo y contractilidad cardíaca, resistencia vascular periférica, actividad respiratoria. Dosis: 2,5 mg H8 + 5 mg i/m H20 que iremos ajustando según respuesta.
Tiapridal: no es de 1ª elección. Derivado de NL con poco efecto EP, que no da depresión de conciencia. Activo frente a agitación, contribuye a la sedación. Indicaciones : • si no anda con BZD • si hay insuficiencia respiratoria • usado s/t en DT • si hubo TEC (por posibilidad de efecto paradojal de BZD) Dosis: 400 mg v/o en 4 dosis, o 300 mg i/m (1 amp de 100 cada 8), 1 amp de 100 en 1 l suero cada 8 que iremos según respuesta a 900-1200 mg.

En suma:

. Paciente normal: Diazepam 5\-5\-10 v/o
. Baja tolerancia: Lorazepam 2 mg v/o c/6\-8 hs
. Condiciones especiales: Tiapridal 100 mg v/o c/6\-8

En cuanto el cuadro agudo se estabilice pasaremos la medicación a vía oral.

==== A largo plazo

.Tratamiento de la dependencia de alcohol

Naltrexona: antagonista opiáceo que actúa por bloqueo del sistema opioide endógeno (delta y mu) reduciendo la apetencia por el alcohol. Previo a eso nos aseguraremos que el paciente tiene voluntad de continuar el tratamiento, descartaremos consumo de opioides en los 10 días previos y descartaremos la existencia de insuficiencia renal o hepática así como de hepatitis en curso y embarazo. El funcional hepático debe tener valores menores a los normales x 3, con bilirrubina a niveles normales. La dosis inicial es igual a la de mantenimiento, de 50 mg/día en una sola toma. De aparecer efectos secundarios (náuseas, mialgias, insomnio, dolores osteoarticulares) se puede bajar la dosis a 25 mg/día. El tratamiento debe prolongarse por 12 semanas, con controles con examen físico, funcional y enzimograma hepático (semanas 2, 4, 8 y 12).

En caso de haber síntomas depresivos, valoraremos el uso de antidepresivos de tipo ISRS a dosis están-dar (Paroxetina y Fluvoxamina > Sertralina y Citalopram > Fluoxetina). Si bien la frecuencia de síntomas depresivos durante la abstinencia es alta, la gran mayoría remiten en forma espontánea, no requiriendo AD. Los antidepresivos pueden ser de utilidad para mantener la abstinencia, lo que aún no está totalmente demostrado.

En caso de Celotipia alcohólica, los antipsicóticos deben usarse de forma prolongada dada la cronicidad del proceso.

Si se usaron benzodiacepinas, disminuirlas de forma gradual (paciente con tendencia a adicciones), pudiendo usar Buspirona 20-40 mg/día en 1\-2 tomas, sabiendo que tiene una latencia de hasta 2 semanas para su efecto ansiolítico. En caso de síntomas de ansiedad-angustia: se prefiere la Buspirona a las benzodiacepinas.

Para monitorizar la abstinencia puede recurrirse a la paraclínica:

* Hemograma: el VCM se normaliza a los 90 días de abstinencia.
* Funcional y enzimograma hepático: la GGT disminuye a un 50\% a los 7 días de abstinencia.

.Psicológico

Cuadro actual: entrevistas de apoyo con seguimiento estricto.

A largo plazo: podría ser de utilidad la TCC con uso de múltiples estrategias: terapia conductual de pareja, estrategias de refuerzo social, entrenamiento en autocontrol, entrenamiento en habilidades sociales y técnicas de control de estrés.

.Social

Entre fluctuaciones de semilucidez, ofrecer un marco orientador por parte de la familia. A largo plazo: conectar con grupos de autogestión como AA, que contribuye a la continentación y abstinencia del alcohólico, para lo cual es fundamental obtener la cooperación del paciente y la adquisición por parte de éste de conciencia de su alcoholismo como enfermedad crónica. También es fundamental la psicoeducación de la familia con respecto del alcoholismo y conexión con AlAnon, grupo de familiares de alcohólicos, de utilidad para dar apoyo, continentación y adecuado manejo de la culpa y autoestima.

\subsection*{Evolución y pronóstico}

Inmediato DAS: evolución favorable hacia la curación en pocos días, favorecido por la terapéutica. Más raramente:

* DT: 5-10\% mortalidad (infección, arritmias, disionías)
* Secuelas: transitorias (ideas fijas postoníricas), crónicas (psicosis alcohólica crónica)

Depende del éxito del tratamiento etiológico y sujeto a la reversibilidad de éste. Tiende a la curación sin secuelas. Puede ocurrir una fase de "despertar" luego de oscilaciones con ideas fijas postoníricas que desaparecen en días Pueden quedar ideas permanentes postoníricas (delirio de evocación de la experiencia confuso-onírica). Pueden ocurrir recaídas provocadas por factores etiológicos concurrentes (infecciones, emociones, par-tos) Alejado Depende del alcoholismo, enfermedad crónica con frecuentes recaídas. El pronóstico depende de la abstinencia. De no lograrse las complicaciones pueden ser:

. Orgánicas:
.. digestivas: hemorragias, esofagitis, gastritis, cirrosis, ulcus, pancreatitis
.. hematológicas: anemia
.. neurológicas: polineuritis, traumatismos, degeneración cerebelosa, miopatía
.. CV: HTA, miocardiopatía dilatada
. Psiquiátricos:
.. intoxicación aguda y sus complicaciones (accidentes, homicidios, suicidios)
.. abstinencia
.. déficit de tiamina Wernicke, Korsakoff
.. alucinosis
.. intoxicación crónica: Korsakoff, demencia, psicosis alcohólica.
. Sociales: deterioro FA SE SO LA Estado de vulnerabilidad encefálica que puede desencadenar nuevos episodios ante aparición de factores comórbidos con los consiguientes riesgos y complicaciones. Depende de la compensación del trastorno (por ej. CV).
\printbibliography[]