=== F60.3 Trastorno de la Personalidad Límite (Inestable)

==== Diagnóstico
Patrón general de inestabilidad de las relaciones interpersonales, autoimagen y afecto, con marcada impulsividad que comienza al inicio de la edad adulta y se presenta en una variedad de contextos. Tiene que tener 5 o más de:

* Esfuerzos excesivos para evitar abandonos reales o imaginados
* Patrón de inestabilidad de relaciones interpersonales, alternancia idealización-devaluación.
* Inestabilidad de la autoimagen o el sentido de sí mismo
* Impulsividad en al menos 2 áreas (gastos, sexo, drogas, comidas, conducción)
* Comportamiento suicida recurrente, amenazas de AE o autoagresiones
* Inestabilidad afectiviva, hiperreactividad del humor (horas - pocos días de duración)
* Sensación crónica de vacío
* Ira inapropiada / intensa / de difícil control.
* Ideación paranoide transitoria o síntomas disociativos severos

==== Psicopatología

Múltiples concepciones históricas:
* Perspectiva psicodinámica: lo percibe como un nivel de organización de la personalidad (Kernberg) comprendiendo varias patologías del carácter entre la neurosis y la psicosis.
* Perspectiva centrada en el humor (Akiskal): lo ve como un continuum con los trastornos afectivos.
* Perspectiva centada en el control de impulsos (Zanarini): centrado en la incapacidad de posponer la gratificación.
* Perspectiva centrada en el estrés postraumático (Herman, Kroll): basada en el alto porcentaje de historia de abuso sexual.

===== Variantes (Millon)
* Desanimado (Evitativo, Depresivo, Dependiente): dócil, sumiso, leal, humilde; se siente vulnerable y en riesgo constante; sentimientos de desesperanza, depresivos, indefensión e impotencia.
* Petulante (Negativista): negativista, impaciente, inquieto, testarudo, desafiante, hosco, pesimista y resentido; fácilmente se siente despreciado y desilusionado.
* Impulsivo (Histriónico, Antisocial): caprichoso, superficial, frívolo, distraible, frenético, seductor; temor a la pérdida, se vuelve agitado, sombrío e irritable; potencialmente suicida.
* Autodestructivo (Depresivo o Masoquista): vuelto hacia sí mismo, enojo autopunitivo; deterioro de comportamientos conformistas, deferentes e insinuantes; progresivamente nervioso y malhumorado; potencialmente suicida. 

===== Dominios funcionales y estructurales
.Dominios funcionales
* Comportamiento expresivo: espasmódico.
* Conducta interpersonal: paradojal
* Estilo cognitivo: caprichoso
* Mecanismo regulatorio: regresión
.Dominios estructurales:
* Autoimagen: incierta
* Representaciones objetales: incompatible
* Organización morfológica: escición
* Humor y temperamento: lábil.