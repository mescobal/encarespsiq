\chapter{Trastornos del humor: generalidades}
\faStopCircle En transición. Capítulo que será eliminado (se superpone con los 2 siguientes)
\section*{Notas clínicas}
Formas clínicas no contempladas en encares clásicos: trastornos subafectivos, trastrorno depresivo breve recurrente, trastorno esquizoafectivo, depreisón en esquizofrenia.
\subsection*{Trastornos subafectivos}
Clínica depresiva o cíclica moderada y crónica.
\subsubsection*{Distimia}
Trastorno persistente de al (-) 2 a de intensidad leve / moderada de inicio en adolescencia (depre doble) / ASECAD / curso persist o intermitente
* trast ambulatorio compatible c/ func social, pero estabilidad precaria por hiperocupación c/ incapacidad para disfrutar del ocio o act fliar o social lo que lleva a fricción marital / sentim y falta de gusto por la vida fuera de su trabajo / dinám conyugal alterada
* hay adaptación a los síntomas por resistencia cognitiva al cambio, lo otro no le resulta fliar
* se ven a sí mismos como "deprimidos desde siempre o desde el nacimiento"
* Akiskal sostiene la mayor validez de criterio B alternativo que agrupa sínt subjetivos . autoestima, autoconfianza, pesimismo, desesperanza, incapacidad, desinterés gralizado, aislamiento social, cansancio crónico, cavilaciones referidas al pasado, < productividad - efectividad, trast [ ] memoria, indecisión
* inicio + precoz > probab de complicarse C/ EDM
* pánico (10 %) / fob social (21%)
* 1/3 inicio < 12 a : "niños incapaces de disfrutar" / > \% de AF de humor
* "depresión existencial" / id suicida 31 \% / comportam suicida 24 %
* en suma : depresión de bajo gr y larga duración, fluctuante, experimentado como parte del YO habitual representando una acentuación de rasgos observados en temperamento depresivo (en apéndice de DSM IV)
\subsubsection*{Ciclotimia}
Caract hipomaníacas + subdepresivas
* en suma : disregulación bifásica c/ cambios endoreactivos de una fase a otra, cada fase dura pocos días x vez c/ infrecuente eutimia

.Cambios de categoría diagnóstica
Siempre depende de la intensidad del ep de exaltación de humor

* TD > : 9% a Bip II, 4% a Bip I
* Ciclot : 30% a Bip II, 5% a Bip I
* Bip II: 5-13% a Bip I

\subsection*{Trastorno esquizoafectivo}
EDM, maníaco, mixto + sínt de fase activa de esquizofrenia / sd delirante alucinatorio en dicho período de al (-) 2 sem en ausencia de sínt de humor prominentes / sínt de humor presentes en parte sustancial de fase activa y residual de los ep psicóticos

* tipo bipolar (maníaco o mixto) / tipo depresivo
* a > terreno esquizofr peor pronóstico

\subsection*{Trastorno depresivo pospsicótico en la esquizofrenia (NOS)}
* incidencia 25%
* peor pronóst / > probab de recaídas / > incidencia de suicidios que en esquizofr s/ este trast
* DD : fase residual (atimormia, sgs negativos) / ef 2º de APS / esquizoafectivo
* criterios de EDM en fase residual de esquizofr
\subsection*{Trastorno depresivo breve recidivante (NOS)}
De 2 días a 2 sem / deterioro pragmático en el episodio / vida más perturbada que TDM por frec cambios de humor

nunca hubo EDM ni existen criterios para distimia

\subsection*{Basados en etiología}

(debidos a enf médica / induc por sust [ c/ sínt depresivos, maníacos, mixtos] de inicio durante la intox o la abstinencia)

AP de enf médica o consumo, no de trast humor

no hay AF de humor

si coexisten hay implicancias pronósticas y terap

A. Inducido por sustancias : diagnóstico doble es lo más adecuado.
Sustancias: enmascaran síntomas afectivos.
Abuso de OH / sust : resistencia al tto

. bip I : 60 \% de dx doble
. bip II : 50\%

adicción : proceso de aprendizaje biológico

OH

. tipo I : como ansiolítico (trast de ansiedad o pánico)> 20-25 a / para manejo del stress / + ambiental / 60%
. tipo II : como estim ( trast de humor): < 20-25 a / comportam agresivo y antisocial / drogas / > tasa de depresión e IAE / > componente genético / búsqueda de ef euforizantes / 40\%

similitudes c/ bipolar: edad inicio / carga genética / > suicidio impulsividad violencia / variación estacional / "búsqueda de estím" / sd deficiencia de 5HT

* 3-4 sem detox para evaluar tto
* comorb peor resp
* beber social tb interfiere c/ tto
* ante dudas 3-4 sem detoxificación para determinar que es primario
* comorbidez entre ambos . peor resultado a tto convencional que cada grupo por separado

drogodependencias

patología dual

prevalencia:

ciclotimia 50\%OH

hombre + manía >40\% ingesta de OH

30\% consumo de coca asocian bipolar

bipolar patología de eje I con mayor riesgo de asociar

consumo de sust y OH

agrava evolución y empeora pronóstico

* + lentitud en recuperación
* + nº de internaciones

B debido a enf médica

Enfermedades que causan resistencia:

. endocrinopatías: tiroides, hipercortisolemia
. neoplasias / infecciones
. AVE / convulsiones / enfermedad cerebrovascular
. esclerosis múltiple / lupus

Demencia

(> de 65 años, primer episodio). Puede tratarse de un episodio de manía en el curso de un trastorno bipolar complicado por la instalación de una encefalopatía degenerativa a descartar por paraclínica ya que sabemos que en esos casos disminuye la respuesta a la medicación. El abandono de la medicación puede estar precedido por un episodio de omnipotencia maníaca.

T de P

secuencia temporal: ¿TP post afectivo?

BL ¿superposición? / 1/3 responden a AD (ISRS) o estabilizadores

en su labilidad afectiva /no en trast vinculares

preferible, ante duda, dx humor

fact pronósticos!!

diátesis de eje II puede tranf en trast de eje I (puede empeorar por iatrogenia)

anastomosis humor / personalidad

EJE IV

Muerte de progenitor x < de 11a puede llevar a desarrollo post / pérd de cónyugue está vinculado al inicio del ep depresivo al reactivar aquella pérd de la infancia

eventos vitales del episodio y estressores crónicos mantenidos

disfunción fliar : incide en recaídas, readaptación y recuperación


\subsection*{Diagnóstico diferencial}
PLANTEO DIFERENCIAL c/otro tipo de episodio

1. mixto
2. inducido por sust
3. enf orgánica

DX DIFERENCIALES DENTRO DE OTRAS PATOLOGÍAS MAYORES

Esquizofrenia

* curso longitudinal
* cuadro actual
* remisión incompleta / sínt (-)
* anhedonia / aplanamiento afectivo
* fuga de ideas / pensamiento desorganizado
* schneiderianos no es patognomónico de esquizofrenia

en corte transv puede ser difícil diferenciar de esquizofrenia / a medida que progresa el episodio va aumentando el componente psicótico : est I y II (disforia-manía) / est III (imposible diferenciar) / est II y I ( disforia-manía)

basarse en : AF / func premórbido / carácter previo / curso de episodios

suicidio no equivale a vulnerabilidad genética para trast de humor

Ps breves

Tr adaptativo c/ est de ánimo depresivo ( < 3 m post a stressante c/ resp > a la esperable y deterioro pragmático)

basar diferencial en :

consec interpersonales del func

quiebre del yo premórbido usual

cualitativamente diferente al pesar normal u otras reacc comprensibles

s/t : recurrencia / AF

Tr ansiedad : presente durante, como precursor / se sugiere, al (-) en algunos casos, diátesis común / comorbilidad frec. importa c/ crisis de pánico ya que puede agregar morbilidad : abuso de OH y drogas / si empieza por ansiedad y luego instala gradualmente la depre : tto ansiolítico desde el comienzo

Tr control impulsos: imp crónico y circunscripto

en bipolar: episódico y generalizado.

. TCA
. TDAH: 3 síntomas similares a los criterios de  manía.

¿bipolares precoces?

\subsection*{Etiopatogenia y psicopatología}
\subsubsection*{Etiopatogenia}
\paragraph{Biológico}

1. alt del sueño (en + ó -), apetito, impulso sexual y cambios endócrinos, inmunológicos y cronobiológicos (alt del ritmo circadiano) hablan de disregulación en SNC (sist límbico, GB e hipotálamo) como sustrato etiopatogénico neuroanatómico
2. MUY genético s/t bipolar / reactivado por el ambiente : kindling (ELECTROFISIOLÓGICO) : estím subumbrales reiterados llegado un momento originan un potencial de acción

. 1º epis : hay desencadenante en 60\%
. 2º epis : 30\% / 3º epis 20\%
. luego del 4º : no hay evidencias
. fenóm vinculado al uso de cocaína
. > jóvenes < necesidad de estímulo / > + resist por lo que se ha planteado la profilaxis del kindling en ptes c/ vulnerabilidad genética mediante CBZ que luego se suspendería / el estím no crece la respuesta sí
. post menopausia = tasa de depre que en hombres / > nº de depre en mujeres es entre post pubertad y pre menopausia / el reemplazo hormonal con estrógenos tiene otras ventajas vinculadas al trast posmenopáusico pp dicho (ej osteoporosis)

3. bioquímico: desequilibrio a/n de NT con hipersensibilidad en receptores postsinápticos beta adrenérgicos y 5 HT2
4. constitucionales: hábito pícnico de Kretschemer.
5. neuroendócrinos : cortisol / tiroides

.PSICOLOGICO Y SOCIAL

Yo débil con dificultad para superar pérdidas y para adaptarse a situaciones nuevas. Sobre un terreno de vulnerabilidad encontramos factores psicosociales actuando como desencadenantes (pérdidas, dificultades interpersonales, pérdida de roles laborales, pérdida de posición social). Puede intrincarse con desencadenante biológico (abandono de la medicación).

Entorno fliar deprivado y perturbado

Estresores - crónicos : deprivación financiera, dificultades interpersonales (¿1º o 2º al trast humor?), amenaza persistente a la seguridad (barrios amenazantes) / erosión de soporte social puede complicar evolución favorable del episodio depresivo (muerte o enf de 3º significativos) / cotidianos : manejo de la economía del hogar / alt vinculares c/ vecinos

adolescencia :

pérd y separaciones : estabilidad de imagen corporal / bisex potencial (omnipotencia) / separación del vínculo infantil a objetos edípicos

manejo de agresividad (excitac pulsional) y culpabilidad consiguiente :

defensas psíq y comportamentales

1. retorno sobre sí de la agresividad : cond peligrosas / equivalentes suicidas / id de AE / IAE
2. inhibición y pasividad : repliegue sobre sí mismo / desinterés / clinofilia / enclaustrado en habitación
3. huída y distanciamiento

Transf del equilibrio entre investiduras objetales (intereses por el mundo ext) vs investiduras narcicistas (intereses por el mundo int) : ruptura c/ intereses de la infancia (ya que los ve como sumisión respecto a imágenes edípicas) y al mismo tpo sobre la investidura de sí mismo (preguntas ontológicas : ¿quién soy yo?) / oscilaciones en la idealización de sí : de narcicismo exacerbado a profunda minusvalía c/ sentim de vacío c/ eventual creación de ideal intermediario (idealizac de grupo : religioso, filosófico, deportivo, cultural, dietética), adhesión masiva, absoluta y a/v sin crítica / trabajo que finalmente tendría que concluir con la conformación del IDEAL DEL YO.

Por lo tanto se mantienen los 3 ejes evocados en psicopatología de cualquier depresión : 1) pérd objetales reales o fantasmáticas / 2) agresividad y culpabilidad derivada de ella / 3) narcicismo y reconsideración del sist de idealización

duelo :

3 etapas: 1) negación / 2) ira, bronca, reivindicación / 3) resignación, aceptación

PATOLÓGICO : > intensidad / duración > 6 m / aparición de fenóm patológicos :

negación masiva : x ej fenóm seudoperceptivos al servicio de la negación

proyección masiva : x ej denuncias a médicos

el pte puede quedar en etapas previas a la de aceptación

a/v el duelo queda trunco por sustitución del padre / hijo por esposo muy > o <

vejez:

prevalencia 25-50%

< NSE / pérd s/t cónyugue / pat médica concomitante / aislamiento social

pérd de roles c/ claudicación de defensas psicológicas

puerperio:

tristeza posparto : 50-80% / breve : 1-4 días / labilidad emocional, llanto fácil, sentim de incapacidad para cuidar al niño / cambios en la dinámica familiar

psicosis posparto : 0,5 - 2 por mil (depre-manía) / se desarrolla en 24-72 hs pero riesgo s/t 1º mes, se habla hasta de 12 meses / infanticidio > 10% / riesgo de recaída en futuros embarazos

depres posparto no psicótica : 10-15% / dura entre 2 sem y 2 meses / riesgo 1º 6 meses s/t en 1º y 2º / infanticidio 5%

\subsubsection*{PSICOPATOLOGIA}

Binswanger: modalidad regresiva global con modificación de la estructura temporal de la vida psíquica con desencadenamiento de los impulsos.

Análisis estructural de Ey: comporta un aspecto negativo (regresivo o deficitario) y aspecto positivo, de liberación de instancias inferiores. Existe una desestructuración ético-temporal de conciencia (de 1º grado), con pérdida de la capacidad de adaptación y moderación a las exigencias del presente.

Psicoanálisis: la crisis de manía es interpretada como una regresión súbita a los estadíos infantiles del desarrollo psicosexual, anteriores a toda frustración exterior con liberación de las pulsiones orales pregenitales. Sería un mecanismo de defensa psicótico de negación de la pérdida y de la melancolía de fondo, de la cual sería contracara.

Teo cognitivo conductual: indefensión aprendida

\subsection*{TRATAMIENTO}

IMPORTANTE : se revierte la depresión pero no se trata la vulnerabilidad

EDM : 3 ó + : TDM recurrente : tto de mantenimiento permanente

2 : + AF / instalac precoz / recidiva en 1º año : IDEM (seudounip)

epis único severo o inicio súbito en últ 3 años

¡ plantear pasar a Li !

joven 1º episodio : susp al año asintomático

>50-60 a / 3º epis / AF / 2 ó + en > 40 a : de por vida

desesperanza crónica influye en el nº de suicidios : PST reduce riesgo

RESISTENCIA : falla en remisión completa en 3 ensayos (incluyendo ISRS y TCA) por 12 sem c/u a la máx dosis tolerada

ensayo adecuado (para hablar de resistencia)

Dx adecuado

AD apropiado

dosis adecuada

nivel plasmát óptimo

durac adecuada (12 sem)

buen cumplimiento

tto sobre OH y sust

alternativas de potenciación

. evaluar fact interferencia (lo antedicho + enf médicas + otro fco)
. agregar ag endócrinos (h tiroidea -T3 entre 25 a 50 microgr- mejora en 10 días / estrógenos - en peri o postmenop)
..func tiroidea: T3 "bajos" dentro de lo normal asoc c/ recaídas, T4 "bajos" dentro de lo normal asoc c/ letargia y alt cognitiva, se agrega hormona aunque esté en rango normal (1/3 inf)
. Li (resp en 10 días / potencia sist serotononérgico)
. cambiar a otra clase de AD ( de TCA a IMAO ó a ISRS / de 2ª generac a otro de 2ª (ej : de SSRI a Bupropión o venlafaxina)

advertir sobre evoluc despareja lo que por otro lado es sg de que el AD está funcionando / en este lapso son frec los abandonos de medicac ya que al recaer el pte se frustra

una vez hallada terap adecuada para el ep agudo debe ser continuada por 6-9 meses , período en el cual la vulnerabilidad de recaída es elevada (50%)

si existen sínt residuales (< sueño, anergia, < [ ], despertar precoz leve) aumentar agresividad terap con aumento de dosis o potenciación

reducción gradual de dosis puede llevar a incremento discreto de sínt depre obliga a continuar terap a = dosis / descenso gradual para evitar sínt < de abstinencia

predictor clínico eficaz : curso de ep anteriores en cuanto a tpo de tto y probables recaídas

FASES DEL TTO :

agudo :dominar el cuadro actual

mantenimiento : evitar recaída de epis actual (6-12 m a dosis plenas)

profilaxis : prevenir recurrencia luego de 6 m de remisión completa / se plantea según nº de epis previos / severidad de éstos / durac de intervalo asintomático / presencia de sínt entre los epis / evolución de episodios

. TTOS DE 1ª LÍNEA:

distimia : isrs (sertralina : náusea,dispepsia, diarrea, hiperdefecación), bupropión (no provoca alt sex), Venlafaxina (cefaleas, náuseas, HTA)

EDM leve-moderada : igual

severa s/t c/ melancolía : TCA o ECT

severa c/ atípicos : IMAO / Fluoxetina hasta 40-80 mg

. DISTIMIA : dosis más elevadas que para TDM : MOCLOBEMIDA : 600 MG / sertralina puede llegar a 150 - 200 mg

PST : aceptarse a sí mismos / optimismo razonable / mantener compliance / manejo de conflictos acumulativos (FASESOLA) / cambio a personalidad postdepresiva / movilizar destrezas y recursos

1) ISRS

fluoxetina

perfil de ef 2º benignos

NO asociado a : ganancia de peso / ht ortostática / sínt anticolinérgicos / letalidad por sobredosis

EF 2º : inquietud e insomnio / cefalea / temblor / molestias GI / disfunción sexual

wash out de 6 sem previo a IMAO (por norfloxetina, vida 1/2 de eliminac de 5-7 días)

por cit 450 aumenta la [ ] de TCA un montón al asociar

sertralina

< vida 1/2 que floxetina

no tiene metabolito duradero

EF 2º : GI (diarrea -Hdefecación- / náusea / dispepsia) SEX : retardo eyac / anorgasmia / disminuc libido / disfunc eréctil

venlafaxina

perfil mixto de acción : inh recap de serot / NA / DAM en < grado

se ha comunicado 40 % de respuesta en quienes ha fallado el tto (incluso IMAO e ECT)

vida 1/2 de eliminac 5-6 hs por ende 2/3 tomas diarias

EF 2º : náusea, sudoración, sedación, boca seca. disfunción sexual / excepto náusea el resto son dosis dependientes y se pueden atenuar a lo largo del tiempo o con reducc de las dosis

aumento de PA diastólica : 3% con menos de 100 mg /d

18 % de ptes c/ > de 300 mg/día

ojo en HTA !!!!!

dosage: 25 mg x 3 inicio

aumentar 75 mg /d cada 4 días hasta 225 mg/d

se puede llegar hasta 125 mg x 3 como dosis máx

FACT DE RIESGO PARA RESIST AL TTO

trast orgánico

uso de OH u otras sust

trast personalidad

stressores múltiples pre e intra epis

inicio tardío de tto adecuado

bipolar II

depre doble

ancianos

aspectos de personalidad (que pueden llevar a no compliance)

narcicistas : "no soy enfermo"

paranoides : "me van a dañar"

Syoico : "no necesito ayuda"

TRAST BIPOLAR

ante EDM : revaluar litemia / func tiroidea / eventos vitales

>func tiroidea / > litemia a 1,2 / litio + AD

s/ tener en cuenta bipolares inducción de manía : ISRS 3,7% / TCA 11,2%

depre bipolar : 1º Bupropión / 2º ISRS / 3º IMAO

Bupropión - ventajas : activante, no aumento de peso, no disfunción sexual, no alt del sueño, < tasa de viraje / ISRS al dar insomnio aumentan riesgo de viraje

APS atípicos : olanzapina -aumento de peso- y risperidona (70% de resp o mejoría) / clozapina efectiva pero de uso + complicado, > sedación que olanzapina

ante mixto / CR : retirar AD y NL / Li + cbz / valproato

edad de inicio

18 a: sínt

22a: 1º tto

28a: 1º episodio

60% inicio depresivo

joven + sínt psicóticos c/mejoría rápida: predictor de bipolar

frec episodios

1- 50% posib de recaída

4- 70%

5- 90%

media de 10 epis

durac período de remisión

se acorta con sucesión de episodios pudiendo llegar a CR (¿kindling?)

en inicio tardío es más corto (no confundir con peor pronóstico)

consideraciones del tratamiento

Directivas: cuadro actual - a largo plazo (compensar enfermedad de fondo, profilaxis de recaídas, evitar complicaciones).

OBJETIVOS : cura del episodio (no hay cura del trastorno)

< morbimortalidad / < frecuencia y severidad / < consecuencias psicosociales / mejorar funcionamiento interepisódico

CUADRO ACTUAL

Internaremos al paciente en hospital psiquiátrico, de ser posible con aprobación del paciente (de lo contrario será compulsiva). Justificamos por:

evitar complicaciones

disminuir duración del acceso

actos ML, heteroagresividad, dilapidación de bienes, ultraje público al pudor

Protegiendo al paciente de sí mismos y de los demás (y viceversa).

Habitación aislada, evitando estímulos y el contacto con otros enfermos a quienes puede transmitir su excitación.

Evitaremos medidas de contención a menos que sean imprescindibles, con riesgo de su integridad física o de terceros, de recurrir a ellas se llevarán a cabo por personal entrenado según normas del MSP.

Realizaremos estrictos controles de pulso, PA, temperatura e hidratación.

Monitorizaremos diariamente la EPM, sueño y síntomas psicóticos. Estaremos alertas a la inversión del humor.

BIOLOGICO

Haloperidol 5 mg i/m H8 - H20, por su efecto antimaníaco inmediato, actuando sobre la EPM y la ideación megalomaníaca (síntomas psicóticos). Ajustaremos la dosis, pudiendo llegar a 15 mg/día si la mejoría clínica no es satisfactoria. Estaremos alertas a efectos secundarios extrapiramidales. Si aparecen (rigidez, rueda dentada, bradiquinesia, temblor) concentraremos la dosis en la noche ya que no se producen durante el sueño. Si con esa medida no podemos controlarlo, agregaremos un antiparkinsoniano de síntesis como el Biperideno a dosis de 2 mg v/o H8 y H14. Si es menor de 35 años, sexo masculino lo agregamos de entrada por mayor riesgo de presentar distonías agudas. Si aparecen: 5 mg i/m con lo que ceden inmediatamente.

Sedaremos al paciente con Clonazepam 2 mg v/o c/8 hs que actúa como estabilizador del humor, combatiendo la irritabilidad, impulsividad y disforia. Iremos aumentando hasta obtener el efecto deseado pudiendo llegar a 16 mg/día (pasar a Levomepromazina previo a ECT ya que Clonazepam el umbral convulsivo). En caso de ansiedad psicótica MIDI/agitación: Levomepromazina 25 mg c/8 i/m.

Para combatir el insomnio: Flunitrazepam 2 mg H20 v/o.

Indicaremos desde el inicio carbonato de Litio que pese a su latencia de 8-10 días para el inicio de su acción, proporciona un efecto antimaníaco más específico, además de ser estabilizador del humor y profiláctico de recidivas. Empezamos con 300 mg v/o c/8 hs con las comidas, probando tolerancia ya que al inicio son frecuentes los trastornos digestivos leves que al igual que la sintomatologia neurológica inespecífica (letargia, fatiga, debilidad muscular y temblor fino distal), polidipsia y poliuria son todos fenómenos reversibles y transitorios. Indicaremos abundantes líquidos v/o. Estaremos atentos a la aparición de estos síntomas. Controlaremos la aparición de estos síntomas. Controlaremos la aparición de signos incipientes de toxicidad (ataxia, temblor grueso, disartria, fasciculaciones).


PSICOLOGICO

Durante la internación: entrevistas diarias de apoyo y continentación, evolución del delirio, etc. Vínculo, etc.

SOCIAL

Contacto con familiares, psicoeducación, alianza terapéutica.

A LARGO PLAZO

Mantendremos a largo plazo el Carbonato de Litio a las mismas dosis con que se obtuvo mejoría. Efectuaremos controles seriados en policlínica inicialmente semanales, que luego se irán espaciando.

Control de litemia cada 15 días el primer semestre, luego mensualmente. Control clínico y paraclínico del polo renal, tiroideo, ionograma, de las complicaciones posibles, así como de las intercurencias de enfermedades (nefropatía, diuréticos) que alteren la litemia pudiendo favorecer la intoxicación, lo que nos obligará a un monitoreo más estricto o eventualmente a retirar el Litio.

De no ser satisfactoria la respuesta clínica o de aparecer CI, valoraremos el agregado o la sustitución por Carbamazepina a dosis de 200 mg v/o c/12 que iremos aumentando a 1200-1400 mg/día con control de función hepática y hematológica (siendo la concentración terapéutica de 4-12 µg/ml de plasma).

PSICOSOCIAL

Entrevistas reiteradas, afianzar el vínculo. Psicoeducación familiar. Evitar abandono de medicación. Signos precoces de descompensación, diagnóstico y tratamiento instituido, importancia de controles y adhesión al tratamiento.

Mujer en edad genital activa: derivación a planificación familiar (potencial teratogénico del Litio y la Carbamazepina).

manejo (mejor que tto ya que abarca todos los aspectos)

establecer alianza terapéutica

carta de vida : monitoriza evoluc

psicoeducación contribuye a compliance al tto

50 % dejan en algún momento el tto

"cicatrices" de la enf que deben trabajarse en la terapia (pragmáticas, autoestima)

psicot dinám es removedora del pasado : re-kindling

mantener integridad circadiana ya que bipolar es un trast en el reloj biológico / promover act diarias y sueño (alt en ciclo luz-oscuridad lo que implica buena resp a luminoterapia y deprivación de sueño)

promover comprensión de efecto psicosocial ("pérd de seducción")

sínt señal - x lo gral insomnio

manejo

mejor combinar dosis bajas de varios estab que dosis altas de uno solo

mantener func tiroidea en rangos altos normales

mantener óptimos ritmos biológicos

no comenzar ni interrumpir bruscam el tto

PST : (+ medicac reduce nº de recaídas) / psicoeducac, fliar, comportamental, grupos de autoayuda / en bipolar dirigida a técnicas de manejo del stress y cumplim del tto, eventos socio personales que gatillan recaída y consec sociales y personales

flexible

manejo colaborativo

comprender consec de no compliance

psicot de pérdidas

REALISTAS : < euforia / hipersexualidad

SIMBÓLICAS : omnipotencia / "ser especial"

NO REALISTAS : proyecc del déficit en la medicación

trabajar c/ la pareja

estabilizar ritmo social

hombre, joven, pocos episodios : perfil de bajo cumplimiento

razones de poca adhesión al tto en bipolar : estado de excitac recurrente es reforzante / mín sufrimiento subjetivo / severo trast del insight

pasaje a la cronicidad del epis maníaco : no compliance / OH o drogas episódico o crónico / patología cerebral / ojo c/ error DX con esquizofrenia

LITIO

80% de respuesta en manía aguda

acc serotoninérgica

si hay en epis agudo un importante deterioro, intensa psicosis o agresividad debe ser suplementado en fases tempranas (NL-BZD)

MANTENIMIENTO Y PROFILAXIS

20 % de no respuesta

60 % respuesta completa

20 % respuesta parcial

los resultados c/ Li mejoran c/ el tiempo, mejor en el 2º año

cada año agrega 1% de riesgo de hipotiroidismo

unipolar c/ múltiples episodios : mejor profilaxis c/ Li que con ADT

patrones predictores de resp a Li

MDE- Li 80% / DME- Li 33% / circular: CR- 28%, lentos: 77% / s/patrón 80%

ptac clásica c/ euforia más que severa o disfórica

AP de pocos episodios

AF de trast humor en fliares de 1º gr

EFICACIA

BP s/ complicaciones : 60-80%

mixta : 30-40%

CR : 20-30%

1 episodio 80%

2 epis 50%

5 epis 37%

interrup abrupta de 1-2 días > riesgo de recaídas en 24 meses

refractariedad inducida por abandono

la interrupc de Li puede generar refractariedad tanto al Li como a otros ttos

FACT LIMITANTES DE PROFILAXIS (modifican respuesta)

niveles plasmáticos

clínica

CR: + htiroidismo + uso de ADT

mixto

TP/sust

genio evolutivo (3 epis en últ 3 años)

comorbilidad

psicosocial

EFECTOS SECUNDARIOS

pico plasmático : temblor fino (beta bloq)

relac c/ dosis : poliuria, polidipsia, edema (diuréticos) / > peso, alt cognitivas, sedación, letargia,

alt coordinación / acné (ATB tópicos) / alt GI (con comidas)

CV : trast repolarización en ECG

renal : < capac en [ ] orina por disminuc de resp renal a ADH (poliuria y/o polidipsia) pudiendo llevar a diabetes insípida nefrógena / tto : dosis única al acostarse, si persiste, aumentar consumo de agua disminuyendo consumo proteico; si persiste agregar hidroclorotiazida (25 a 75 mg) y bajar Li a la mitad para compensar aumento de la reabsorción / tb amilorida ( ahorrador de K)

tiroides : hipotiroidismo en 5 a 35% / > frec en mujeres post 6-18 m de tto con Li y puede asociarse a CR / gralm reversible al susp Li pero no es contraindicación / adm Liotironina 20 - 60 mg/día / riesgo de depresión y de CR

Li tiene ef antiinsulínico

ojo c/ diabetes : x lo gral se puede manejar c/ dieta

ojo c/ intolerancia a glucosa y aumento de peso

ojo c/ inestab de la glicemia

asoc frec entre diabetes y trast de humor

Li reduce 8 veces el riesgo de suicidio

depre en trast bipolar que toma Li

LEVE :

mayor nº de consultas

aumentar Li hasta 1,2 meq

maximizar func tiroidea

agregar otro estabilizador (antes que AD)

AD : bupropión / ISRS

MÁS GRAVES : IMAO en altas dosis

OPTIMIZAR TRATAMIENTO

si se necesitan grandes dosis de Li para mantenim : reducir Li y agregar anticonvulsivante

ajustes lentos

mantener niveles de tiroides alto o supranormal (dosis mínima eficaz) / sustitutivo (si hay htiroidismo) - potenciador (llevando al rango máx normal)

dosis única para minimizar ef cognitivos (1-2 gr B12)

psicoeducación OH, drogas

stress ambiental

INICIO CON ANTICONVULSIVANTES

* CR / manía mixta / AP mala resp al LI / manía 2ª / sust abuso
* Li + valcote < frec de recaídas y asociación menos compleja

.CBZ

* refract a Li
* dosis inicio : 200 c/ 8 hs y aumentar hasta 1200 mg [ 6-12 microgr / ml ]
* ef 2º : dosis depend : leuco y trombopenia leves, > enz hepáticas, hipoNa, diplopía, ataxia, fatiga, visión borrosa, temblor,> peso, erupc cutáneas, náuseas, vóm, retención de líquidos / idiosincrásicos : corazón (ef quinidinosímil, control en cardiópatas), agranulocitosis, anemia aplásica, insuf hepática
* hemograma (c/ lám y recuento plaquetario) y funcional hepático : c/ 2 sem x 2 meses y luego c/ 3 meses ya que discrasias y hepatopatías tienen lugar s/t en 1º 3 a 6 meses
* inductor de Cit P450 : induce su pp metabolismo y otros metabolismos hepáticos, por lo tanto múltiples interacciones
* CBZ no mantiene sus efectos c/ el tiempo, puede disminuir en 1 ó 2 años

\subsection*{Evolución y pronóstico}

===== TDM

* 1º epis 50% de 2º / 50% de recaída en 1º año
* 2º " 75% de 3º
* 3º " 90 % de otros (practicamente crónico)

Factores de recurrencia :

* AP EDM
* distimia previa
* otro trastornos (no de humor)
* enf médica gral

mortalidad 2, 3 veces pob gral (suicidio, enf CV / neo)

adicción : > epis mixtos / resist al litio / respuesta más lenta

curso y evolución

1º episodio depresivo en joven c/ mucha inhibición - estupor puede predecir curso bipolar

en caso de recurrir a ECT: trastornos mnésicos leves

siempre latente el riesgo de recaídas (sigue pauta individual)

Estadísticamente: 70% mejoran con Litio, 30% con Carbamazepina y 1% con combinación. El 80% se controla en forma adecuada. 20% de difícil manejo.

concordancia entre < edad de comienzo y > sínt psicóticos

la media en ptes s/ tto es de 18 episodios / unipolar 7 episodios

intervalo libre tiende a disminuir : 1º-2º : 3a / 2º-3º : 2a / 3º-4º : 1a

enf crónica c/remisiones y exacerbaciones

carácter crónico y recidivante

normalidad interepisódica relativa

tto puede modificar curso

. positiva
. (-) ADT: ciclación rápida / viraje a manía

predictores de curso

CLÍNICOS

CR

patrón estacional - peor (pero se puede instrumentar profilaxis)

inicio postparto - mejor

sínt psicóticos - peor (s/t incongruentes)

proximidad con último episodio - peor

Bip II - > epis que I pero (-) graves

AF - a (+) peor

PSICOSOCIAL : life events

suicidio

1.suicidio frustrado
2.IAE
3.parasuicidio

criterios de clasificación

gravedad médica

método : violento/no violento (cortes, psicof)

intencionalidad

posibilidad de rescate

repercusiones : físicas/psicosociales

fact riesgo

1. trast mental: ep depresivo 50% de suicidio total - parasuicidio > TP, adaptativos
2. sexo: 1 y 2 + frec en hombres, 3 + frec en mujeres
3. edad: + viejos
4. enf orgánicas
5. E. civil
6. life events
7. genético-biológico (5 HT)

Otros datos

mayor riesgo en 12 meses post inicio de depresión

fase depresiva post exaltac s/ eutimia

ESTADO MIXTO: grave por coexistencia de sent depresivos en pte

desinhibido

diferentes poblac para IAE (mujeres x 4) y suicidio (hombres / OH / bipolares) / suicidio tiene genética propia

predictores de suicidio

al año siguiente : anhedonia / ansiedad severa psíquica / crisis pánico / abuso de OH o drogas (en intoxicac o abstinencia)

entre 1 y 5a : desesperanza severa / ideación suicida / ansiedad somática / AP de IAE

GOODWIN

hombres empiezan x ep maníaco / mujeres x ep depresivo (x lo gral)

manía unipolar < 2%

infancia y adolesc : + delirios y consumo / > irritab que júbilo / peor respuesta / + epis mixtos / a inicio más precoz > probabilidad de responder a anticonvulsivantes, < al Li / diferencial : TDAH

recaídas : 81% c/placebo - 33%c/Li (ahora no tan buena)

recaídas en unipolar recurrente : 50% más c/AD que los ttados con Li

valproato : trast cognitivos / alt de memoria / caída de cabello / alt hepáticas

NOTAS

Latencia de los antidepresivos para el tratamiento del EDM: 2 semanas.

La venlafaxina a dosis altas parece tener una latencia menor (CITA).

Predictores de riesgo de inicio de un trastorno bipolar ante un primer EDM:

* Historia familiar de TB
* Aparición antes de los 25 años
* Inicio en el posparto
* Hipomanía farmacológica inducida por el antidepresivo
* Presencia de síntomas psicóticos
* Hipersomnia y/o inhibición psicomotriz

En pacientes con depresiones recurrentes plantear uso de antirrecurrencial / estabilizadores.

DISTIMIA: la combinación de psicoterapia + medicación es más eficaz que la medicación sola (CITA)
Depresión doble: la medicación AD consigue no solo la remisión del EDM sino la de la distimia.
