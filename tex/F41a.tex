== Trastorno de Pánico

NOTA: separo Trastorno de pánico de Trastorno de ansiedad generalizada. Voy pasando de a poco TAG al encare correspondiente

=== Encare
Motivo de consulta: síntomas somáticos de diferentes tipos (cardíacos, digestivos), derivaciones desde otras especialidades (cardiólogo, neurólogo). Puede consultar por alguna comorbilidad.

==== Agrupación sindromática

===== Síndrome de ansiedad-angustia

Entendemos por ansiedad una experiencia displacentera con un componente psíquico (afectivo, cognitivo: miedo a perder el control, a enloquecer o a morir, desrealización / despersonalización) y uno somático (psicomotriz, neurovegetativo: palpitaciones, taquicardia, opresión torácica, sudor, escalofríos, disnea, temblores, parestesias, vértigos, mareos, disfagia, náuseas, malestar abdominal). Está centrada en el sentimiento de aprensión causado por la anticipación de un peligro (interno, externo, real o imaginado). Puede presentarse como episodio crítico o como estado más o menos persistente:

.Crisis de angustia
Definir crisis de angustia. De X evolución, de inicio brusco, paroxísticas, inesperadas o predispuestas por determinadas situaciones (si aparecen determinadas casi invariablemente por una situación, considerar el diagnóstico de fobia). Con rápido aumento de intensidad, de duración breve.  Crisis que se presentan clínicamente con X (lista de síntomas). Crisis que han tenido una evolución X (edad de inicio, pauta, frecuencia, repercusión, cambio evolutivo).

.Ansiedad basal
Fondo de ansiedad difusa y permanente. Dado por:

* Fondo permanente de ansiedad y preocupación no realista o excesiva sobre una gama amplia de acontecimientos.
* Tensión motora: temblor, sacudidas, tensión o dolor muscular, inquietud, fatigabilidad excesiva, hiperactividad vegetativa (ahogo, palpitaciones, sudoración, boca seca, mareos, sensación de inestabilidad, trastornos gastrointestinales, sofocaciones, escalofríos, micción frecuente, "nudo en la garganta", cefaleas).
* Hipervigilancia: sentirse en peligro, alarma exagerada, dificultad para concentrarse, insomnio, irritabilidad.
* Alteraciones de las conductas basales, alteraciones somáticas funcionales (anorexia, disfunción sexual, insomnio).

TIP: La semiología puede dividirse en componentes: psíquicos (afectivo / cognitivo) y somáticos (psicomotriz / neurovegetativo).

===== Síndrome fóbico

Entendiendo por fobia X. Que aparece (o no) a partir de las crisis de angustia configurando una Agorafobia (definirla).

===== Síndrome depresivo

Vinculado a la ansiedad. Marcar la cronología (primero ansiedad, luego depresión).

===== Otros síndromes

* Síndrome disociativo
* Síndrome conversivo: motor (como crisis o más permanentes), sensorial, sensitivo, neurovegetativo.
* Síndrome obsesivo-compulsivo
* Síndrome de despersonalización: puede hablar de la gravedad de la ansiedad-angustia.

==== Personalidad y nivel

Nivel: cualquiera.

Personalidad: rasgos evitativos, conflictividad infantil, introversión, timidez, rasgos de cualquier serie, síntomas de cualquier serie. Dificultad para superar e integrar separaciones y pérdidas. Dificultades para adapta|rse a situaciones nuevas. Dificultad para manejar la agresividad.

Destacar: buena relación interpersonal, pedido de ayuda.

==== Diagnóstico positivo

===== Nosografía clásica

.Neurosis
icon:paste[] Fragmentos: Neurosis.

.Neurosis de angustia
Ya que el cuadro está centrado en la ansiedad angustia. Y si bien existen elementos de otras series (sobre todo fóbicos), éstos no bastan para yugular la angustia. No llegando a estructurarse en una neurosis fóbica (no existe organización en fobia única, no existe evitación ni conductas tranquilizantes).

.Descompensada
Por

* Exacerbación de la sintomatología de ansiedad-angustia con aparición de episodios críticos de angustia y utilización de mecanismos de defensa tipo fóbicos, intento fallido por el momento de estructuración en neurosis fóbica.
* Síndrome depresivo con elementos disfóricos como consecuencia de un desencadenante (reactivo)... (depresión neurótica).

.Causa de descompensación

Conflicto psicosocial, estrés ambiental, separaciones, situaciones de pérdida, situaciones de competencia (agresividad culpabilizada), frustraciones de orden sexual, alteración de relaciones interpersonales, pérdida de apoyos sociales.

.Diagnóstico de gravedad:

* Gravedad de las crisis de angustia: por cantidad de crisis (grave > 8 al mes), según períodos intercríticos.
* Gravedad de la evitación agorafóbica: según limitaciones en la vida cotidiana y uso de acompañantes al salir.

===== DSM IV - CIE-10

* Trastorno de angustia con Agorafobia.
* Trastorno de angustia sin Agorafobia.
* Trastorno de ansiedad generalizada.
* Trastorno de ansiedad no especificado.
* Trastornos adaptativos

.Diagnóstico de Crisis de Pánico
No codifica aislado.

Aparición aislada de miedo o malestar intensos, de inicio brusco, con expresión máxima en los primeros 10 minutos, con 4 o más de: palpitaciones, sudoración, temblores, disnea, sensación de atragantarse, opresión torácica, molestias digestivas, mareos / inestabilidad / desmayo, desrealización / despersonalización, miedo a enloquecer, miedo a morir, parestesias, escalofríos / sofocaciones.

.Diagnóstico de Agorafobia
No codifica aislado.Ver F40.

.F40.01 Trastorno de pánico con Agorafobia
Requiere:

. Crisis de pánico inesperadas recidivantes (al menos 2) seguidas de 1 mes o más de ansiedad anticipatoria o preocupación por las crisis y sus consecuencias o cambio del comportamiento vinculado a las crisis.
. Presencia de Agorafobia.
. Descartar sustancias y enfermedad médica.
. No se explica mejor por otro trastorno mental (excluir sobre todo fobias).

.F41.0 Trastorno de pánico (sin Agorafobia)
Requiere: lo mismo que F40.01, pero SIN Agorafobia

.F41.1 Trastorno de ansiedad generalizada
Requiere:

. Ansiedad y preocupación excesivas (expectación aprensiva) sobre una amplia gama de acontecimientos o actividades por más de 6 meses.
. Dificultades para controlar el estado de preocupación.
. Asociada a 3 o más síntomas de: inquietud / impaciencia; fatigabilidad; dificultades para concentrarse; irritabilidad; tensión muscular; alteraciones del sueño.
. La ansiedad no se limita a ser sintomática a otro trastorno del eje I (no vinculado a crisis de pánico, fobia social, trastorno de somatización, etc.).
. Alteración de pragmatismos.
. Descartar sustancias, enfermedad médica y trastornos psicóticos.

.F43.xx Trastornos adaptativos
Requiere:

. Aparición de síntomas emocionales o comportamentales en respuesta a un estresante identificable (aparece dentro de los 3 meses de sucedido el evento).
. Malestar mayor al esperable para el desencadenante, con afectación de pragmatismos.
. No cumple criterios para otro trastorno del eje I (descartar sobre todo TAG, EDM y TEPT) y no es una exacerbación de otro trastorno del eje I.
. No es una reacción de duelo.
. Una vez que cesa el estresante los síntomas no permanecen más de 6 meses.

.Especificadores

* agudo (dura menos de 6 meses) o crónico (más de 6 meses)
* con estado de ánimo depresivo, con ansiedad, mixto, con trastorno del comportamiento, no especificado.

En caso de desencadenantes graves considerar los diagnósticos de Trastorno por Estrés agudo y Trastorno por Estrés Postraumático.

==== Diagnóstico diferencial

===== Nosografía clásica

.Con otras neurosis
* Neurosis fóbica: acá la ansiedad-angustia va ligada al objeto o situación fóbica. En la neurosis de angustia no hay desencadenante específico, aunque el paciente puede evitar por ansiedad anticipatoria los lugares donde puede quedar expuesto en caso de sufrir las crisis, pero no porque estos lugares sean desencadenantes por sí mismos. Tampoco tienen conductas tranquilizadoras.
* Neurosis de histeria
* Neurosis obsesivo-compulsiva.

.Trastorno afectivo primario
Depresión reactiva / melancolía.

.Causas orgánicas de crisis de angustia:
* Cardiovasculares: angor, IAM, TEP, arritmias.
* Pleuropulmonares: hiperventilación, broncoespasmo, neumotórax.
* Endocrinológicas: feocromocitoma, hipertiroidismo, hiperparatiroidismo.
* Neurológicas: epilepsia de lóbulo temporal, tumores cerebrales, migrañas, trastornos vestibulares.
* Metabólicas: hipoglicemia, insulinomas.
* Fármacos: corticoides, tiroxina.

.UISP
También debemos descartar que esto sea secundario a abstinencia de sustancias psicoactivas depresoras de SNC (alcohol, sedantes, benzodiacepinas, barbitúricos) o síntomas de intoxicación por estimulantes del SNC (cafeína, cocaína, anfetaminas)

Aunque no existan datos concretos que apunten a una causa de las enumeradas, de cualquier modo descartaremos por paraclínica. Destacar que ningún trastorno médico descarta un trastorno psíquico coexistente.

===== DSM IV - CIE-10

.Trastorno de personalidad
Al poder acumular diagnóstico en cada uno de los ejes (pueden coexistir varios trastornos de ansiedad + trastornos del estado de ánimo + trastornos de la personalidad), los DD cambian:

.Entre trastornos de ansiedad

Que no sean acumulables, por ejemplo: DD entre TAG y Trastornos adaptativos. Algunos pueden acumularse, por ejemplo: TAG y Trastorno de pánico pueden coexistir, siempre que la ansiedad no haga referencia exclusivamente al trastorno de pánico. Similares consideraciones para TAG y TEPT.

.Causas médicas y sustancias de trastornos de ansiedad

Descartar en especial Hipertiroidismo (temblores, palmas frías y húmedas, nerviosismo), feocromocitoma, hipoglicemias.

. Otas causas de síntomas de ansiedad

* Sustancias: intoxicación (cocaína, estimulantes) o abstinencia (alcohol, benzodiacepinas, barbitúricos).
* Fármacos: corticoides, T4.

.Trastorno por somatización
Trastornos somatomorfos/disociativos

==== Diagnóstico etiopatogénico y psicopatológico

===== Etiopatogenia

Diferentes niveles explicativos:

.Factores biológicos

El modelo biológico es el de vulnerabilidad/estrés. La ansiedad es un comportamiento posible como respuesta al estrés (respuesta programada de lucha/huída). El trastorno sería el producto de la disregulación del sistema que procesa información de forma errónea aumentando la respuesta de ansiedad.

Genéticos: incidencia familiar (es más probable que se hereden formas con Agorafobia). Riesgo x4 o x8 en familiares de primer grado. Apoyado en estudios con gemelos.

Disregulación a nivel del SNC y SNP, aumento del tono simpático con adaptación más lenta a estímulos repetidos.

Neurotransmisores implicados: relación con el sistema Gaba y función de éste en la modulación de otros sistemas de transmisión neurohumoral (y su relación con las benzodiacepinas), relación con sistemas serotoninérgico y noradrenérgico.

Consideraciones neuroanatómicas: se correlaciona con el tronco cerebral (locus coeruleus y neuronas serotoninérgicas del núcleo del rafe), sistema límbico (génesis de la ansiedad anticipatoria) y córtex prefrontal (génesis de evitación fóbica).

En estudios imagenológicos se señalan alteraciones en lóbulos temporales (hipocampo) con disregulación del flujo sanguíneo a ese nivel.

.Factores psicológicos

*Teoría cognitivo-comportamental*

La ansiedad sería una respuesta condicionada a estímulos ambientales específicos (exposición primaria generalización + conceptualización cognitiva). También estarían implicada la imitación de conductas parentales.
En lo cognitivo: estructura cognitiva que determina una interpretación catastrófica de las sensaciones somáticas e interoceptivas y una percepción de uno mismo y del riesgo del entorno erradas.

*Psicodinámico*

Relación con la aparición de pulsiones agresivas. En un intento de enfoque comprensivo del paciente podemos vincular este trastorno a:

* Dependencia
* Coartación en la infancia de las manifestaciones de AA (ambiente rígido, padres severos)
* Dificultad para vivenciar la agresividad, cólera culpabilizada o reprimida
* Intolerancia a los propios sentimientos de odio
* Factores sexuales: deseo culpabilizado, temor a la sexualidad
* Estrés psicosocial (duelo patológico: duración, síntomas disociativos relacionados con el mismo) reactiva viejas pérdidas depresión culpa.

===== Psicopatología

Para el psicoanálisis la ansiedad sería una señal de la inminente aparición a nivel consciente de un impulso intolerable para el Yo. Esto implica una falla del mecanismo de represión, lo que motivaría el uso de otros mecanismos de defensa que pueden producir formaciones sintomáticas constituyéndose la neurosis. Según el psicoanálisis existirían 4 tipos de ansiedad: ansiedad impulsiva o del Ello, ansiedad de separación, ansiedad de castración y ansiedad del Superyo. Existe una mala estructuración del Yo que fracasa en la síntesis e integración de los impulsos instintivos del Ello, las exigencias normativas y prohibitivas del Superyo y las presiones de la realidad exterior. La reactivación del conflicto psíquico inconsciente vinculado a la angustia de castración sobrepasa el mecanismos de represión (destinado a mantener la pulsión fuera del campo consciente) por lo cual los impulsos rechazados amenazan irrumpir en la conciencia alterando la homeostasis emocional. El enfrentamiento del yo con dicha conflictiva provoca la ansiedad cuya función es la de anunciar la existencia de un peligro. 

La neurosis de angustia es el primer estadio de las restantes neurosis en la cual el yo no ha aprendido a defenderse. Si bien existen intentos de recuperar el equilibrio recurriendo a mecanismos de defensa estructurados de tipo histéricos (disociativos, conversivos), obsesivos, fóbicos. En el caso de coexistencia con síntomas agorafóbicos, estarían en juego mecanismos de defensa tales como la represión, desplazamiento, evitación y simbolización.

==== Paraclínica

El diagnóstico es clínico.

===== Biológico

Descartar causas tratables ya citadas. EF completo y PC e interconsultas según hallazgos. Rutinas de valoración general. ECG: trastornos de la conducción por ADT, extrasístoles. En especial: Función Tiroidea.

===== Psicológico

Entrevistas reiteradas profundizando en el conflicto, evaluación más concreta de eventos vitales. Superado el cuadro actual con vistas a un abordaje psicoterapéutico procurando conocer la fortaleza yoica, mecanismos de defensa y manejo de la angustia. Tests de personalidad proyectivos (TAT y Rorscharch), no proyectivos (MMPI-Minessotta), tests de nivel (Weschler). Puede ser de utilidad para valorar la respuesta al tratamiento la aplicación de inventarios tales como el cuestionario de Hamilton para la ansiedad de 14 ítems (HAM-A).

===== Social

Entrevistas con terceros (familiares, amigos, compañeros de trabajo), red de soporte social, evaluando repercusión, objetivando reacciones de ésos así como su tolerancia con respecto al trastorno del paciente. Investigación de elementos que pueden coadyuvar a mantener el trastorno.

==== Tratamiento

Ambulatorio. Internar en caso de poca continencia familiar. Actuaremos s/t a nivel sintomático sobre:

* Crisis de pánico y sus complicaciones eventuales
* Ansiedad de fondo
* Síndrome depresivo y sus complicaciones (IAE)
* Insomnio

Como primera medida, estableceremos un buen vínculo, realizando continentación y apoyo, permitiendo expresión de emociones. Será de importancia el establecer determinadas medidas higiénico-dietéticas tales como: eliminación de cafeína o nicotina que pueden excerbar los síntomas. Tratamiento biológico y psicológico específico: ver más adelante. Combatiremos el insomnio con Zolpidem 10 mg en la noche, que retiraremos lentamente una vez regulados los parámetros de sueño con el tratamiento ansiolítico, antidepresivo y de fondo. (Opción: Flunitrazepam 2 mg).

Psicológico Entrevistas reiteradas con la frecuencia necesaria mitigando la ansiedad, en un marco de continentación. Psicoterapia de apoyo, procurando obtener los niveles previos de funcionamiento. Psicoeducación en relación a las crisis de pánico, explicando que son autolimitadas en el tiempo, que no son perjudiciales que no revisten peligro de muerte, etc.

===== Trastorno de pánico
.Biológico

*Durante la crisis*

En urgencias Alprazolam 1 mg o Diacepam 5 mg s/l. Para controlar la hiperventilación y la posible tetania puede ser útil reciclar el propio CO2 con una mascarilla obturada. Tras el control de la crisis puede comenzarse el tratamiento de base.

*Tratamiento de base*

Se recomienda la combinación de tratamiento farmacológico con psicoterapia. 

El tratamiento farmacológico con mejor perfil de eficacia y tolerancia es con Benzoadiacepinas, ISRS o antidepresivos tricíclicos. Se prefieren los ISRS a los ADT por mejor perfil de seguridad \footnote{Chawla, N., Anothaisintawee, T., Charoenrungrueangchai, K., Thaipisuttikul, P., McKay, G. J., Attia, J., \& Thakkinstian, A. (2022). Drug treatment for panic disorder with or without agoraphobia: systematic review and network meta-analysis of randomised controlled trials. bmj, 376.}.

Se puede comenzar con benzodiacepinas + ISRS, retirando en forma progresiva las benzodiacepinas una vez que los ISRS comienzan a actuar, luego de su período de latencia (4-8 semanas). Si bien todos los ISRS tiene una eficacia similar, se recomiendan en primer lugar los sedativos ya que tienen acción sobre la ansiedad de fondo.

ISRS : 

* Por eficacia: Fluoxetina 20-40 mg/día → Fluvoxamina 100-200 mg/día → Escitalopram 10-20 mg/día → Paroxetina 20-40 mg/día → Sertralina 50-100 mg/día → Citalopram 20-40 mg/día.
* Por tolerancia: Escitalopram → Sertralina → Fluvoxamina → Paroxetina → Citalopram → Fluoxetina.
* Mejor perfil eficacia/tolerancia: Escitalopram.

Debe comenzarse con dosis muy bajas y aumentos graduales por la sensibilidad de éstos pacientes que pueden presentar exacerbación de los síntomas si se inicia de forma brusca (iniciar con 1/4 o 1/2 comprimido de cualquier ISRS). Latencia media: 4 semanas.

Benzodiacepinas: Alprazolam liberación prolongada 1-2 mg/día en toma única -> Clonazepam 0.5 - 6mg/día en 2-3 tomas -> Alprazolam común 1 - 3 mg/día en 3 tomas (otras benzodiacepinas no tienen acción antipánico demostrada). Latencia media: 1 semana.

Segunda línea

Pueden usarse antidepresivos tricíclicos: Clorimipramina 150-250 mg/día, Imipramina 100-300 mg/día. Se inician con 25 mg/día en 2--3 tomas (1-2 tomas en caso de Clorimipramina de liberación sostenida), con aumentos graduales de 25 mg hasta desaparición de las crisis. Latencia media 6 semanas (mayor que ISRS).

Casos resistentes: Fenelzina 30-90 mg/día (ver en F40 manejo de IMAOs) o asociaciones ISRS + ADT (a dosis menores que las usadas en monoterapia). También puede plantearse el uso de otros antidepresivos (Venlafaxina), análogos de la somatostatina, antagonistas del calcio, carbamazepina, lamotrigina o gabapentina (todos ellos en fase experimental).

El tratamiento medicamentoso será mantenido por 12 meses luego de la remisión sintomática. A partir de ese momento puede intentarse una reducción gradual (de 20% al mes de todos los fármacos). Si reaparece la sintomatología se reinstaura el tratamiento a las dosis eficaces por 24 meses. Eventualmente puede requerir tratamiento por tiempo indefinido.

.Tratamiento psicoterapéutico

Terapia cognitivo-comportamental: psicoeducación + técnicas de manejo de la ansiedad y de la crisis de pánico (respiración controlada, relajación, reestructuración cognitiva).

===== Ansiedad de fondo

.Biológico
Si coexiste con crisis de pánico: el tratamiento queda cubierto con lo expuesto.

Si aparece como único síndrome: se aconseja usar una única benzodiacepina. La elección de la misma se determinará según: edad del paciente, estado físico, respuesta previa a otra BZD, propiedades farmacológicas de cada BZD.

* Adulto joven sano: Diazepam 5-20 mg/día en 1-3 tomas (o equivalente de vida media larga [Clonazepam] o intermedia [Alprazolam, Bromazepam]).
* Paciente añoso: Lorazepam 2-8 mg/día en 3-4 tomas u Oxazepam 15-45 mg/día en 2-3 tomas (carecen de metabolitos activos

En caso de falta de respuesta: aumento de dosis o cambio a otra benzodiacepina. De forma concomitante o alternativa, puede plantearse el uso de antidepresivos ISRS sedativos En caso de que sea necesario evitar el efecto sedativo, manteniendo un efecto ansiolítico: Buspirona 20-45 mg/día en 1-2 tomas, sabiendo que puede presentar una latencia de hasta 2 semanas en su acción ansiolítica. Debe tenerse en cuenta la posibilidad de antagonización de su efecto sedativo en caso de uso concomitante con ISRS.

.Psicológico
Tratamiento psicoterapéutico: Terapia cognitivo-comportamental: psicoeducación + técnicas de manejo de la ansiedad, técnicas de relajación, reestructuración cognitiva.

.Síntomas fóbicos

Ver F40. 

===== General

Mantendremos la psicoterapia de apoyo y control de medicación antidepresiva que mantendremos a largo plazo ya que su suspensión aumenta el índice de recaídas de las crisis de angustia. En algunos casos puede indicarse psicoterapia de corte analítico, una vez superado el cuadro actual. Indicado en casos de: paciente joven, con buen nivel intelectual, con deseos de curarse, que ha pedido ayuda, que inició recientemente los síntomas, con buen insight y problemática global que exceda al cuadro actual.

===== Social

Psicoeducación explicando su enfermedad y la necesidad de tratamiento y controles, actuando s nivel de desencadenantes y factores que contribuyan a mantenimiento. Eventualmente: terapia familiar.

==== Evolución y pronóstico

Trastorno con tendencia a la cronicidad con curso variable: 30% libre de sintomatología, 50% síntomas leves, 20% síntomas más graves. Del cuadro actual: cederá con el tratamiento instituido, presentando un bajo riesgo suicida. Se habla de un 80% de remisiones para el tratamiento combinado de fármacos + psicoterapia. Como complicaciones de la crisis de pánico cabe citar: ansiedad persistente, evitación fóbica, depresión, abuso de alcohol, drogas, trastorno por somatización (hipocondría secundaria), dependencia, alteraciones médicas (mayor morbilidad por HTA, UGD). Posibilidades evolutivas de la neurosis de angustia:

* Organización en neurosis más estables: fóbica, histérica, obsesiva.
* Aparición de síntomas hipocondríacos con centralización de la atención y ansiedad sobre determinados órganos, alteraciones psicosomáticas.
* Abuso de sustancias psicoactivas (en 20-40% de los pacientes).
* Episodios depresivos (complica el cuadro en un 40-80% de casos).

El pronóstico psiquiátrico alejado dependerá de:

* Fuerza/madurez del yo (de la estructura de la personalidad y su capacidad de elaborar mecanismos de defensa más adaptativos).
* Peripecias vitales a las cuales estos pacientes son especialmente sensibles, dependerá de capacidad para enfrentar nuevas situaciones penosas.
* Éxito de la terapéutica y adhesión a la misma.

Los elementos de buen pronóstico son:

* corta edad con buen nivel intelectual
* corta duración
* pedido de ayuda


.Nota: Depresión neurótica
Concepto caduco en lo nosológico, útil en la clínica.
- Tonalidad afectiva más próxima al sentimiento de tristeza normal
- Ansiedad intensa
- El contenido está relacionado con el acontecimiento desencadenante (o es + comprensible)
- Busca compasión/consuelo
- Acusa a otros de su suerte (y no a sí mismo)
- Sentimiento de impotencia que proyecta a los demás
- Mayor sensibilidad a influencias del medio que el melancólico
- Fondo de depresión está en relación a herida narcisista
- Pico vespertino
- Avidez afectiva puede alcanzar un carácter tiránico
- Menor inhibición psicomotriz que permite expresión más dramática

Rasgos típicos de los Ataques de Pánico:

* Historia temprana
* AF
* Presencia de crisis nocturnas

==== Fuentes

* Kaplan
* DSM IV
* RTM II, 1999.
* Clinical Handbook of Psychotropic Drugs - Bezchlibnik-Butler, 8th ed, 1998.
* The Journal of Clinical Psychiatry 60 (supp 18), 1999.
* Encares: Dr. Curbelo - Dr. Escobal

Falta: adecuada organización y puesta al día de la parte de psicopatología y etiopatogenia, al incluir varios trastornos de ansiedad, se hace necesario discriminar cada uno en cada ítem. Lo ideal sería hacer encares independientes.
