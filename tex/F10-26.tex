= Síndrome de Korsakoff


== Descripción

Parte del Síndrome de Wenicke-Korsakoff (SWK) descrito en 1881 (Wernicke) y 1889 (Korsakoff). Cuadro de comienzo súbito con un período de confusión y ocasionalmente con oftalmoplegia-nistagmo o ataxia. Post-mortem se encontraban pequeñas hemorragias en los cuerpos mamilares (área del cerebro involucrada en la memoria reciente y en la visión binocular)

Los síntomas precoces comprenden: deficiencia de tiamina, pérdida de memoria reciente retrógrada, pérdida de memoria anterógrada, precedida de consumo crónico de alcohol. La pérdida de ambras memorias, producen una extinción de la evocación de eventos desde el inicio del cuadro.

== Terminología

- Síndrome de Wernicke: generalmente se aplica a la deficiencia de Tiamina sin uso de alcohol.
- Síndrome de Korsakoff: deficiencia de Tiamina precedida de uso crónico de alcohol, con inicio agudo y florido.

== Clínica
Presentaciones muy dispares con diferentes componentes de: amnesia footnote:[Blansjaar, B. A., Takens, H., & Zwinderman, A. H. (1992). The course of alcohol amnestic disorder: a three‐year follow‐up study of clinical signs and social disabilities. Acta Psychiatrica Scandinavica, 86(3), 240-246.], delirium footnote:[ Wijnia JW, Oudman E (2013) Biomarkers of delirium as a clue to diagnosis and pathogenesis of Wernicke-Korsakoff syndrome. Eur J Neurol. 20:1531–1538.], demencia footnote:[Ridley NJ, Draper B, Withall A (2013) Alcohol-related dementia: An update of the evidence. Alzheimers Res Ther. 5:3.], encefalopatía footnote:[Isenberg-Grzeda E, Hsu AJ, Hatzoglou V, Nelso C, Breitbart W (2015) Palliative treatment of thiamine-related encephalopathy (Wernicke's encephalopathy) in cancer: A case series and review of the literature. Palliat Support Care. 13:1241–1249], trastorno de personalidad footnote:[Plutchik R, DiScipio WJ (1974) Personality patterns in chronic alcoholism (Korsakoff's syndrome), chronic schizophrenia, and geriatric patients with chronic brain syndrome. J Am Geriatr Soc. 22:514–516.], abstinencia de alcohol footnote:[Trevisan LA, Boutros N, Petrakis IL, Krystal JH (1998) Complications of alcohol withdrawal: Pathophysiological insights. Alcohol Health Res World. 22:61–66.], trumatismo footnote:[Brion S, Plas J, Mikol J, Jeanneau A, Brion F (2001) Post-traumatic Korsakoff's syndrome: Clinical and anatomical report. Encephale. 27:513–525.] y psicosis footnote:[Ramayya A, Jauhar P (1997) Increasing incidence of Korsakoff's psychosis in the east end of Glasgow. Alcohol Alcohol. 32:281–285.]

Por el consumo de alcohol puede haber: neuropatía, falla hepática, várices esofágicas, otros trastornos neuro-psiquiátricos, infeciones, ansiedad-depresión, formas de cáncer vinculadas al alcohol footnote:[Westermeyer JJ, Soukup B (2021) Late-onset cases of Korsakoff amnestic syndrome with CNS comorbidities: Suggestions for long-term management. Addict Disord Treat. 20:69–73.].

=== Signos patognomónicos y curso
Los pacientes poseen una evocación "arcaica" intacta a pesar de las diferentes alteraciones en la memeoria presentes footnote:[Cutting J (1978) The relationship between Korsakoff's syndrome and ‘alcohol dementia’. Br J Psychiatry. 132:240–251.] footnote:[Spiegel DR, O'Connell K, Stocker G, Slater J, Spiegel A (2020) A case of Wernicke-Korsakoff syndrome initially diagnosed as autoimmune limbic encephalitis: Differential diagnosis of delirium and short-term memory deficits. Prim Care Companion CNS Disord. 22:20l02693.]. Existe una afectación de memoria y aprendizaje de forma desproporcionada en relación a otras funciones cognitivas, en un paciente alerta footnote:[Victor M (1989) The Wernicke-Korsakoff syndrome and related neurologic disorders due to alcoholism and malnutrition. Philadelphia: F.A. Davis.].

A esto se le agrega neuropatía periférica y signos oftalmológicos footnote:[Caine D, Halliday GM, Kril JJ, Harper CG (1997) Operational criteria for the classification of chronic alcoholics: Identification of Wernicke's encephalopathy. J Neurol Neurosurg Psychiatry. 62:51–60.]. Sin embargo la ataxia ocurre solo en 23% de los casos agudos y la parálisis de músculos extraoculares aparece en el 29% footnote:[Sechi G, Serra A (2007) Wernicke's encephalopathy: New clinical settings and recent advances in diagnosis and management. Lancet. 6:442–455.]. La ataxia (pora la degeneración del vermis cerebelar inducida por alcohol) puede ocurrir de forma independiente de la amnesia footnote:[Akbar U, Ashizawa T (2015) Ataxia. Neurol Clin. 33:225–248.]. Por lo tanto, las neuropatías pueden estar ausentes en el SWK, así como pueden estar presentes en el alcoholismo sin SWK. Pueden remitir de forma independiente de la amnesia.

El tratamiento con tiamina (aunque demore 1 o 2 días) puede provocar una recuperación parcial. Si se continúa el tratamiento por meses o años, puede verse una mejoría incremental lenta con cierta amnesia persistente. Si el tratamiento se retrasa 1-2 semanas, puede comprometerse la recuperación parcial footnote:[Thomson AD, Guerrini I, Marshall EJ (2012) The evolution and treatment of Korsakoff's syndrome: Out of sight, out of mind?Neuropsychol Rev. 22:81–92].

La memoria inemdiata en el SWK dura pocos minutos. Los detalles relacionados a eventos recientes con componente emocional (por ejemplo una tormenta eléctrica reciente) se pierden, aunque por condicionamiento puede aparecer una respuesta emocional en relación al evento original. Este déficit implica la imposiblidad de una vida autónoma. La falta de conciencia de los eventos que llevan a la circunstancia actual implican una falla en la resolución de problemas y en el juicio. El estado emocional del paciente puede ir desde una indiferencia calma hasta la conducta demandante e irritabilidad. La desinhibición puede precipitar conductas violentas footnote:[Gerridzen IJ, Hertogh CM, Depla MF, Veenhuizen RB, Vershuur EML, Joling KJ (2018) Neuropsychiatric symptoms in people with Korsakoff syndrome and other alcohol-related cognitive disorders living in specialized long-term care facilities: Prevalence, severity, and associated caregiver distress. J Am Med Dir Assoc. 19:240–247.].

La memoria "arcaica" (eventos pre inicio del SWK) permanece accesible (nombres, eventos, aritmética básica, lecto-escritura). Puede perderse por falta de uso. Puede conservarse la realización de conductas complejas (nadar, tipear, conducir) si se practican (memoria procedural). Pueden continuar una actividad laboral en entornos supervisados. Con cuidados adecuados la sobrevida es buena footnote:[Sanvisens A, Zuluaga P, Fuster D, Rivas I, Tor J, Marcos M, Chamorro AJ, Muga R (2017) Long-term mortality of patients with an alcohol-related Wernicke-Korsakoff syndrome. Alcohol Alcohol. 52:466–471.].

.Funciones mnésicas en el SWK
[%header]
|===
|Categorías cronológicas relacionadas con la memoria|Estado en el SWK
|Memoria inmediata (hasta varios minutos)|Intacta: luego de la fase aguda, el paciente funciona normalmente en aspectos inmediatos del estado mental y en el examen psicométrico
|Memoria reciente (minutos-días) | Falla: se pierde la memoria de episodios recientes y de episodios experimentados de forma personal, así como reportados por otros.
|Memoria remota: intermedia (semanas-meses) y a largo plazo (años-décadas)| Intacta: se retienen recuerdos pre inicio del SWK (experimentados o aprendidos de otros). Esto se conoce como "memoria arcaica".

Falla: recuerdos posteriores al inicio del SWK (experiencia personal o transmitida).

Intacta: memoria post inicio del SWK con contenido emocional, dependiente del estado, somática (todo lo no-semántico o episódico).
|===

== Diagnóstico

El reconocimiento del SWK depende de tenerlo presente como diagnóstico footnote:[Wijnia JW, Nieuwenhuis KG (2011) Difficulties in identifying Wernicke-delirium. Eur J Intern Med. 22:e160–e161.]. Es necesario una anamnesis que se remonte décadas atrás, un examen psiquiátrico que contemple lo cognitivo, un examen neurolóico y el acceso a datos de terceros. Es útil intentar provocar la confabulación (relatos imaginados por el paciente para llenar lagunas mnésicas).

Descartar otros consumos de sustancias, trastornos del control de los impulsos, trastornos de ansiedad, trastornos del humor.

== Diagnóstico diferencial

El SWK representa una de las posibles causas de amnesia vinculada a alcohol y nutricionales. Estas patologías difieren en su fisiopatología, signos, síntomas, curso y pronóstico.

- Amnesia transitoria, reversible ("blackout") que acompaña algunos episodios de intoxicación. Revierte luego de algunas horas de abstinencia. Puede ocurrir precozmente en el curso del alcoholismo.
- Pérdida de memoria asociada a disfunción ejecutiva y desinhibición sugerente de atrofia del lóbulo temporal, que ocurre tardíamente en el alcoholismo o bien por TEC.
- Demencia progresiva, por atrofia pan-cortical que puede asemejarse a una Enfermedad de Alzheimer, pero de inicio más precoz.
- Lesiones cerebrales localizadas secundaria a complicaciones del alcoholismo (traumatismos, infección, cáncer, patología vascular)
- Deficiencias nutricionales (especialmente otras formas de vitamina B, C o deficiencia proteica [pelagra, escorbuto])

Los medicamentos antipsicóticos, antidepresivos, ansiolíticos y usados para el tratamiento de la demencia no provocan mejoría en los síntomas nucleares del SWK. De todos modos se usan para tratamiento de comorbilidades. 

== Confabulación como adaptación

Los pacientes con SWK se habitúan a llenar los huecos de memoria reciente y remota. Si se les pregunta por eventos del día anterior inventados, responden apoyando la afirmación agregando información inventada o percepciones sensoriales falsas. Con el tiempo cesa la confabulación en la esfera sensorial, pero pueden continuar con la conceptual a la hora de explicar eventos sobre los cuales tienen amensia. Eventualmente pueden llegar a admitir que tienen un problema de memoria. Si el entorno deja de preguntar por eventos que no puede recordar, en general se detiene la conducta.

== Epidemiología
Los estudios de mortalidad dependen de las autopsias en las que se encuentra hemorragia en los cuerpos mamilares. Se plantea una prevalencia de 1% de las muertes footnote:[Harper CG, Sheedy DL, Lara AI, Garrick TM, Hilton JM, Reisanen J (1998) Prevalence of Wernicke-Korsakoff syndrome in Australia: Has thiamine fortification made a difference?Med J Aust. 168:542–545.]. De este 1% solo el 16% fueron diagnosticados en vida. Este 1% es la reducción (desde el 2%) debida a la obligatoriedad de pan con suplemento de tiamina en Australia. Se estima que el SWK afecta a un 10% de alcoholistas severos footnote:[Galvin R, Bråthen G, Ivashynka A, Hillborm M, Tanasescu R, Leone MA; EFNS (2010) EFNS guidelines for diagnosis, therapy and prevention of Wernicke encephalopathy. Eur J Neurol. 17:1408–1418.].

== Desencadenantes
En general se identifican como desencadenante el agotamiento de las reservas de tiamina que resulta en una disfunción mitocondrial, con deterioro de la oxidación celular y disminución de la energía neuronal disponible. Desencadenantes comunes:

- abstinencia alcohólica severa (como el DT)
- sepsis, neumonia, infecciones urinarias footnote:[Wijnia JW, Oudman E, van Gool WA, Wierdsma AI, Bresser EL, Bakker J, van de Wiel A, Mulder CL (2016) Severe infections are common in thiamine deficiency and may be related to cognitive outcomes: A cohort study of 68 patients with Wernicke-Korsakoff syndrome. Psychosomatics. 57:624–633.]
- falla orgánica (páncreas, hígado, corazón, riñones)
- trauma tisular (cirugía, TEC, fracturas)

Se desconocen las vías por las cuales estos eventos agotan la tiamina. La abstinencia de alcohol aumenta los niveles de cortisol footnote:[Keedwell PA, Poon L, Papadopoulos AS, Marshall EJ, Checkley SA (2001) Salivary cortisol measurements during a medically assisted alcohol withdrawal. Addict Biol. 6:247–256.] que podría ser un mediador.

== Estudios de neuroimagen

PET: muestra hipometabolismo durante el SWK footnote:[Reed LJ, Lasserson D, Marsden P, Stanhope N, Stevens T, Bello F, Kingsley D, Colchester A, Kopelman MD (2003) FDG-PET findings in the Wernicke-Korsakoff syndrome. Cortex. 39:1027–1045.] con hipermetabolismo en la sustancia blanca cercana. Esto sugiere muerte celular en las áreas vinculadas a la memoria reciente con actividad axonal compensatoria.

RNM volumétrica: muestra atrofia de los cuerpos mamilares, de otras áreas talámicas, de lóbulos frontales y otras áreas footnote:[Sullivan EV, Deshmukh A, Desmond JE, Lim KO, Pfefferbaum A (2000) Cerebellar volume decline in normal aging, alcoholism, and Korsakoff's syndrome: Relation to ataxia. Neuropsychology. 14:341–352.] footnote:[Sullivan EV, Pfefferbaum A (2009) Neuroimaging of the Wernicke-Korsakoff syndrome. Alcohol Alcohol. 44:155–165.].

Pueden verse anmesias similares a la del SWK en otras afecciones:

- Lesiones cerebrales localizadas de etiología vascular footnote:[Rahme R, Mousa R, Awada A, Ibrahim I, Ali Y, Maarrawi J, Rizk T, Nohra G, Okais N, Samaha E (2007) Acute Korsakoff-like amnestic syndrome resulting from left thalamic infarction following a right hippocampal hemorrhage. AJNR Am J Neuroradiol. 28:759–760.], neoplásica footnote:[de Falco A, De Simone M, Spitaleri D, de Falco FA (2018) Non-alcoholic Wernicke-Korsakoff syndrome heralding non-Hodgkin lymphoma progression. Neurol Sci. 39:1493–1495.] y traumáticas footnote:[Kahn EA, Crosby EC (1972) Korsakoff's syndrome associated with surgical lesions involving the mammillary bodies. Neurology. 22:117–125.].
- Desnutrición asociada caquexia por hambruna footnote:[DeWardener HE, Lennox B (1947) Cerebral beriberi (Wernicke's encephalopathy); Review of 52 cases in a Singapore prisoner-of-war hospital. Lancet. 1:11–17.], pelagra, beriberi footnote[Di Marco S, Pilati L, Brighina F, Fiero B, Cosentino G (2018) Wernicke-Korsakoff syndrome complicated by subacute beriberi neuropathy in an alcoholic patient. Clin Neurol Neurosurg. 164:1–4.] y ayuno de causa psiquiátrica footnote:[Hargrave DD, Schroeder RW, Heinrichs RJ, Baade LE (2015) Wernicke-Korsakoff syndrome as a consequence of delusional food refusal: A case study. Cogn Behav Neurol. 28:215–219.].
- Malabsorción intestinal y alteración en fluidos por by-pass gástrico y disfunción intestinal footnote:[Fandiño JN, Benchimol AK, Fandiño LN, Barroso FL, Coutinho WF, Appolinário JC (2005) Eating avoidance disorder and Wernicke-Korsakoff syndrome following gastric bypass: An under-diagnosed association. Obes Surg. 15:1207–1210.], enfermedad celíaca footnote:[Sahu M, Beal M, Chism K, Becker MA (2020) Wernicke-Korsakoff syndrome in a patient with celiac disease and obsessive-compulsive disorder: A case report. Psychosomatics. 61:375–378.], tratamiento excesivo con inhibidores de la bomba de protones footnote:[Miyanaga R, Hisahara S, Ohhashi I, Yamamoto D, Matsumura A, Suzuki S, Tanimoto K, Hirakawa M, Kawamata J, Kato J, Shimohama S (2020) Hyperemesis-induced Wernicke-Korsakoff syndrome due to hypergastrinemia during long-term treatment with proton pump inhibitors. Intern Med. 59:2783–2787.], hiperemesis gravídica footnote:[Ashraf VV, Prijesh J, Praveenkumar R, Saifudheen KJ (2016) Wernicke's encephalopathy due to hyperemesis gravidarum: Clinical and magnetic resonance imaging characteristics. Postgrad Med. 62:260–263.]
- Cáncer con niveles séricos bajos de tiamno, baja ingesta, pérdida de peso y afectación gastrointestinal o hematológica footnote:[Isenberg-Grzeda E, Alici Y, Hatzouglou V, Nelson C, Breitbart W (2016) Nonalcoholic thiamine-related encephalopathy (Wernicke-Korsakoff syndrome) among inpatients with cancer: A series of 18 cases. Psychosomatics. 57:71–81.]

== Medios alternativos de aprendizaje

Alternative Means of Learning
Intact Brain Functions
The sentinel learning deficit in WKS involves recall of events, also known as episode amnesia (Pitel et al., 2009). These episodes or events fail to register in short-term memory stores after WKS onset—a liability that prevents the selective entry of new events and semantic information into long-term memory. Therefore, before launching into extensive rehabilitation, clinicians serving WKS patients obtain multidisciplinary assessments to identify intact versus damaged brain structures and functions. Depending on the patient, these assessments might include consultations with neurology, neuroimaging, neuropsychology, primary care, addiction psychiatry, addiction nursing, and psychosocial rehabilitation (Van Dam et al., 2020).
“A WKS patient with extensive frontal lobe vascular damage required 1 year of behavioral therapies to address addictive behaviors besides AUD (e.g., pathological gambling, online pornography, nicotine dependence). Another WKS patient with earlier Agent Orange exposure had extensive basal ganglia lesions, parkinsonian tremors, and parkinsonian-gait ataxia. He responded to antiparkinsonian medications and physical therapy for his gait. Both patients have since lived at home with their families for several years.”
After adequate assessments have been completed, with strengths and deficits identified, a treatment team can plan appropriate, often staged interventions. The latter rely extensively on language-free approaches, such as emotional arousal, state-dependent paradigms, and other approaches that typically remain operational in WKS.
Operant Conditioning
The link between a causal behavior and its consequential effect underlies operant conditioning. Addiction processes follow the principles of operant conditioning (Koob, 2017), which usually survives WKS onset and can contribute to learning, adaptation, and recovery from addiction. People with WKS rapidly learn the location of bathrooms in buildings, heated areas in cold weather, and places to rest after strenuous activity. Caregivers can foster operant learning by observing what pleases the WKS patient.
“A WKS patient enjoyed the hospital cafeteria where his quarterly clinic visits occurred. He could locate the cafeteria from most areas of the building. His spouse could find him in the cafeteria if they became separated during the day.”
Classical Conditioning
This method involves pairing a natural reward to a desirable new behavior (O'Brien et al., 1992). The new behavior may be desired by the patient, family, or caregiver. For example, breakfast may be available for a reasonable period after a morning alarm, to establish daily schedules and avoiding oversleeping, or an after-dinner treat for clearing the table and helping to clean up can elicit participation in household chores.
“A WKS patient who owned rental property had enjoyed making repairs and upgrades to his property prior to WKS onset. Post-WKS-onset, he started three complex projects, each of which led to a large mess and expensive repairs when he could not complete them. Subsequently, his spouse hired skilled workman to undertake challenging projects. The WKS spouse happily served as an assistant in the process, anticipating the tools and supplies that workmen needed and assisting with clean-up. Paired with a teenaged grandson, the spouse was able to undertake simpler, repetitive chores, such as repairing a fence or doing gardening chores.”
Aversive Conditioning
This approach involves negative or unpleasant consequences linked to unhealthy or dangerous behavior. For example, WKS patients, despite their grave disability, can be amazingly persistent and creative in pursuing alcohol and other addictions. Access to even small, infrequent volumes of alcohol can goad WKS patients to devote time and ingenuity to this pursuit, with painful consequences.
“Within weeks of onset, a WKS patient began creatively collecting funds to purchase alcohol. He saved church and hospital Bingo winnings, searched furniture cushions for loose change, took coins from unattended purses, and spent a family coin collection. The attending physician and the patient's legal guardian (his spouse) agreed to administer monitored disulfiram on a daily basis, beginning after they informed the patient and providing him with written materials. Although he agreed to the intervention, he forgot the details and tried drinking. Since that single alcohol-disulfiram reaction, he has not attempted alcohol use during the intervening decade.”
Metronidazole, an antiprotozoal drug that produces a milder acetaldehyde reaction, can be used in patients whose health contravenes disulfiram use.
Ethical analyses must be considered in aversive conditioning to ensure that the goal is recovery and not punishment, with reasonable risk and equity (Sullivan et al., 2008). Bioethical committees in clinical settings can help in addressing these concerns and devising a humane conditioning protocol. When caretakers fail to interrupt alcohol use, WKS patients usually enter a crisis-ridden period ending in liver failure within a few years.
Contingency Contracting
Contractual agreements can entail a therapeutic exchange between a patient and a friend, relative, spouse, or employer (Sullivan et al., 2008). For example, a family might shelter an alcohol-addicted member willing to take monitored disulfiram daily. The federal government can assign payment of federal pension funds through a representative payee, who ensures that alcohol or drug purchases do not replace expenditures for food, shelter, and clothes.
Errorless Learning
Training so as to prevent learning errors has been helpful in learning disabilities. This approach requires extra time and nuance, but it reduces negative corrections and provides more pleasant learning experiences. It has helped WKS patients with nonsemantic learning (Rensen et al., 2019).
Affective Recruitment
Pairing affective responses with successfully achieving learning objectives is key to this approach. It does not favor new learning if the same positive outcome occurs regardless of the patient's behavior. Emotional experiences can assist WKS patients in learning to distinguish categories (Labuddha et al., 2010). Social network support for abstinence has facilitated recovery in non-WKS AUD (Galanter et al., 1990) and appears therapeutic for WKS patients.
Schedules, Structures, and Keeping on Track
WKS patients retain the ability to read and comprehend written material, despite inability to retain new semantic materials. The key lies in enabling them to access an external data source when anxious or confused. For example, providing a daily schedule can reduce caregiver burdens imposed by repeated questions regarding “what comes next.” Bulletin boards, smart phones, or iPads can help patients implement a daily plan. WKS survivors become motivated by rewarding outcomes when checking the daily plan (e.g., obtaining lunch) and driven by negative outcomes when not consulting them (e.g., missing lunch). These resources can also help WKS patients manage unexpected problems (e.g., becoming lost).
Becoming lost comprises a common WKS problem, which caregivers can address by using structure, schedules, and anticipation. The following case exemplifies how some WKS patients become rigid homebodies and how to reverse this dilemma.
“A man with recent-onset WKS walked away from his family's newly acquired residence. He wandered through the night into a forested rural area. He was found several miles away from home, dehydrated, with scratches and torn clothes. Although unable to describe his experience, he subsequently became anxious away from home. As a vehicular passenger on a shopping trip or clinic visit, he became visibly agitated and shouted every several minutes ‘Where are we? Where are we going?’ Responses to these queries reassured him for some minutes until the reassurance slipped from his immediate memory. The resolution lay in providing him before each trip with 1) the destination and rationale for the current trip and 2) a map with the route denoted. Over time, these documents relieved his anxiety with assuring reorientations to his location, the destination, and the purpose for the trip.”
The following vignette, observed years apart in two patients from different states, epitomizes the panicky “flight” scenario that can erupt in lost WKS patients.
“Family caretakers of two employed WKS patients eventually allowed them to drive alone to and from work, after having ensured that they could drive safely along the same route. This system worked for years until road repairs resulted in detours off the usual route. Each man became lost but continued driving through the night until they ran of fuel. The next day, police located each stranded man over a hundred miles from home.”
Helping WKS patients to avoid becoming lost requires strategic planning. The plan might involve a cell phone, a list of instructions, or localizer-transmitters manufactured for trekkers who might become lost. Caregivers can limit such crises by considering how their WKS relative or patient might use their still partially intact but amnestic brain to manage various problems, along with access to printed materials and modern technology. Experienced caregivers, once oriented to alternative learning principles, may acquire dependable intuitions on such matters. Predictably, a longitudinal study has shown that WKS patients do better in small-scale, individualized, homey residences than in large, more impersonal hospitals or boarding institutions (Cutting, 1978).
Calendars and Smart Phones
Chronological and geographic aids have improved WKS patients' management skills and functionality (de Joode et al., 2013; Lloyd et al., 2019). Time, effort, training, and on-going technological support are needed to achieve practical computer-based utility. Modern technologies can facilitate the constant supervision so often needed, while alleviating the spatial and temporal constraints that can restrict WKS patients and their caregivers.
Day Programs
Attendance at day programs can aid rehabilitation for AUD patients facing major life changes (Favazza and Thompson, 1984). Day programs likewise enable WKS patients to expand their daily routines, adapt to another place and different activities, and reexpand their intimate social network back toward normal (Westermeyer and Neider, 1988). These forays involve trying out new emotion-challenging roles (e.g., reading a newspaper to the group, preparing or serving a meal, leading chair exercises). Programs range from 2 to 6 hours daily, and from one to several sessions weekly.
Our WKS patients have done well in day programs designed for other diagnostic groups, such as Alzheimer dementia or Parkinson disease. WKS patients can rely on other patients for recent memory support, while contributing abilities that other patients lack. Day programs can replicate the multigroup affiliation of intimate social networks that are central to human well-being (Pattison, 1977) and to AUD recovery (Gorden and Zrull, 1991). Programs also relieve primary caregivers from constant supervisory duties and burnout.
Adjunctive Treatments
Neural Transmission
Rivastigmine, a cholinergic medication beneficial in some dementias, has not been effective in a controlled study of WKS patients (Luykx et al., 2008). Transcranial magnetic stimulation has not been correlated with cholinergic activity in WKS syndrome, suggesting that reduced choline does not cause WKS's amnesia (Nardone et al., 2010). Antidepressants and other medications can relieve depression, anxiety, Parkinson disease, Alzheimer dementia, psychosis, or insomnia associated with WKS, but they do not relieve uncomplicated WKS amnesia.
Alcoholics Anonymous
Wernicke syndrome and Korsakoff syndrome patients share certain core brain lesions, but they do not share previous addiction. Familiarity with both Wernicke and Korsakoff syndromes can abet clinicians who treat either disorder. Our Wernicke cases consisted of refugees, combatants, and prisoners of war encountered in Asia and Minnesota (Westermeyer, 1982; Westermeyer, 1989; Williams and Westermeyer, 1986). By definition, Wernicke syndrome patients do not typically require intervention for AUD. Therein lies an enormous difference.
Korsakoff syndrome patients can cathect readily to AA groups. Using their intact archaic memory, they can identify with AA members who have similar life experiences. AA affiliation can contribute another group to the WKS patient's recovery-oriented social network (Hall and Nelson, 1996). Orienting the sponsor and the AA group to WKS pathophysiology (see Table 1) helps to understand and relate to their WKS member.
Avoiding AUD Recurrence
Some families and guardians erroneously believe that memory losses erase addiction from the WKS brain. Others aver that their relative “deserves a drink now and then” after years of sobriety. More malignant motivations may emerge if an inheritance is involved.
“A married man continued working for several years following onset of WKS syndrome. During that time, he received monitored disulfiram treatment, but no AA or other recovery interventions. When he reached retirement age, his wife and adult offspring discussed his resuming alcohol use, believing that they could limit his intake to one standard drink daily. Contrary to their expectations, the patient began obsessing about alcohol and finding means to obtain it. He died 2 years later from liver failure.”
“A single man was placed in an abstinence-oriented adult foster home following diagnosis of WKS syndrome. The patient requested that his court-appointed guardian transfer him to a program where he could resume drinking. His guardian ignored clinical recommendations and approved transfer to a ‘wet’ setting, where the patient received a cash sum weekly and was permitted to leave the premises to purchase beverage alcohol. He resumed daily drinking and died within 1 year from bleeding esophageal varices and liver failure.”
Courts handling WKS cases might consider appointing two guardians, one for person and the other for property, so that these interests can be separated, monitored, and arbitrated when necessary. Courts can add convenances for vulnerable adults at mortal risk to self and others if readdicted.
Comorbid Brain Conditions
Additional brain lesions in WKS may ensue from falls, fights, obstructive sleep apnea, alcohol-drug overdoses, disuse atrophy of neglected brain centers, infections, and neoplasms. Among 63 hospitalized WKS patients, 13 (or 21%) also had alcohol dementia (Cutting, 1978). A 2-year longitudinal study demonstrated that WKS-only patients maintained stable cognition and function over time, whereas Alzheimer and vascular dementia patients with WKS experienced progressive decline (Oslin and Cary, 2003).
WKS Research
Unique aspects of WKS pathophysiology present opportunities for creative research. One team compared two autobiographical interviewing methods to facilitate optimal assessment in WKS versus controls (Rensen et al., 2017). The two methods did not differ within the WKS subsample, although differences did exist between WKS patients versus controls.
Surgical decompression of supracellar masses producing acute WKS syndromes in four patients rapidly reversed the syndrome (Savastano et al., 2018), enhancing knowledge regarding WKS pathophysiology. A translational rodent model showed neuroanatomical substrates for memory loss resembling, although not wholly replicating human WKS (Savage et al., 2012). A European center found that WKS patients started on supplemental vitamin D within the previous year had a higher cancer rate than other WKS patients at p < 0.011 (Wijnia et al., 2019). Logistic regression analysis further revealed that tobacco smoking and length of stay significantly increased the odds ratios for cancer in WKS patients (respectively, odds ratio = 2.74 and 1.68). An in vitro study of cancer cells indicated that, after hypoxic stress, supplemental thiamine increased cancer cell growth—an effect that may theoretically be reversed by cell-permeable antioxidants (Jonus et al., 2018). The ability to conduct ethical studies among large numbers of WKS patients has begun to guide their treatment.
Clinical and research protocols increasingly use specific test batteries for WKS (see Table 2). Some tests have shown minimal or no pathology in some WKS cases; the Montreal Cognitive Assessment is an example. Such exploratory work holds promise for improved understanding and care of WKS patients.
DISCUSSION
Historical Changes
Underdiagnosis of WKS poses a worldwide problem today (Barata et al., 2020; Donnelly, 2017; Nikolakaros et al., 2018; Sechi and Serra, 2007; Wijnia et al., 2014). The Australian autopsy study suggests that 80% to 90% of WKS decedents are not identified premortem (Harper et al., 1998). Clinicians miss WKS diagnoses among patients presenting to privileged settings where AUD and WKS are not anticipated (Isenberg-Grzeda et al., 2012; Westermeyer and Soukup, 2021). In addressing this issue, Holland furnishes an outstanding model, with a national WKS research center (Wijnia et al., 2016), regional clinical centers (Gerridzen et al., 2021), and world-class experts in several fields of WKS study and service (Wijnia et al., 2014).
If the life-time WKS incidence were 1% and their mean survival time was one decade, then the 331 million people living in the United States during mid-2020 would produce 3.3 million WKS cases and 33.1 million years of WKS morbidity over their lifetimes. Although these data are not based on known information, neither are they exorbitant based on extrapolations from Australia and Europe.
Limitations, Implications, and Opportunities
Without sufficient epidemiological understanding, targeted health planning remains stymied. Several health measures suggest a serious and evolving problem worldwide in countries like ours. These data include increasing alcohol consumption among middle-class people aged 50 to 70 years and case reports of missed diagnoses in otherwise first-class medical facilities. The stigma associated with AUD can sway many people toward secrecy even when the diagnosis is known.
Training in WKS prevention, early diagnosis, and timely care needs to expand into the several disciplines that nowadays provide acute health care for AUD patients, that is, nursing, social work, psychology, physician assistants, police, and jail guards. Within clinical services, instruction on WKS prevention and care should involve emergency departments, consultation-liaison services, anesthesiology, surgical specialties, infectious disease and gastroenterology medicine, neurology, psychiatry, and addiction medicine (Barata et al., 2020; Donnelly, 2017; Donnino et al., 2007). Perhaps most heartrending, every new case of WKS syndrome creates a costly tragedy that could have been prevented by timely administration of a vitamin.
ACKNOWLEDGMENT
Brian A. Conn at the Minneapolis VAHCC library greatly enhanced our literature search through his skills in accessing special databases, international publications, and articles available online but not yet in printed form. We also appreciate the perspectives and experiences conveyed to us by families of our WKS patients.
DISCLOSURE
The authors declare no conflict of interest.

    REFERENCES
    Barata PC, Serrano R, Afonso H, Luís A, Maia T (2020) Wernicke-Korsakoff syndrome: A case series in liaison psychiatry. Prim Care Companion CNS Disord. 22:19br02538.[Context Link]
    Bermejo-Velasco EB, Ruiz-Huete C (2006) Korsakoff's syndrome secondary to left thalamic bleeding. Neurologia. 21:733–736.[Context Link]
    
    Blazer DG, Wu LT (2011) The epidemiology of alcohol use disorders and subthreshold dependence in a middle-aged and elderly community sample. Am J Geriatr Psychiatry. 19:685–694.[Context Link]
    Brion M, DeTimery P, Mertens de Wilmars S, Maurage P (2018a) Impaired affective prosody decoding in severe alcohol use disorder and Korsakoff syndrome. Psychiatry Res. 264:404–406.[Context Link]
    Brion M, Dormal V, Lannoy S, Mertens S, de Timary P, Maurage P (2018b) Imbalance between cognitive systems in alcohol-dependence and Korsakoff syndrome: An exploration using the alcohol flanker task. J Clin Exp Neuropsychol. 40:820–831.[Context Link]
    Brion M, Pitel AL, Beaunieux H, Maurage P (2014) Revisiting the continuum hypothesis: Toward an in-depth exploration of executive functions in Korsakoff syndrome. Front Hum Neurosci. 8:498.[Context Link]
    
    
    de Joode EA, van Boxtel MP, Hartjes P, Verhey FR, van Heugten CM (2013) Use of an electronic cognitive aid by a person with Korsakoff syndrome. Scand J Occup Ther. 20:446–453.[Context Link]
    Diener HC, Dichgans J, Bacher M, Guschlbauer B (1984) Improvement of ataxia in alcoholic cerebellar atrophy through alcohol abstinence. J Neurol. 231:258–262.[Context Link]
    Donnelly A (2017) Wernicke-Korsakoff syndrome: Recognition and treatment. Nurs Stand. 31:46–53.[Context Link]
    Donnino MW, Vega J, Miller J, Walsh M (2007) Myths and misconceptions of Wernicke's encephalopathy: What every emergency physician should know. Ann Emerg Med. 50:715–721.[Context Link]
    El Haj M, Nandrino JL (2017) Phenomenological characteristics of autobiographical memory in Korsakoff's syndrome. Conscious Cogn. 55:188–196.[Context Link]
    
    Favazza AR, Thompson JJ (1984) Social networks of alcoholics: Some early findings. Alcohol Clin Exp Res. 8:9–15.[Context Link]
    Galanter M, Talbott D, Gallegos K, Rubenstone E (1990) Combined alcoholics anonymous and professional care for addicted physicians. Am J Psychiatry. 147:64–68.[Context Link]
     
    Gerridzen IJ, Hertogh CM, Joling KJ, Veenhuizen RB, Vershuur EM, Janssen T, Depla MF (2021) Caregivers' perspectives on good care for nursing home residents with Korsakoff syndrome. Nurs Ethics. 28:358–371.[Context Link]
    Gerridzen IJ, Joling KJ, Depla MF, Veenhuizen RB, Verschuur EML, Twisk JWR, Hertogh CMPM (2019) Awareness and its relationships with neuropsychiatric symptoms in people with Korsakoff syndrome or other alcohol-related cognitive disorders living in specialized nursing homes. Int J Geriatr Psychiatry. 34:836–845.[Context Link]
    Gorden AJ, Zrull M (1991) Social networks and recovery: One year after inpatient treatment. J Subst Abuse Treat. 8:143–152.[Context Link]
    Hall GB, Nelson G (1996) Social networks, social support, personal empowerment, and the adaptation of psychiatric consumers/survivors: Path analytic models. Soc Sci Med. 43:1743–1754.[Context Link]

    Harper C, Fornes P, Duyckaerts C, Lecomte D, Hauw JJ (1995) An international perspective on the prevalence of the Wernicke-Korsakoff syndrome. Metab Brain Dis. 10:17–24.[Context Link]
    Harper C, Gold J, Rodriquez M, Perdices M (1989) The prevalence of the Wernicke-Korsakoff syndrome in Sydney, Australia: A prospective necropsy study. J Neurol Neurosurg Psychiatry. 52:282–285.[Context Link]
    
    Irving W (1819–1820) The sketch book. Chicago: Encyclopedia Britannica.[Context Link]

    Isenberg-Grzeda E, Kutner HE, Nicholson SE (2012) Wernicke-Korsakoff-syndrome: Under-recognized and under-treated. Psychosomatics. 53:507–516.[Context Link]
    Jonus HC, Hanberry BS, Khatu S, Kim J, Luesch H, Dang LH, Bartlett MG, Zastre JA (2018) The adaptive regulation of thiamine pyrophosphokinase-1 facilitates malignant growth during supplemental thiamine conditions. Oncotarget. 9:35422–35438.[Context Link]
   
    Kok AF (1991) Developments in the care of Korsakoff patients. Tijdschrift voor Alcohol, Drugs, en Andere Psychotopic Stoffen. 17:3–9.[Context Link]
    Koob GF (2017) The dark side of addiction. J Nerv Ment Dis. 205:270–272.[Context Link]
    Korsakoff SS (1889) Etude medical-psychologique sur une forme des maladies de la memoire. Revue Philosophie. 20:501–530.[Context Link]
    Labuddha K, von Rothkrech N, Pawlikowski M, Laier C, Brand M (2010) Categorization abilities for emotional and nonemotional stimuli in patients with alcohol-related Korsakoff syndrome. Cogn Behav Neurol. 23:89–97.[Context Link]
    Lloyd B, Oudman E, Altgassen M, Postma A (2019) Smartwatch aids time-based prospective memory in Korsakoff syndrome: A case study. Neurocase. 25:21–25.[Context Link]
    Luykx HJ, Dorresteijn LD, Haffmans PM, Bonebakker A, Kerkmeer M, Hendriks VM (2008) Rivastigmine in Wernicke-Korsakoff's syndrome: Five patients with rivastigmine showed no more improvement than five patients without rivastigmine. Alcohol Alcohol. 43:70–72.[Context Link]

    Nardone R, Bergmann J, De Blasi P, Kronbichler M, Kraus J, Caleri F, Tezzon F, Ladurner G, Golaszewski S (2010) Cholinergic dysfunction and amnesia in patients with Wernicke-Korsakoff syndrome: A transcranial magnetic stimulation study. J Neural Transm (Vienna). 117:385–391.[Context Link]
    Nikolakaros G, Kurki T, Paju J, Papageorgiou SG, Vataja R, Ilonen T (2018) Korsakoff syndrome in non-alcoholic psychiatric patients. Variable cognitive presentation and impaired frontotemporal connectivity. Front Psychiatry. 31:204.[Context Link]
    Nikolakaros GT, Kurki T, Myilymäki A, Ilonen T (2019) A patient with Korsakoff syndrome of psychiatric and alcoholic etiology presenting as DSM-5 mild neurocognitive disorder. Neuropsychiatr Dis Treat. 15:1311–1320.[Context Link]
    O'Brien CP, Childress AR, McClellan AT, Ehrman R (1992) Classical conditioning in drug-dependent humans. Ann N Y Acad Sci. 654:400–415.[Context Link]
    Oslin DW, Cary MS (2003) Alcohol-related dementia: Validation of diagnostic criteria. Am J Geriatr Psychiatry. 11:441–447.[Context Link]
    Oudman E, Postma A, Van der Stigchel S, Appelhof B, Wijnia JW, Nijboer TC (2014) The Montreal Cognitive Assessment (MoCA) is superior to the mini mental state examination (MMSE) in detection of Korsakoff's syndrome. Clin Neuropsychol. 28:1123–1132.[Context Link]
    Pattison EM (1977) Clinical social systems interventions. Psychiatry Dig. 38:25–33.[Context Link]
    Pitel AL, Beaunieux H, Guillery-Girard B, Witkowski T, de la Sayette V, Viader F, Desgranges B, Eustache F (2009) How do Korsakoff patients learn new concepts?Neuropsychologia. 47:879–886.[Context Link]
      
    Rensen YCM, Egger JIM, Westhoff J, Walvoort SJW, Kessels RPC (2019) The effect of errorless learning on psychotic and affective symptoms, as well as aggression and apathy in patients with Korsakoff's syndrome in long-term care facilities. Int Psychogeriatr. 31:39–47.[Context Link]
    Rensen YCM, Kessels RPC, Migo EM, Wester AJ, Eling PATM, Kopelman MD (2017) Personal semantic and episodic autobiographical memories in Korsakoff syndrome: A comparison of interview methods. J Clin Exp Neuropsychol. 39:534–546.[Context Link]
    
    Robin F, Moustafa M, El Haj M (2020) The image of memory: Relationship between autobiographical memory and mental imagery in Korsakoff syndrome. Appl Neuropsychol Adult. 1–7. doi: 10.1080/23279095.2020.1716759.[Context Link]
    
    Savage LM, Hall JM, Resende LS (2012) Translational rodent models of Korsakoff syndrome reveal the critical neuroanatomical substrates of memory dysfunction and recovery. Neuropsychol Rev. 22:195–209.[Context Link]
    Savastano LE, Hollon TC, Barkan AL, Sullivan SE (2018) Korsakoff syndrome from retrochiasmatic suprasellar lesions: Rapid reversal after relief of cerebral compression in 4 cases. J Neurosurg. 128:1731–1736.[Context Link]
        
    Sullivan MA, Birkmayer F, Boyarsky BK, Frances RJ, Fromson JA, Galanter M, Levin FR, Lewis C, Nace EP, Suchinsky RT, Tamerin JS, Tolliver B, Westermeyer J (2008) Uses of coercion in addiction treatment: Clinical aspects. Am J Addict. 17:36–47.[Context Link]
    
    
    Van Dam MJ, Van Meijel B, Postma A, Oudman E (2020) Health problems and care needs in patients with Korsakoff's syndrome: A systematic review. J Psychiatr Ment Health Nurs. 27:460–481.[Context Link]
   
    Visser PJ, Krabbendam L, Verhey FR, Hofman PA, Verhoeven WM, Tuinier S, Wester A, Den Berg YW, Goessens LF, Werf YD, Jolles J (1999) Brain correlates of memory dysfunction in alcoholic Korsakoff's syndrome. J Neurol Neurosurg Psychiatry. 67:774–778.[Context Link]
    Wernicke C (1881) Die acute haemorrhagische poliencophalitis superior. Lehrbuch der Gehirnkkrankheiten fur Aerzle und Studirende (Vol. 2–3, pp 229–242). Berlin: Verlag von Theodor Fischer.[Context Link]
    Westermeyer J (1982) Poppies, pipes, and people: Opium and its use in Laos. Berkeley: University of California Press.[Context Link]
    Westermeyer J (1989) Psychiatric care of migrants: A clinical guide. Washington: American Psychiatric Press.[Context Link]
    Westermeyer J, Neider J (1988) Social networks and psychopathology among substance abusers. Am J Psychiatry. 145:1265–1269.[Context Link]
    
    Wijnia JW, Oudman E, Bresser EL, Gerridzen IJ, van de Wiel A, Beuman C, Mulder CL (2014) Need for early diagnosis of mental and mobility changes in Wernicke encephalopathy. Cogn Behav Neurol. 27:215–221.[Context Link]

    Wijnia JW, Oudman E, Wierdsma AI, Oey MJ, Bongers J, Postma A (2019) Vitamin D supplementation after malnutrition associated with time-related increase of cancer diagnoses: A cohort study of 389 patients with Wernicke-Korsakoff syndrome. Nutrition. 66:166–172.[Context Link]
    Williams CL, Westermeyer J (Eds) (1986) Refugee mental health in resettlement countries. Series in clinical and community psychology. Washington: Hemisphere Publishing Corp.[Context Link]
    Wong A, Moriarity K, Crompton S (2007) Wernicke-Korsakoff syndrome: Who cares?Gastroenterology Today. 17:46–49.[Context Link]