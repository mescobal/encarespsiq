\chapter{Trastornos por consumo de opioides}
\section*{Notas clínicas}
\subsection*{Evaluación del Trastorno por consumo de opioides}

Descartar otros trastorno psiquiátricos.
Descartar otros consumos de sustancias actuales y previos.
En caso de comorbilidad con otros consumos, considerar el tratamiento en nivel superior de atención.

==== Recomendaciones diagnósticas

Confirmar diagnóstico (a pesar de que venga con diagnóstico hecho).

\section*{Encare}
\subsection*{Paraclínica}
\subsubsection{Biológico}
Primer paso: identificar cualquier trastorno médico o psiquiátrico que requiera tratamiento de urgencia y hacer la derivación adecuada (por ejemplo: intoxicación son sobredosis).

HC completa con evaluación de toda condición médica incluyendo las infecciosas (hepatitis, HIV, tuberculosis), traumatismos, embarazo. Examen físico.
\begin{itemize}
	\item Hemograma completo
	\item Funcional y enzimograma hepático
	\item Serología de hepatitis
	\item HIV y VDRL
	\item beta HCG
	\item tóxicos en orina
\end{itemize}
\subsubsection{Psicológico}
Pueden usarse escalas validadas: OOWS (Objective Opioid Withdrawal Scale), SOWS (Subjective Opioid Withdrawal Scale), COWS (Clinical Opioid Withdrawal Scale).
\subsubsection{Social}
Evaluación de barreras de acceso al tratamiento, en especial farmcológico.
Obtención de consentimiento informado para el tratamiento (donde consten aspectos vinculados a riesgos de recaida y otros riesgos vinculados a la retirada gradual como único tratamiento).
\subsection*{Tratamiento}
Encuadre: adicción como enfermedad bio-psico-social-espiritual.
El tratamiento debe ser una decisión compartida con el paciente. Debe contemplarse la preferencia del paciente, y la efectividad de tratamientos previos. Elegir el nivel de tratamiento:
\begin{itemize}
	\item Ambulatorio (OBOT: Office Based Opioid Treatment)
	\item Internación: considerarlo si hay comorbilidad con otros trastornos por consumo de sustancias (alcohol, sedantes, hipnóticos). Considerarlo aún en conexto de consumo que parezca no problemático.
	\item Centro de rehabilitación especializado en adicciones
\end{itemize}

Alternativas footnote:[*Kampman K, Jarvis M. American Society of Addiction Medicine (ASAM) National Practice Guideline for the Use of Medications in the Treatment of Addiction Involving Opioid Use. J Addict Med. 2015;9(5):358-367. doi:10.1097/ADM.0000000000000166
]:
  * Buprenorfina: primera línea
  * Metadona: pacientes bajo supervisión o con falla de tratamiento con buprenorfina. Contexto: internación.
  * Naltrexona: generalmente tiene escasa adherencia. Reservado para pacientes que pueden cumplir.

WARNING: la retirada controlada del opioide como única estrategia no se considera un tratamiento válido.

En opiodes de acción corta, se pueden usar dosis decrecientes diarias de la metadona comenzando con 20-30 mg x día completando la retirada en 6-10 días.

Se recomienda la inclusión de clonidina como coadyuvante para la abstinencia de opioides. No está aprobada por la FDA, pero su uso es extendido: v/o 0.1-0-3 mg c/ 6-7 horas con un máximo de 1.2 mg/día. Debe vigilarse la hipotensión.

Otros fármacos: benzodiacepinas (ansiedad), loperamida (diarrea), acetaminofeno o AINEs (dolor), ondansetron (náuseas).

.Metadona
Tratamiento recomendado para pacientes con dependencia fisiológica de opioides, con capacidad de dar consentimiento y que no tienen contraindicaciones específicas para el tratamiento con agonistas en el contexto de un plan apropiado que incluya intervenciones psicosociales.

Dosis inicial: 10-30 mg con evaluación en 3-4 horas y una eventual segunda dosis no superior a los 10 mg el primera día si persisten los síntomas de abstinencia.
Dosis habitual: 60-120 mg con aumentos de 5-10 mg cada 7 días según respuesta clínica (para evitar sobresedación, toxicidad iatrogénicas). No hay un límite recomendado en el tiempo de tratamiento.
Debe monitorizarse la administración de la medicación hasta que clínicamente sea apropiado prescribir sin monitorización.

Falta de eficacia: considerar buprenorfina. Primero se debe bajar la dosis de metadona si la dosis es superior a 40 mg/día.

Cambio a naltrexona: primero debe retirarse por completo la metadona y otros opioides ANTES de recibir naltrexona (esto puede llevar 14 días) -> contexto de instalación especializada.


.Buprenorfina
Con la buprenorfina: el paciente debe esperar a presentar síntomas de abstinencia leves a moderados antes de tomar la primera dosis de buprenorfina para reducir el riesgo de precipitación de la abstinencia.

Inducción: comenzar con 2-4 mg, con aumentos de a 2-4 mg (preferentemente en contexto hospitalario).

Luego de que se establece la buena tolerancia de la dosis inicial, puede aumentarse de forma rápida a dosis que sean eficaces para 24 horas (de a 8 mg x día). Máximo: 24 mg/día.

Luego del alta: controles semanales al inicio. Prescripción de cantidades controladas de comprimidos.

Discontinuación: lenta, de duración indefinida y con controles frecuentes (aún después de la suspensión). Duración: meses.

Cambio a naltrexona: debe mediar un lapso de 7-14 días SIN buprenorfina antes de empezar con Naltrexona (no debe haber dependencia). Puede hacer un test de dosis inicial para comprobar que no haya dependencia física.

Cambio a metadona: sin interrupción. La adición de un agonista completao a un agonista parcial en general no da reacciones adversas.

.Naltrexona
Se recomienda para la prevención de recaídas de un TCO.

===== Otros
Durante el tratamiento: test de tóxicos en orina.
Ofrecer vacunación para hepatitis.
Ofrecer asesoramiento y tratamiento para cesación de tabaco.

===== Psicosocial

Orientación y apoyo con énfasis en mantener el cumplimiento luego del alta (pacientes que no tienen seguimiento recaen con más frecuencia).

Contactar con grupos de apoyo para pacientes y familiares.

Paciente y familiares deben recibir psicoeducación en cuanto a riesgo de sobredosis si retoma el uso de opioides luego del tratamiento. En Uruguay solo hay v/o.
Dosis: 50 mg x dia (350 mg x semana) o repartido en la semana en 3 tomas (100, 100, 150).
No hay una duración de tratamiento recomendad. No hay dependencia física. Se puede suspender de forma abrupta.

Cambio a metadone o buprenorfina: planificado (no antes de 24 horas libre de medicación en la VO). En general menos complejo que el cambio inverso. La dosis inicial de buprenorfina o metadona pueden ser más bajas (al no haber dependencia física).

===== Poblaciones especiales

.Mujeres
Embarazo: si hay dependencia física deben recibir metadona o buprenorfina y no solo retiro de medicación con manejo de abstinencia. Debe comenzarse el tratamiento de forma precoz en el embarazo. Se recomienda hospitalización, en especial en el primer trimestre.

Metadona: en internación, 20-30 mg, sin exceder los 40 mg en el primer día. Dosis incrementales de 5-10 mg cada 3-6 horas según sea necesario para tratar la abstinencia. El emabrazo afecta la farmacocinética de la metadona. A medida que avanza la edad gestacional los niveles de metadona bajan (aumenta el clearance). Puede requerir aumento de dosis y/o la frecuencia (en 2 tomas es más efectivo y tiene menos efectos secundarios).

Buprenorfina (alternativa a la metadona): comenzar cuando hayan sintomas de abstinencia leves-moderados, antes de que sean severos (6 horas aproximadamente luego de la última dosis de un opiode de acción corta y 24-48 horas luego de un opioide de acción prolongada). Se recomienda hospitalización. Luego de la inducción aumentos de 5-10 mg x semana. No requiere de ajuste de dosis. No se recomienda la discontinuación antes de una cesárea electiva ya que aumenta el riesgo de abstinencia fetal.

Debe incluirse a obstetra en el equipo.

Embarazo durante el tratamiento con naltrexona: discontinuarla. Puede continuarse si hay alto riesgo de recaída y con consentimiento informado.

Lactancia: se recomienda estimular la lactancia durante el tratamiento con metadona o buprenorfina.

.Dolor

En los pacientes con dolor es importante contar con un correcto diagnóstico y que se identifiquen alternativas de tratamiento (acetaminofeno, AINEs).

Metadona: los pacientes pueden requerir dosis adicionales de opioides además de la dosis diaria de matadona para el manejo del dolor agudo severo. Pueden requerir opioides de acción corta adicionales para el manejo del dolor postoperatorio.

Buprenorfina: se peude aumentar transitoriamente para el dolor moderado agudo. Para el dolor severo agudo se recomienda discontinuar buprenorfina y comenzar con un opioide alta potencia (como fentanil). Debe controlarse al paciente para evaluar si necesita intervenciones adicionales (tales como anestesia regional).
La decisión de discontinuar la buprenorfina antes de una cirugía electiva debe ser hecha en conjunto con anestesista. De hacerlo debe ser 24-36 horas antes de la cirugía recomenzando en el postoperatorio luego de que no se necesite analgesia con agonistas opioides postoperatoria.

Naltrexona: los pacientes con naltrexona no responen a la analgesia con opioides del modo usual. Se recomienza usar AINEs para dolor leve y ketorolac por períodos cortos en dolor moderado-severo.
La naltrexona oral debe discontinuarse 72 horas antes de una cirugía.

.Adolescentes
Buprenofrina está aprobado para >= 16 años.

.Comorbilidad psiquiátrica
Evaluar existencia de riesgo suicida.
Controlar de forma más estricta a pacientes con antecedentes de IAEs.

.Población carcelaria
Se recomienda tratamiento de forma independiente de la duración de la sentencia.
Debe iniciarse la farmacoterpia al menos 30 días antes de la salida de prisión.

.Sobredosis
En caso de sobredosis debe adminstrarse naloxona (indicado también en embarazadas con sobredosis).
Se recomienda psicoeducación a familiares en el manejo de la naloxona y dar prescripciones con indicaciones de administración en caso de sobredosis.

=== Bibliografía
* Nielsen, S., Larance, B., \& Lintzeris, N. (2017). Opioid agonist treatment for patients with dependence on prescription opioids. Jama, 317(9), 967-968.
* Nielsen, S., Larance, B., Degenhardt, L., Gowing, L., Kehler, C., \& Lintzeris, N. (2016). Opioid agonist treatment for pharmaceutical opioid dependent people. Cochrane Database of Systematic Reviews, (5).
* Center for Substance Abuse Treatment. Medication-Assisted Treatment for Opioid Addiction in Opioid Treatment Programs. Rockville (MD): Substance Abuse and Mental Health Services Administration (US); 2005.
* Veilleux, J. C., Colvin, P. J., Anderson, J., York, C., \& Heinz, A. J. (2010). A review of opioid dependence treatment: pharmacological and psychosocial interventions to treat opioid addiction. Clinical psychology review, 30(2), 155-166.
