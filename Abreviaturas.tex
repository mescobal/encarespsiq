\section{Abreviaturas}
\begin{description}
\item [AAV] Alucinaciones auditivo-verbales
\item [ACV] Accidente cerebrovascular.
\item [AF] Antecedentes familiares
\item [AFP] Antecedentes familiares psiquiátricos.
\item [AIT] Accidente isquémico transitorio.
\item [AP] Antecedentes personales.
\item [APPG] Antipsicóticos de primera generación
\item [APSG] Antipsicóticos de segunda generación
\item [APM] Antecedentes personales médicos.
\item [APP] Antecedentes personales psiquiátricos.
\item [AVE] Accidente vascular encefálico
\item [BOTE] Bien orientado en tiempo y espacio.
\item [BZD] Benzodiacepina/s
\item [CB] Conductas basales
\item [CIE-10] Clasificación Internacional de Enfermedades, editado por la Organización Mundial de la Salud, en su 10a edición.
\item [DASA] Delirio Alcohólico Subagudo.
\item [DD] Diagnóstico diferencial.
\item [DM] Diabetes mellitus / Dolor moral (según el contexto).
\item [DOTE] Desorientación temporal.
\item [DSM-IV] Manual diagnóstico y estadístico de los trastornos mentales, editado por la American Psychiatric Association, en su 4ª edición.
\item [EA] Enfermedad actual.
\item [ECT] Electro-convulso-terapia.
\item [EP] Examen psiquiátrico.
\item [EPA] Episodio psicótico agudo.
\item [EPM] Excitación psicomotriz.
\item [EPOC] Enfermedad pulmonar obstructiva crónica.
\item [FC] Frecuencia cardíaca
\item [FR] Frecuencia respiratoria
\item [GGT] Gamma glutamil transferasa
\item [HC] Historia clínica.
\item [HIV] ver VIH
\item [HTA] Hipertensión arterial.
\item [HTEC] Hipertensión endocraneana.
\item [IAE] Intento de autoeliminación.
\item [IAM] Infarto agudo de miocardio.
\item [ICC] Insuficiencia cardíaca congestiva.
\item [IPM] Inhibición psicomotriz.
\item [MC] Motivo de consulta.
\item [MSEC] Medio socio-económico-cultural.
\item [OH] Alcohol, alcoholismo, trastorno por consumo de alcohol.
\item [PA] Presión arterial
\item [PEIC] Procesos expansivos intracraneales.
\item [PMD] Psicosis maníaco depresiva.
\item [PPA] Pronóstico psiquiátrico alejado.
\item [PPI] Pronóstico psiquiátrico inmediato.
\item [PVA] Pronóstico vital alejado.
\item [PVI] Pronóstico vital inmediato.
\item [RAP] Rasgos acentuados de personalidad.
\item [SAM] Síndrome de automatismo mental
\item [SDD] Síndrome disociativo-discordante
\item [SNM] Síndrome neuroléptico malgino
\item [TBC] Tuberculosis.
\item [TC] Tónico-clónica.
\item [TCA] Trastornos de la conducta alimentaria.
\item [TCC] Terapia Cognitivo Comportamental.
\item [TDAH] Trastorno de déficit atencional con hiperactividad.
\item [TGD] Trastornos generalizados del desarrollo
\item [TEC] Traumatismo encéfalo-craneano.
\item [TEPT] Trastorno por estrés postraumático
\item [TOD] Teoría Organodinámica (Ey)
\item [TPA] Tratorno psicótico agudo.
\item [UISP] Uso indebido de sustancias psicoactivas.
\item [VIH] Virus de inmunodeficiencia humana.
\end{description}