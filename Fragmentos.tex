== Fragmentos

En los lugares marcados con el ícono "pegar" (icon:paste[]) pueden insertarse textos similares a los que se encuentran a continuación. No se recomienda la copia textual, porque puede tener un efecto negativo a los ojos del tribunal.

.Neurosis
icon:cut[] Por la presencia de sintomatología en el corte longitudinal de la existencia con alteraciones estructurales de la personalidad, expresiones de un conflicto intrapsíquico, con inhibición de conductas sociales y fondo de ansiedad-angustia. Es egodistónico (provoca malestar), autoplástico (exigencias dirigidas hacia sí mismo, no al entorno). Tiene conciencia de enfermedad, pide ayuda (colabora con la entrevista). En la entrevista hay un buen rapport y no se evidencia pérdida del juicio de realidad.

.Nivel en diferido
icon:cut[] Lo evaluaremos luego de remitido el cuadro actual, sobre la base de escolaridad, logros, entrevistas, entrevistas con terceros. Oportunamente valoraremos la necesidad de realización de test de nivel. Sabemos que todos estos elementos están teñidos por el medio socio-económico-cultural. Clínicamente impresiona X.

.Psicosis
icon:cut[] Porque el paciente perdió contacto con la realidad, encontrándose sumido en un mundo propio, incompartible, por él creado, con el cual se relaciona de una manera nueva y del cual no puede sustraerse voluntariamente, con pérdida del juicio de realidad, sin conciencia de morbidez, estableciéndose en la entrevista un mal rapport.

.Psicosis aguda
icon:cut[] Por ser una experiencia sensible y actual, de X días de evolución, más vivida que relatada, con alteración importante del humor y la afectividad y con afectación de las conductas basales.
