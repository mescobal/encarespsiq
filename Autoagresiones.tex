=== Autoagresiones

Definición: herida o envenamiento autoprovocado, independientemente del propósito aparente. Excluye: estereotipas autoagresivas.

==== Información y apoyo
Se debe brindar información y apoyo a la persona que se autoagredió. Se debe brindar información a familia y cuidadores. Temas a tratar:

* Qué es la autoagresión
* Qué puede desencadenarla
* Tratamientos disponibles
* Pautas de autocuidado, incluyendo cuando buscar ayuda.
* Cómo proceder con las heridas.
* Qué hacer con las cicatrices.
* Planes de cuidado
* Impacto del estigma en la autoagresión

People have the right to be involved in discussions and make informed decisions
about their care, as described in NICE's information on making decisions about your
care.
Making decisions using NICE guidelines explains how we use words to show the
strength (or certainty) of our recommendations, and has information about
prescribing medicines (including off-label use), professional guidelines, standards
and laws (including on consent and mental capacity), and safeguarding.
The recommendations apply to staff from all sectors that work with people who have self-
harmed, unless a recommendation or section specifically states that it is for a certain
group. Because of the wide variety of criminal justice system settings that exist and the
need to take other relevant national guidance into account, the recommendations in the
guideline may need to be tailored for certain criminal justice system settings during
implementation.
Putting recommendations into practice can take time depending on how much change in
practice or services is needed. Most of the recommendations in this guideline reinforce
best practice and will not need additional resources to implement if previous guidance has
been followed. If changes to current local practice are needed to implement the
recommendations, they may take time and significant additional resources.
The recommendations apply to all people who have self-harmed, unless a
recommendation specifically states that it is for adults or children and young people only.
1.1 Information and support
1.1.1 Provide information and support for people who have self-harmed. Share
information with family members or carers (as appropriate). Topics to
discuss include:
• what self-harm is
Self-harm: assessment, management and preventing recurrence (NG225)
© NICE 2022. All rights reserved. Subject to Notice of rights (https://www.nice.org.uk/terms-and-
conditions#notice-of-rights).
Page 7 of
81
• why people self-harm and, where possible, the specific circumstances of the
person
• support and treatments available
• self-care (also see recommendation 1.11.12 in the section on harm
minimisation), including when to seek help
• how to deal with injuries
• how to manage scars
• care plans and safety plans, and what they involve
• the impact of encountering stigma around self-harm
• who will be involved in their care and how to get in touch with them
• where appointments will take place
• what to do if they have any concerns
• what do to in an emergency
• local services and how to get in touch with them, including out-of-hours
• local peer support groups, online forums, local and national charities, and how
to get in touch with them.
1.1.2 Provide information and support for the family members or carers (as
appropriate) of the person who has self-harmed. Topics to discuss
include:
• the emotional impact on the person and their family members or carers
• advice on how to cope when supporting someone who self-harms
• what to do if the person self-harms again
• how to seek help for the physical consequences of self-harm
• how to assist and support the person
• how to recognise signs that the person may self-harm
Self-harm: assessment, management and preventing recurrence (NG225)
© NICE 2022. All rights reserved. Subject to Notice of rights (https://www.nice.org.uk/terms-and-
conditions#notice-of-rights).
Page 8 of
81
• steps to reduce the likelihood of self-harm in the future
• support for families and carers and how to access it
• the impact of encountering stigma around self-harm
• local services and how to get in touch with them, including out-of-hours
• local peer support groups, online forums, local and national charities, and how
to get in touch with them
• their right to a formal assessment of their own needs including their physical
and mental health (known as a 'carer's assessment'), and how to access this
(see the NICE guideline on supporting adult carers).
1.1.3 Information for people who have self-harmed and their family members
or carers should be:
• tailored to their individual needs and circumstances, taking into account, for
example, whether this is a first presentation or repeat self-harm, the severity
and type of self-harm, and if the person has any coexisting health conditions,
neurodevelopmental conditions or a learning disability
• provided throughout their care
• sensitive and empathetic
• supportive and respectful
• consistent with their care plan, if there is one in place
• conveyed in the spirit of hope and optimism.
For more guidance on communication, providing information (including different
formats) and shared decision making, see the NICE guidelines on shared
decision making, service user experience in adult mental health, patient
experience in adult NHS services and babies, children and young people's
experience of healthcare.
1.1.4 Recognise that support and information may need to be adapted for
people who may be subject to discrimination, for example, people who
are physically disabled, people with neurodevelopmental conditions or a
Self-harm: assessment, management and preventing recurrence (NG225)
© NICE 2022. All rights reserved. Subject to Notice of rights (https://www.nice.org.uk/terms-and-
conditions#notice-of-rights).
Page 9 of
81
learning disability, people from underserved groups, people from Black,
Asian and minority ethnic backgrounds and people who are LGBTQ+.
For a short explanation of why the committee made these recommendations and how
they might affect practice, see the rationale and impact section on information and
support.
Full details of the evidence and the committee's discussion are in:
• evidence review A: information and support needs of people who have self-
harmed
• evidence review B: information and support needs of families and carers of people
who have self-harmed.
1.2 Consent and confidentiality
1.2.1 Healthcare professionals and social care practitioners who have contact
with people who self-harm should be able to:
• understand when and how to apply the principles of the Mental Capacity Act
2005 and its Code of Practice, the Mental Health Act 2007 and its Code of
Practice, and the Care Act 2014 and the Care Act 2014 statutory guidance
• assess mental capacity
• make decisions about when treatment and care can be given without consent
• understand when and how to seek further guidance about consent to care
• direct people to independent mental capacity advocates (IMCAs).
Also see the NICE guidelines on decision making and mental capacity, service
user experience in adult mental health, and babies, children and young
people's experience of healthcare, and the Department of Health and Social
Care's consensus statement on information sharing and suicide prevention.
1.2.2 Healthcare professionals and social care practitioners who have contact
Self-harm: assessment, management and preventing recurrence (NG225)
© NICE 2022. All rights reserved. Subject to Notice of rights (https://www.nice.org.uk/terms-and-
conditions#notice-of-rights).
Page 10
of 81
