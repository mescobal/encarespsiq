== Estupor

=== Notas clínicas

Esto es un escueto resumen de un encare de un paciente que se presenta con un estupor catatónico en una esquizofrenia. El resto de los datos los completan con elementos de otras historias. Tomar en cuenta que un estupor puede presentarse en diferentes patologías (estupor melancólico, histérico, esquizofrenia, confusional).

=== Encare

==== Agrupación sindromática

1. síndrome de inhibición psicomotriz Permanente, cotidiano, grave. Estupor.
2. síndrome conductual • Impulsivo: IAE, Crisis de EPM con hetero-agresividad, sin reflexión / meditación. • Conductas basales y pragmatismos.
3. síndrome disociativo-discordante IDEA Hacemos diagnóstico de síndrome Catatónico, máxima discordancia psicomotriz: reducción de iniciativa motriz, máxima inhibición psicomotriz (estupor catatónico), con hipomimia, hipogestua-lidad, clinofilia. Sobre este fondo se presentan descargas motoras enigmáticas y absurdas (IAE, hetero): • Inhibición psicomotriz • Descarga impulsiva • Negativismo-oposicionismo

==== Diagnóstico positivo

Psicosis icon:arrow-circle-right[] Psicosis Crónica icon:arrow-circle-right[] Esquizofrenia icon:arrow-circle-right[] En período de estado icon:arrow-circle-right[] Tipo catatónica icon:arrow-circle-right[] Descompensada

==== Diagnósticos diferenciales

. Otras causas de estupor:
.. Depresivo: más lento, precedido de síntomas afectivos, AP o AF afectivos.
.. Confusional: organicidad (fiebre, etc.), no existe catalepsia
.. Histérico: se jerarquiza el mutismo, pero se comunica por otros medios.
. Depresión psicótica en esquizofrenia: (depresión postpsicótica), donde existen síntomas depresivos y no existen síntomas discordantes. Clinofilia y disminución de la expresividad van por cuenta de síntomas negativos.
. Otras formas de esquizofrenia: hebefrénica (predomina la jovialidad pueril y trastornos conductuales), paranoide (predomina el delirio).
. Otras psicosis crónicas: Parafrenia (30-50 años, delirio polimorfo fantástico, persecutorio, megalomaníaco, a mecanismo imaginativo, para-lógico, bipolaridad), paranoia (ambos carecen de evolución deficitaria, no condicionando retirada a un mundo autista).

==== Paraclínica

Descartar lo orgánico.

==== Tratamiento

Antipsicóticos. ECT.

==== Evolución y pronóstico

MALO por estupor, actos impulsivos. Potencialmente mortal por deshidratación, infec-ciones, trastornos hidroelectrolíticos con arritmias.
