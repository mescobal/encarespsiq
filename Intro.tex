\documentclass[encares.tex]{subfiles}
\begin{document}
\subsection*{¿Que es?}
Este conjunto de notas sobre encares de psiquiatría tiene su origen en la información que fui colocando en Internet desde que comencé a preparar la prueba de residencia de psiquiatría (año 1996) junto con los Dres. Elbio Paullier y Daniel Hazan. En ese momento me sorprendió la solidaridad entre quienes preparaban la prueba y la generosidad de los postgrados, residentes y psiquiatras ya recibidos que ponían el material propio a disposición.

La construcción colectiva del conjunto de conocimientos que implica un encare era totalmente concordante con la filosofía del Software Libre \footnote{https://www.fsf.org} a la que le debemos GNU/Linux y el concepto de Copyleft. Como forma de contribuir a esa construcción colectiva, fui subiendo todo el material, junto con mis propios apuntes a un sitio: www.sitiosaludmental.com. Esto siguió durante un tiempo.
\begin{wrapfigure}{r}{0.5\textwidth}
	\centering
	\includegraphics{copyleft.png}
\end{wrapfigure}
Hace algunos años atrás, me comentaron acerca de la existencia del "Libro de las estrellitas". Alguien, siguiendo la misma filosofía, imprimió los encares, los encuadernó y puso a disposición en una fotocopiadora frente al Hospital de Clínicas. Alguien (no se si fue la misma persona) adornó la tapa con estrellitas.

Mirando el material 24 años después, resulta asombroso cuántas cosas están vigentes y cuántas cosas ya no. Entonces decidí retomar la tarea.

Junto con la filosofía del Software Libre, viene el concepto de ``liberar frecuentemente, liberar rápido'' (``release often, release soon'') \footnote{RAYMOND, Eric. The cathedral and the bazaar. Knowledge, Technology \& Policy, 1999, vol. 12, no 3, p. 23-49.}. Esto quiere decir que muchas cosas son una construcción permanente y no hay que esperar a que estén perfectas para comenzar a usarlas. Siguiendo esa línea, este PDF se libera en formato borrador y va a tener una nueva versión de forma frecuente. El número de versión sigue el formato año.mes-revisión.

\subsection*{Etapas del trabajo}
\begin{itemize}
\item Etapa 1 (actualmente en curso): \faBatteryHalf (40\%) juntar el contenido de "El libro de las estrellitas" con otros encares (añejos) que tengo en la computadora. La información que puede resultar útil es: la estructura del encare, las nociones clásicas y alguna nota clínica. La parte de tratamiento y las nociones más biológicas pueden estar groseramente desactualizadas. Hay mucha información duplicada / redundante / obsoleta.
\item Etapa 2: \faBatteryEmpty (4\%) eliminar información duplicada.
\item Etapa 3: \faBatteryEmpty (2\%) actualizar información obsoleta (primero tratamiento y luego etiopatogenia)
\item Etapa 4: \faBatteryEmpty (2\%) pulir estructura / diagramación.
\item Etapa 5: \faBatteryEmpty (0\%) actualizar sistemas de clasificación.
\end{itemize}
\faStop No lo impriman!!! Sale una versión nueva aproximadamente cada 15 días.
\subsubsection*{Cambios}
Ultimos cambios:
\begin{itemize}
\item 07/21: agrego información al encares sobre TDAH
\item 06/21: agrego cosas al encare de Trastornos del Espectro Autista
\item 05/21: cambios en el encare de F10
\item 04/21: inclusión del contenido de F32 Trastorno depresivo
\item 03/21: inclusión del contenido de F23 Episodio Psicótico agudo
\item 02/21: citas bibliográficas F32
\item 01/21: inclusión del contenido de F44 Trastorno conversivo-disociativo
\item 2020: inclusión de Trastornos por consumo de opiáceos
\end{itemize}
\subsection*{Colaboración}
Estos encares fueron posibles gracias a la generosa contribución de varios colegas (entre ellos los Dres.: Eduardo Curbelo, Daniel Hazan, Jorge Panizo, Elbio Paullier). Es muy importante contar el aporte de todos. Ruego enviar toda corrección, sugerencia, aporte a mi dirección de correo. En caso de aportar modificaciones, se valorará que incluyan alguna referencia bibliográfica.
\subsection*{Licencia}
Todo el contenido tiene licencia Creative Commons (Atribución \- Compartir igual) 4.0 Internacional \footnote{http://creativecommons.org/licenses/by-sa/4.0/}.

\begin{wrapfigure}{r}{0.5\textwidth}
	\centering
	\includegraphics{cc.png}
\end{wrapfigure}

Esto significa que se puede:
\begin{itemize}
\item Compartir: copiar y redistribuir el material en cualquier medio o formato
\item Adaptar: remezclar, transformar y construir a partir del material para cualquier propósito, incluso comercialmente.
\end{itemize}
Bajo los siguientes términos:
\begin{itemize}
\item Atribución: se debe dar crédito de manera adecuada, brindar un enlace a la licencia, e indicar si se han realizado cambios.
\item Compartir igual: si se remezcla, transforma o crea a partir del material, se debe distribuir la contribución bajo la la misma licencia del original.
\end{itemize}
\subsection*{El encare}
El encare es (parece obvio pero hay que decirlo) sobre un paciente. Es una prueba CLINICA. No es un "fill in the blanks". Hay que "ver" al paciente y luego hacer el encare. Por lo tanto:
\begin{itemize}
\item No hablar en el aire. Adaptarlo al paciente.
\item Ver criterios del tribunal.
\item Tomarse el tiempo de escritura/exposición: de nada sirve saber, si no se expone en el tiempo estipulado.
\end{itemize}
Notas con respecto a DSM y proceso diagnóstico: recordar que el DSM y la CIE son manuales de clasificación. Primero hay que hacer un diagnóstico y luego ver en qué categoría entra ese diagnóstico. Es frecuente ver la orientación de un encare hacia una categoría diagnóstica. Muchas veces los pacientes reales desafían las clasificaciones.

Notas con respecto a mezclar nosografías: aquí hay dos posturas antagónicas. Para algunos es pecado mezclar nosografías y lo penalizan con la descalificación. Para otros, prescindir de algunas nosografías es un empobrecimiento de la psiquiatría.

Estoy a favor de lo segundo, pero a su vez estoy de acuerdo con que no hay que mezclar. Si planteamos un diagnóstico siguiendo una nosografía, no podemos plantear diagnósticos diferenciales siguiendo otra. El problema es que la alternativa (mantener dos diagnósticos y 2 grupos de diferenciales, por ejemplo) consume mucho tiempo.

\subsection*{Apéndices}

Al final hay una lista de las abreviaturas más usadas, así como fragmentos de texto que pueden ser usados como base para armar algunas partes de los encares que suelen tener una estructura estandarizada.

\subsection*{Material de estudio}
\subsubsection*{Material básico}

El material mínimo para preparar la prueba es:
\begin{itemize}
\item Un tratado de psiquiatría (quizás sea más práctico en forma de compendio). Uno de los más usados en este texto es el compendio de Kaplan \& Sadock \footnote{SADOCK, Benjamin J. Kaplan \& Sadock. Sinopsis de psiquiatría. Wolters Kluwer Health, 2016.}.
\item El tratado clásico más popular en Uruguay (es una característica local, no está extendido en el mundo): Ey\cite{ey1996}.
\item Un manual de clasificación de enfermedades mentales. Si bien entre los docentes es popular el DSM-IV \footnote{America Psychiatric Association. DSM-IV-TR: Manual diagnóstico y estadístico de los trastornos mentales. American Psychiatric Pub, 2008.}, hay que recordar que todos los documentos vinculados a salud se guían por la guía de la OMS: CIE-10 \footnote{WORLD HEALTH ORGANIZATION, et al. Guía de bolsillo de la clasificación CIE-10: clasificación de los trastornos mentales y del comportamiento. 2000.}. El DSM-5 puede no ser popular entre algunos docentes (por buenos motivos).
\item Un texto sobre psicofarmacología. Uno de los más populares es el de Stahl \footnote{STAHL, Stephen M.; STAHL, Stephen M. Stahl's essential psychopharmacology: neuroscientific basis and practical applications. Cambridge university press, 2013.}.
\end{itemize}
\subsubsection*{Material complementario}

Hay material que se usó como base en muchos de los capítulos que siguen, por lo cual no están citados en cada capítulo:
\begin{itemize}
\item Uno de las mejores referencias en psicofarmacología por la abundancia de tablas y accesibilidad de la información es el "Clinical Handbook of Psychotropic Drugs" \footnote{PROCYSHYN, Ric M.; BEZCHLIBNYK-BUTLER, Kalyna Z.; JEFFRIES, J. Joel (ed.). Clinical handbook of psychotropic drugs. Hogrefe Verlag, 2019.}.
\end{itemize}
\end{document}