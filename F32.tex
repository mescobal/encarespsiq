\chapter{Depresión}
\section*{Notas clínicas}
\subsection*{Depresión en el adulto mayor}
Es un constructo clínicamente útil\footnote{Agüera-Ortiz, L., Claver-Martín, M. D., Franco-Fernández, M. D., López-Álvarez, J., Martín-Carrasco, M., Ramos-García, M. I., \& Sánchez-Pérez, M. (2020). Depression in the Elderly. consensus statement of the Spanish Psychogeriatric Association. Frontiers in psychiatry, 11, 380.}.
Otros conceptos clínicos útiles: depresión de inicio tardío, depresión de causa vascular.
Depresión en el adulto mayor:
\begin{itemize}
 \item Mayores niveles de ansiedad.
 \item Mayores niveles de síntomas hipocondríacos
 \item Más asociado a ideación suicida
 \item Más impacto en el funcionamiento.
 \item Asociado a transición de autonomía a dependencia
 \item La demencia es un factor de riesgo significativo
 \item Las enfermedades somáticas graves son un factor de riesgo significativo para el suicidio en varones más que en mujeres
 \item Tratamiento de primera línea: ISRS. Realizar antes rutinas y ECG. Considerar duales. La latencia del antidepresivo es mayor que en los más jóvenes.
 \item Si hay síntomas psicóticos, hay mayor riesgo de demencia y de suicidio.
\end{itemize}
Depresión psicótica en adultos mayores: antidepresivos (duales) + antipsicóticos. Segunda línea: ECT. Mantenimiento: si respondió a ECT -> ECT de mantenimiento + fármacos. Se sugiere no usar litio, ni mantener los antipsicóticos durante el mismo lapso que los antidepresivos.
Los antidepresivos son eficaces para el tratamiento de la depresión en pacientes con demencia.
Depresión vascular y depresión en demencia: puede usarse ECT.

.Recomendaciones según situación clínica

[options="header"]
|==================
|Situación|1ª línea | 2ª línea | Evitar
|Cardiopatía isquémica | SERT, AGOM | DESV, VENL, FLUV, VORT, MIRT|CITA, ESCI, REBO, NORT
|Arritmias|SERT, AGOM|DESV, FLUO, FLUV, VORT, MIRT|CITA, ESCI, REBO, NORT, BUPR
|Hipertensión|SERT, AGOM|CITA, ESCI, FLUO, FLUV, PARO, VORT, MIRT|VENL
|Anticoagulación| |DESV, VENL, DULO, SERT, CITA, ESCI, MIRT, BUPR, AGOM, VORT|
|Diabetes|| DESV, VENL, DULO, FLUO, SERT, CITA, ESCI, AGOM, VORT, BUPR, REBO|
|Dislipidemia| | FLUO, SERT, CITA, ESCI, DULO, BUPR, VORT, AGOM|
|Obesidad| FLUO, BUPR, AGOM | DESV, VENL, DULO, CITA, ESCI, FLUV, SERT, VORT, REBO | MIRT
|Adelgazamiento, anorexia| MIRT | DESV, VENL, DULO, PARO, SERT, CITA, ESCI, NORT | FLUO
|Constipación| | FLUO, FLUV, SERT, CITA, ESCI, BUPR, AGOM, VORT |
|Hemorragia digestiva | | DESV, VENL, MIRT, BUPR, NORT | FLUO, CITA, ESCI
|Hiponatremia| | DESV, VENL, DULO, MIRT, BUPR, AGOM, VORT|FLUO, CITA, ESCI
|Mareos|BUPR|DESV, VENL, FLUO, SERT, CITA, ESCI, REBO, AGOM, VORT| MIRT
|Caídas| | DESV, VENL, DULO, FLUO, SERT, CITA, ESCI, BUPR, VORT|NORT
|Alcoholismo| |DESV, VENL, DULO, FLUO, PARO, SERT, CITA, ESCI, MIRT, BUPR, VORT|
|Disfunción sexual| AGOM, BUPR, MIRT | REBO VORT | FLUO PARO
|Glaucoma| | DESV, SERT, CITA, ESCI, BUPR, AGOM, VORT, TIAN|NORT, VENL
|Convulsiones | | MIRT | NORT, BUPR
|ACV| | DESV, SERT, CITA, ESCI, AGOM, VORT, MIRT |
|Parkinson | BUPR | DESV, VENL, DULO, SERT, AGOM, VORT, MIRT, TIAN|
|Demencia| | DESV, VENL, DULO, SERT, CITA, ESCI, BUPR, AGOM, VORT, MIRT | NORT
|Dolor| DULO, DESV, VENL | NORT |
|==================

.Algoritmo
Resistencia al tratamiento:

. Escalar dosis
. Cambio de antidepresivo
. Combinación de AD
. Potenciación con antipsicótico o con lamotrigina
. Potenciación con litio
. Potenciación con un agonista dopaminérgico o metilfenidato

==== Depresión en adolescentes

- Escitalopram \footnote{EMSLIE, Graham J., et al. Escitalopram in the treatment of adolescent depression: a randomized placebo-controlled multisite trial. Journal of the American Academy of Child \& Adolescent Psychiatry, 2009, vol. 48, no 7, p. 721-729.]} \footnote{FINDLING, Robert L.; ROBB, Adelaide; BOSE, Anjana. Escitalopram in the treatment of adolescent depression: a randomized, double-blind, placebo-controlled extension trial. Journal of child and adolescent psychopharmacology, 2013, vol. 23, no 7, p. 468-480.}
\section*{Encare}
\subsection*{En suma}
Paciente de X años, procedente de MSEC X, con AF de trastornos afectivos / alcoholismo / suicidio, con AP de X, que es traído por una cuadro centrado en la esfera afectiva.
\subsection*{Agrupación sindromática}
\subsubsection*{Síndrome depresivo}
De inicio insidioso / permanente, de X evolución, que centra el cuadro clínico, dado por alteraciones en: el humor y afectividad , psicomotricidad, pensamiento, conductas basales y pragmatismos

.Humor y afectividad
Trastorno que centra el cuadro y del cual nacen las otras alteraciones. Está dado por un humor en menos cualitativamente distinto a la tristeza normal. Está evidenciado por una tristeza vital y profunda transcurso de la entrevista, monótona, que no responde a estímulos externos, que no calma con la entrevista, con polo matinal de peoría, con correlato en la presentación (facies triste, omega melancólico), con deseos de llorar sin poder hacerlo o con llanto intenso. Se acompaña de una anhedonia (pérdida de capacidad para sentir placer con cosas que previamente eran gratas) tanto anticipatoria como de consumación, con apatía (pérdida de interés), desánimo, desgano, abulia (aburrimiento), abatimiento. En la afectividad se presenta un afecto embotado que puede llegar a una anestesia afectiva: incapacidad de querer, de sentir, por fuera del sufrimiento. Hipersensibilidad a acontecimientos no placenteros con insensibilidad para acontecimientos placenteros.

.Pensamiento
En lo formal se destaca la bradipsiquia, percibida a través de un relato de curso lento y pausado, con silencios prolongados, voz tenue y monocorde, respuestas con latencia, asociaciones dificultosas, evocaciones penosas, sensación de "mente en blanco" o embotamiento, puede estar en mutismo o semimutismo. Caudal pobre. Frases monosilábicas

En el contenido destacamos la autodevaluación, con ideas de minusvalía ("no valgo nada", "soy un desgraciado", "soy menos"), ideas de culpa excesiva, autoacusación, ideas de ruina (no ve salida, se percibe sin futuro, no será perdonado, no puede esperar sino cosas malas), lo que constituye el dolor moral. Puede llegar a constituir un síndrome delirante.

TIP: Dolor moral: culpa, ruina, minusvalía.

Pueden existir ideas de muerte (desinterés por vivir), de autoeliminación (deseo, plan y búsqueda), de indignidad, de transformación corporal, elaboraciones hipocondríacas (temor y deseo de enfermedad), elaboración paranoica.

Rumiación: ideación lenta centrada en temas tristes que se repiten indefinidamente.

.Psicomotricidad
Inhibición psicomotriz: definida como disminución global de las fuerzas que orientan el campo de la conciencia, dado en: Presentación:

* Presentación: abatimiento, postura encorvada, inmóvil, cabizbajo. Descuido del aspecto personal, desaliño.
* Actitud de colaboración pasiva, disminución de iniciativa verbal.
* FMYG: Pobreza gestual. hipomimia, hipogestualidad. Rostro marmóreo.
* Impresiona distante
* Inercia: disminución de movilidad espontánea, fatiga ("todo es un esfuerzo").
* Clinofilia

TIP: AAAC: Apatía, Astenia, Anhedonia, Clinofilia

* Apatía: disminución de interés
* Astenia: fatiga psíquica y física
* Anhedonia: incapacidad para obtener y experimentar placer (de intención y/o realización)
* Clinofilia

.Síntomas de subtipos de depresión:
Dentro del síndrome depresivo, hay una serie de síntomas que apuntan a un subgrupo de depresiones con características diferenciales:

Síntomas melancólicos: marcada anhedonia Falta de reactividad al entorno. Cualidad distintiva del estado de ánimo. Peoría matutina (polo matinal de la depresión). Despertar precoz. Enlentecimiento o agitación psicomotor. Anorexia significativa o pérdida de peso. Culpabilidad excesiva o inapropiada.

Síntomas atípicos: reactividad del estado de ánimo al entorno Aumento significativo de peso o del apetito. Hipersomnia. Abatimiento (parálisis plúmbea) Patrón de larga duración de sensibilidad al rechazo interpersonal (no limitado al episodio depresivo).

Síntomas catatónicos: inmovilidad motora con o sin catalepsia. Actividad motora excesiva, sin propósito. Negativismo extremo. Peculiaridades del movimiento voluntario (manierismos). Ecolalia o ecopraxia. Depresión puerperal: inicio dentro de las primeras 4 semanas del postparto.

Patrón estacional: relación temporal sostenida entre el inicio de los episodios afectivos y una determinada épica del año. Las remisiones totales también se dan en de-terminada época del año.

Episodio mixto: humor excitado, disfórico, ira, agitación, ideación suicida, mezcla con grandiosidad/hipersexualidad. Importante diferenciar de depresión agitada.

TIP: Atención a la presencia de elementos mixtos. A veces se presentan de forma tal que no es posible diferenciar episodios. En ese caso quizás se debería hacer un Síndrome de alteración del humor y poner los elementos de humor en más y en menos.

.Catalogar síndrome depresivo:

* X tiempo de evolución, con inicio brusco/insidioso
* reactivo (atípico) o no reactivo a estímulos externos
* con ritmo circadiano (polo matinal o no)
* intensidad: leve, moderado, grave.


===== Síndrome de ansiedad-angustia

TIP: Ansiedad psicótica: MIDI.

Subsidiara al síndrome depresivo. Angustia MIDI (masiva, invasiva, desestructurante, incompartible). Expresada sobre todo a nivel de la psicomotricidad.

===== Síndrome delirante

En el cual las ideas melancólicas adquieren persistencia y convicción delirante volviéndose incompartibles, irreductibles a la lógica y con pérdida del juicio de la realidad quedando en primer plano.

.Temática
De frustración, ruina, desgracia, de autodepreciación moral (autoacusación), culpa, somática (transformación o negación corporal), hipocondría, psíquica (dominación, pasividad Influencia). Puede haber un Síndrome de Cotard completo/incompleto: forma mayor de melancolía (CINE: condenación, inmortalidad, negación, enormidad). Negación: de la existencia, del cuerpo, del mundo. Transformación corporal: creencia de estar muerto, de ser un cadáver, de no tener órganos o de que éstos no funcionan (combinación de nihilismo con megalomanía).
Puede haber un síndrome de influencia subsidiario.
El contenido puede ser congruente (culpa, ruina, hipocondría, humillación, influencia, etc.) o no (persecutorio) con el estado de ánimo.
Puede ser subdelirante / ideación sobrevalorada (excesivas, inapropiadas, que admiten cierta crítica) centrado en ideas de pérdida, disminución de autoestima, autorreproche, culpa excesiva, pesimismo.

TIP: Cotard: condenación, inmortalidad, negación, enormidad (completo o incompleto).

.Mecanismo
Intuitivo, autorreferencial (delirante o subdelirante).

TIP: Características del delirio melancólico: ToMoPoPaDiR

Cumple con las características descritas por Seglas para el delirio melancólico: tonalidad afectiva penosa , monotonía (reiterativo, fijo), pobreza (más ricos en emoción que en contenido ideico, escaso desarrollo temático), pasivas (el paciente acepta su desgracia como si se tratara de una fatalidad, paciente indefenso), divergentes (se extienden a los que lo rodean y al ambiente, con riesgo de homicidio piadoso), referidas al pasado o al futuro (ruina).

===== Síndrome conductual

Dado por IAE (si ansiedad es elevada puede ser en contexto de excitación psicomotriz). icon:directions[] Ver encare correspondiente.

Alteración de conductas basales: insomnio (destacar despertar precoz) o hipersomnia (síntoma atípico), anorexia con adelgazamiento o hiperfagia (síntoma atípico). Disminución del cuidado personal (vestimenta e higiene). Disminución de la libido.

Alteración de las conductas complejas / pragmatismos. Disminución de la libido, tendencia al aislamiento social. Abandono o descuido del trabajo.

===== Síndrome de alteración de la conciencia

Desestructuración de conciencia de 1º nivel (ético-temporal) según lo propuesto por Ey. Evidenciado por incapacidad del paciente de adaptar el campo fenomenológico del ser consciente a las exigencias del aquí y ahora.
En general está BOTE (aunque en ocasiones no, por desinterés o por inatención). Polarización por el estado de humor.
Sensación subjetiva de enlentecimiento del tiempo.

==== Síndrome de alteración cognitiva

TIP: no es un síndrome clásico, pero puede adecuarse más a la comprensión actual de la sintomatología depresiva.

Pérdida de capacidad de concentración, olvidos. Déficit atencional.
Incapacidad para tomar decisiones.

TIP: recordar que no debería diagnosticarse demencia solamente con los síntomas que aparecen dentro de un episodio depresivo.

==== Personalidad y nivel

icon:clipboard[] Ver Fragmentos: "Nivel en diferido"

Personalidad: rasgos X que nos evocan X rasgos del grupo Y. Re-evaluaremos en la evolución pues el cuadro actual no permite un diagnóstico preciso. Realizaremos entrevistas con terceros y de ser necesario recurriremos a tests de personalidad. Podemos encontrar: dificultad para superar frustraciones y adaptarse a situaciones dolorosas de la vida yo débil, duelo patológico, dependencia, existir depresivo.

\subsection*{Diagnóstico positivo}
\subsubsection*{Nosografía clásica}
\faLightbulb Los clásicos clasificaban las depresiones de forma distinta al DSM/CIE-10, con lo cual el encare "clásico" se adapta más a la depresión melancólica. Para otros formatos, evaluar hacer diagnóstico por el DSM/CIE.

.Diagnóstico del episodio

icon:clipboard[] Ver Fragmentos: "Psicosis"

icon:clipboard[] Ver Fragmentos: "Psicosis aguda"

Crisis de melancolía: por las características melancólicas del síndrome depresivo ya analizado. Importa destacar desde ya el RIESGO VITAL del diagnóstico establecido, basado en el riesgo de suicidio, ya que en la melancolía la muerte es sentida como una obligación, castigo necesario y solución para poner fin a la situación vivida. El riesgo está implícito en el diagnóstico establecido ya que si bien a veces no manifiestan sus ideas de muerte, la reticencia a manifestarlos es frecuente.

.Forma clínica

* Simple: IPM + poco DM ("con conciencia"). Predomina la IPM con tendencia a la inacción, inercia, astenia. Dolor moral escaso o falta. Tiene cierta conciencia mórbida (pero sin llegar a configurar una depresión "neurótica" o "reactiva").
* Franca: IPM + DM (dolor moral). Inicio progresivo con o sin desencadenante.
* Estuporosa: gran IPM. Paciente espontáneamente inmóvil, en mutismo, no come, no hace gestos, reactividad disminuida (inhibición extrema con vigilia conservada). Fascies marmóreo con expresión de dolor/desespero (facilita el DD con otras etiologías). Riesgo de muerte por deshidratación/inanición. Ver encare de "Estupor".
* Ansiosa: inquietud , búsqueda de muerte: riesgo de IAE. Cuadro dominado por agitación, ansiedad MIDI, psicomotricidad aumentada (caminar, frotarse las manos, zapatear, moverse, gritar, golpearse, correr, frotarse las manos, sollozar, gemir).
* Delirante: Sº depresivo + Sº delirante.
* Estados mixtos: presencia simultánea o rápidamente alternante de síntomas depresivos y síntomas de exaltación del humor. Clínicamente: turbulencia, agitación, perplejidad, irritabilidad / disforia.

.Diagnóstico nosológico

A. PMD unipolar: AF de cuadros afectivos o alcoholismo; AP de cuadros similares con restitución ad-integrum. No existen episodios previos de manía o hipo-manía
B. PMD: similar, pero en la evolución presentó uno o más episodios de exaltación del humor.

===== DSM-IV

.Diagnóstico del episodio

Para DSM IV: Episodio Depresivo Mayor + especificadores.

Especificadores principales:

* Gravedad: L/M/G
* Con síntomas psicóticos: congruentes / incongruentes con el estado de ánimo.
* En remisión parcial / total (2 meses sin síntomas)

TIP: Síntomas catatónicos: CINEMIA

Especificadores de síntomas catatónicos: 2 o + de 5 síntomas dominando el cuadro (CINEMIA):

* Catalepsia / Inmovilidad motora: incluye flexibilidad cérea o estupor. Inercia, actitudes de pasividad y automatismo (latencia en respuestas, obediencia automática, sugestionabilidad) (CI).
* Negativismo: resistencia inmotivada a órdenes, mantenimiento de postura rígida ante intentos de ser movido. Mutismo. Oposicionismo (al interrogatorio, a la alimentación) (N)
* Ecolalia / ecopraxia / estereotipias (actos motores reiterativos / en el lenguaje: verbigeración)(E)
* Manierismos: tonalidad de afectación teatral, pudiendo llegar al pateticismo. Sonrisas inmotivadas / posturas extrañas. (M)
* Impulsiones (I). Actos en cortocircuito, insensibles a estímulos externos, sobre los cuales el paciente no puede dar cuenta. Pueden ser impulsiones verbales.
* Agitación motora: hiperactividad sin propósito aparente, no influida por factores externos (A)

Especificadores de síntomas melancólicos:

A. Anhedonia y/o humor no reactivo
B. 3 o + de 6:
* Cualidad distintiva del estado de ánimo.
* Peoría matutina (polo matinal)
* Depertar precoz (2 horas antes de lo habitual)
* Inhibición o agitación psicomotriz
* Anorexia significativa / pérdida de peso
* Culpa excesiva o inapropiada

Especificadores de síntomas atípicos:

A. Humor reactivo
B. 2 o + de 4
* Aumento de peso o apetito
* Hipersomnia
* Abatimiento (pesadez plúmbea)
* Patrón de larga duración de sensibilidad al rechazo interpersonal (con afectación de pragmatismos)
C. Exclusión: síntomas melancólicos o síntomas catatónicos.

Especificador de patrón estacional:

A. Relación temporal sostenido entre episodio afectivo y épica del año.
B. REmisión total o cambio de polaridad en determinada época del año.
C. En ultimos 2 años, 2 EDM con período estacional y NINGUN EDM fuera del patrón.
D. Lo EDM estacionales tienen que ser más numerosos que los no estacionales.

Importante al plantear el tratamiento.

Otros especificadores:

* Crónico: > 2 años
* Postparto: inicio < 4 semanas luego del parto
* Curso longitudinal: con o sin recuperación interepisódica total.

.Diagnóstico nosológico

TDM - TDM-R - TB I - TB II
Cursando episodio actual X.

Trastorno Depresivo Recurrente: más jóvenes, puede estar precedido por distimia (depresión doble). Mayor porcentaje de antecedentes familiares. Importante realizar este diagnóstico por cambios al plantear tratamiento.

==== Diagnósticos diferenciales

===== Nosografía clásica
. Depresión sintomática de un trastorno médico o consumo de sustancias. Sobre todo en un primer episodio, si los síntomas son atípicos, cuando la evolución no es la esperada, hay mala respuesta al tratamiento o los hallazgos del EF nos hacen sospechar.
.. Neoplasmas: genital, mamas, cabeza de páncreas, pulmón.
.. Fármacos: neurolépticos, reserpina, alfametildopa, betabloqueantes, ACOs.
.. UISP: OH, BZD, anfetaminas / cocaína. Depende de tipo: abstinencia, intoxicación, dependencia, abuso.
.. Endócrino: hipotiroidismo, encefalopatía hepática, efermedad de Addison, diabetes.
.. Neurológico: enfermedad de Parkinson
. Depresión reactiva: previamente llamada "Depresión Neurótica". Cuadro más leve, con humor reactivo, mejora con el  contacto de la entrevista, oscila, permite vibrar con el relato, se establece mejor rapport, pedido de ayuda, sin síntomas psicóticos, sin dolor moral. Está ligada a acontecimientos vitales.
. Otras psicosis agudas:
.. Otras formas clínicas de melancolía: franca/simple/ansiosa/estuporosa/delirante.
.. Manía (en caso de estados mixtos). Si bien comparte el nivel de desestructuración de la conciencia, la clínica es opuesta a la depresiva.
.. PDA (en caso de melancolía delirante). En la melancolía la experiencia delirante es secundaria al estado de ánimo. No hay polimorfismo. El nivel de desestructuración de la conciencia es menor.
.. Confusión Mental: descartado, pues el paciente está BOTE.
. Psicosis crónicas
.. Depresión como debut clínico de Demencia. Tienen el común algunos síntomas cognitivos (atencionales, memoria a corto plazo, bradipsiquia, indiferencia al entorno). Pero nos aleja del diferencial la presencia de AF y AF afectivos, ausencia de AP de trastornos de las funciones instrumentales, simbólicas y psíquicas superiores.
.. Esquizofrenia: lo descartamos por no haber clínicamente un síndrome disociativo-discordante, ni un existir autista, ni alteración de los pragmatismos fuera del episodio. La inhibición psicomotriz y la indiferencia pensamos que son secundarias al cuadro afectivo.
.. Depresión en una Paranoia (cuando hay delirio incongruente con el estado de ánimo): en la depresión el delirio carece de continuidad con la personalidad y carece de la estructura paranoica típica. El orden temporal en la depresión es primero el síntoma afectivo y luego el delirio.

===== DSM/CIE
. Causa orgánica de depresión:
.. Endócrina: hipotiroidismo, Cushing, Addison
.. Metabólica.
.. Tumorales: cabeza pancreática y cerebrales
.. Fármacos y drogas: antihipertensivos, ß bloqueantes, ACO, fenotiazinas, benzodiacepinas
.. Infecciones: mononucleosis, neurolúes, HIV 2. Inicio de deterioro demencial (en pacientes > 65 años)
. Cuadros Delirantes: • Agudo: PDA, confusión. • Crónico: delirios crónicos: AP.
. Estupor:
.. Confusiónal: organicidad, elementos de infección, oscilación rápida estupor-agitación, no existe catalepsia
.. Catatónico de origen esquizofrénico: precedido de SDD, MC es absurdo/impulsivo
.. Histérico
. Ansiosas: diferencial con neurosis.

==== Diagnóstico etiopatogénico y psicopatológico

===== Etiopatogenia

Se postulan 3 factores que interactúan en la patogénesis de la depresión footnote:[]:
* Factores internalizantes: por ejemplo genética y neuroticismo footnote:[Sullivan, P.F., Neale, M.C., Kendler, K.S., 2000. Genetic epidemiology of major
depression: review and meta-analysis. Am J Psychiatry 157, 1552–1562. https://
10.1176/appi.ajp.157.10.1552.].
* Factores externalizantes: por ejemplo consumo de sustancias footnote:[Compton, W.M., Conway, K.P., Stinson, F.S., Grant, B.F., 2006. Changes in the
prevalence of major depression and comorbid substance use disorders in the United
States between 1991-1992 and 2001-2002. Am J Psychiatry 163, 2141–2147.
https://10.1176/ajp.2006.163.12.2141. ].
* Eventos adversos: por ejemplo trauma y pérdida parental footnote:[Green, J.G., McLaughlin, K.A., Berglund, P.A., Gruber, M.J., Sampson, N.A.,
Zaslavsky, A.M., Kessler, R.C., 2010. Childhood adversities and adult psychiatric
disorders in the national comorbidity survey replication I: associations with first
onset of DSM-IV disorders. Arch Gen Psychiatry 67, 113–123. https://10.1001/arch
genpsychiatry.2009.186.].

.Biológico
* Hereditario: importante penetrancia genética.
* Neurotransmisores: alteración en sistemas noradrenérgicos y/o serotoninérgicos en SNC, basado en criterios farmacológicos.
* Neuroendócrinos: alteraciones en niveles de cortisol con alteraciones a nivel del eje HHSR e Hipófiso-tiroideo.
* Edad: disminución de defensas psicológicas + factores biológicos:
* Embarazo/parto, climaterio.
* Mecanismos inflamatorios: factor de necrosis tumoral alfa(TNF-α), interleuquinas. Se postula que la inflamación podría alterar la barrera hematoencefálica con entrada de moléculas inflamatorias y células inmunes del CNS \footnote{Lee, C.H., Giuliani, F., 2019. The Role of Inflammation in Depression and Fatigue. Front
Immunol 10, 1696. https://10.3389/fimmu.2019.01696.}.

.Comprensión psicológica

Puede encontrarse dificultad para superar pérdidas y para adaptarse a situaciones nuevas. Sobre un terreno de vulnerabilidad (personalidad dependiente, poca autonomía) actúan factores psicosociales: pérdidas, dificultades interpersonales.

Hay etapas vitales con mayor riesgo de presentación de sintomatología depresiva: adolescencia, embarazo, puerperio, climaterio, menopausia, envejecimineto, duelo. Se reviven en la esfera inconsciente pérdidas y abandonos tempranos reales o imaginarios.

.Comprensión social

Estresores sociales como factor exterior sobre la vulnerabilidad de base. Pérdida de roles laborales, pérdida de posición social.

===== Psicopatología

.Psicoanálisis
Para la depresión esta teoría se basa en las relaciones ambivalentes de objeto. Este objeto perdido en etapas tempranas del desarrollo psicológico (amado y odiado al mismo tiempo) es posterior-mente introyectado. Las pérdidas de la vida adulta (reales, temidas o fantaseadas) reactivan este proceso volcando la libido y la agresividad hacia el interior, donde se encuentra este objeto introyectado, lo que desencadena una lucha autodestructiva del Yo con un Superyó sádico que se manifiesta como depresión.

.Teoría organodinámica (Ey)

Estructura positiva y negativa:

* Negativa: pérdida de adaptación a las exigencias del presente con falta de proyección al futuro. El sujeto se halla inmerso en el pasado.
* Positiva: contiene la producción subdelirante.

Binswanger y Ey insistieron en la estructura temporal (tiempo subjetivo) de la melancolía (según TOD: 1º nivel de desestructuración de la conciencia o ético-temporal) en la cual el sujeto está anclado en la fatalidad del pasado y para quien el tiempo es una perspectiva de muerte, lo que nos muestra una incapacidad de adaptación a las exigencias del presente. Lo ético está vinculado a la incapacidad de separarse de la culpa y lo temporal por la incapacidad de proyectarse al futuro si no es desde una perspectiva de dolor.

.Teoría cognitivo-comportamental

Basado en el planteo de Beck de la tríada cognitiva de la depresión: visión peyorativa de sí mismo, del futuro y del mundo.

==== Paraclínica

El diagnóstico es clínico. Se solicitará paraclínica de valoración general, para descartar diferenciales, descartar comorbilidad, con vistas al tratamiento y a evaluar aspectos biológicos de la depresión.

Se solicitará desde un punto de vista integral: biológico, psicológico y social.

===== Biológico

Luego de una valoración clínica general del paciente y según hallazgos:

* Consulta con especialistas según hallazgos clínicos.
* Interconsulta con cardiólogo en caso de plantearse tratamiento con AP con potencial alteración del intervalo QT.
* Estudios imagenológicos, según la clínica: TAC, RNM, SPECT, PET.
* Rutinas: hemograma, glicemia, ionograma, función renal, funcional y enzimograma hepático, HIV, VDRL.
* Dosificación de drogas en sangre y orina
* Estudio de hormonas tiroideas: T3, T4 y TSH
* Descartar contraindicaciones relativas de ECT: IAM reciente, arritmias inestables (ECG, cardiólogo), aneurisma de aorta (RxTx), HTEC por proceso expansivo (examen neurológico con fondo de ojo). En pacientes añosos: valoración cognitiva basal.

Marcadores de endogenicidad de la depresión (con fines académicos, no se piden de rutina):
* Dosificación de TSH a la estimulación con TRH: donde esperamos encontrar una respuesta plana.
* Hipnograma: donde esperamos encontrar una disminución de la latencia REM, con aumento de actividad REM, disminución del tiempo to-tal de sueño con despertares frecuentes. De ser negativo no descarta endogenicidad, pero de ser positivo apoya nuestro diagnóstico.

Si es BIPOLAR: valoración según estabilizador del humor que se plantee usar (ver encare correspondiente).

===== Psicológico

Entrevistas diarias para obtención de datos, valorando repercusión de pérdidas actuales y curso de vida. Entrevistas de continentación, no prolongadas.

Luego de superado el cuadro actual: tests de personalidad proyectivos y no proyectivos donde valoraremos fortaleza yoica, mecanismos de defensa, focos de ansiedad y manejo de la agresividad.

De ser necesario: test de nivel, estudio neuropsicológico.

===== Social

Entrevistas con terceros para:

* objetivar adaptabilidad a las pérdidas
* explicar medidas terapéuticas a efectuar, riesgos y beneficios de ECT, consentimiento informado por escrito. Comienzo del proceso de psicoeducación.
* evaluación de red de soporte social
* valorar funcionamiento premórbido e intercrítico así como existencia de corte existencial.

==== Tratamiento

Será realizado por un equipo interdisciplinario, centrado en el paciente, coordinado por el médico psiquiatra, con enfermería, psicólogo, asistente social y especialistas necesarios.

Destinado a:

1. Yugular cuadro actual acortando duración de las crisis, aliviando el sufrimiento.
2. A largo plazo actuando sobre la enfermedad de fondo, tratando la comorbilidad, previniendo complicaciones y realizando profilaxis de futuras recaídas, reintegrando el paciente a su medio en el mejor estado.

Internación en sala psiquiátrica de hospital general / hospital psiquiátrico (formas más graves), con acompañante a permanencia, fundamentado en:

* Se trata de una urgencia psiquiátrica que coloca al paciente en un riesgo de muerte por auto-eliminación.
* Presencia de síntomas psicóticos
* Repercusión somática: anorexia, adelgazamiento
* Necesidad de reversión rápida del cuadro.

Con supervisión de enfermería las 24 horas, control de hidratación, alimentación y toma de medicación, vigilando eventual intento de fuga o autoeliminación. Acompañante a permanencia. Visitas reguladas según la mejoría clínica de personas significativas, continentadoras, no conflictivas.

===== Biológico

.Antidepresivos

La elección estará determinada entre otras cosas por AP de respuesta a tratamientos previos. En caso de ausencia de antecedentes seleccionaremos antidepresivos según situación clínica \footnote{Cipriani, A., Furukawa, T. A., Salanti, G., Chaimani, A., Atkinson, L. Z., Ogawa, Y., ... \& Geddes, J. R. (2018). Comparative efficacy and acceptability of 21 antidepressant drugs for the acute treatment of adults with major depressive disorder: a systematic review and network meta-analysis. The Lancet, 391(10128), 1357-1366.}:

* Paciente sin tratamientos previos: preferimos el uso de un ISRS, tal como Sertralina 50 mg 1 comp/día, por la menor incidencia de efectos secundarios. En caso de coexistencia de ansiedad, preferimos un ISRS sedativo (Fluvoxamina, Paroxetina). En caso de tratarse de un paciente añoso: Escitalopram. Estos antidepresivos actúan mediante el bloqueo de la recaptación se serotonina produciendo a mediano plazo una regulación a la baja (desensibilización) de los autorreceptores 5HT1a (presinápticos) y 5HT1d (postinápticos) de la neurona serotoninérgica. Estaremos atentos a la aparición de efectos secundarios, sobre todo a nivel digstivo en etapas iniciales, la posibilidad de viraje en plazos medianos y la disfunción sexual (disminución de la libido, retardo en el orgasmo) a mediano/largo plazo.
* Paciente con tratamiento previo con ISRS sin respuesta: planteamos el uso de Venlafaxina, antidepresivo con doble mecanismo (acción sobre sistema noradrenérgico y serotroninérgico). Comenzaremos con 75 mg/día, aumentando a 150 mg/día. Según respuesta puede llevarse hasta 300 mg/día.
* Paciente bipolar: planteamos de primera línea el uso de estabilizadores del humor (Lamotrigina, Litio) con o sin combinación con antipsicóticos atípicos (Aripiprazol). En caso de que haya que usar un antidepresivo, preferimos el uso de Bupropion 150 mg LP, 1 comp/día, ya que hay menos chances de que se produzca un viraje en el humor.

Estaremos atentos a la evolución del tratamiento ya que secuencialmente mejoran: 1° la anorexia y el insomnio, luego la inhibición psicomotriz y recién al final el dolor moral. Previo a este período, el paciente se encuentra desinhibido con potencial suicida por la presencia del dolor moral. En caso de que se trate de un paciente bipolar: controlaremos la posibilidad de viraje del humor.

.ECT

Puede plantearse ante el fracaso del tratamiento farmacológico o (en algunos casos clínicos) puede plantearse de entrada.

De entrada:

Por tratarse de icon:paperclip[] estamos ante una indicación formal de ECT (depresión mayor con síntomas melancólicos, catatónicos o psicóticos; melancolía ansiosa) ya que:

* El tratamiento farmacológico tiene latencia de al menos 15 días
* Ansioso: pasaje al acto con máximo riesgo vital
* Las ideas de muerte pueden no manifestarse por reticencia
* Para provocar alivio sintomático al intenso sufrimiento del paciente
* La posibilidad de AE durante el tratamiento con antidepresivos una vez mejorada la inhibición con persistencia de dolor moral y las ideas de AE

Se realizará con el paciente con al menos 6 horas de ayuno, con el pelo adecuadamente aseado, suspendiendo en esa mañana los fármacos que puedan aumentar el umbral convulsivo (benzodiacepinas, antiepilépticos) o que aumenten las probabilidades de confusión (litio).

El tratamiento conjunto desde el inicio con AD y ECT posee mejor índice de mejoría que c/u por separado.

La ECT será realizada por anestesista y psiquiatra, una sesión cada día por medio, con anestesia general (por ejemplo con Propofol), oxigenoterapia, monitorización ECG y EEG; con el paciente curarizado (por ejemplo con succinilcolina). Regularemos la cantidad de sesiones según respuesta pero pensamos que serán necesarias entre 8-12 sesiones para lograr el efecto deseado. Vigilaremos al paciente después de cada sesión sabiendo que pueden existir cefaleas y trastornos mnésicos de breve duración.

.Estabilizadores del humor
Ver encare de Manía (F31) para el uso del litio.

Lamotrigina: se trata de un fármaco con efecto de estabilización del humor desde abajo con capacidad de prevención de recurrencias depresivas (no de recurrencias maníacas). Debiendo aumentarse de forma gradual por el riesgo de la presentación de rash (8% son benignos, 1:1000 pueden ser graves: Síndrome de Steven-Johnson, necrólisis epidérmica tóxica, reacción a drogas con eosinofilia y síntomas sistémicos). 
Comenzaremos con 25 mg/día aumentando en 15 días a 50 mg, luego en 15 días a 100 mg para llegar finalmente a una dosis de 200 mg. En caso de usarse conjuntamente con divalproato, debe ajustarse la dosis de la lamotrigina a la mitad. En caso de usarla conjuntamente con carbamazepina, debe usarse el doble de dosis que lo habitual.

.Otras formas clínicas

* Delirante: agregar antipsicóticos, preferentemente atípicos: Aripirazol icon:arrow-right[] Olanzapina icon:arrow-right[] Risperidona icon:arrow-right[] Haloperidol.
* Agitada-ansiosa: preferentemente ISRS sedativo (Fluvoxamina). ECT si la agitación es intensa.
* Bipolares: ver encare correspondiente.

.Síntomas accesorios

Para combatir el insomnio usaremos Flunitrazepam 2 mg v/o en la noche para controlar el insomnio (las horas de la madrugada son las de mayor riesgo suicida). En caso de persistir insomnio, agregaremos Midazolam 1 amp i/m si no duerme.

Para la ansiedad, usaremos benzodiacepinas (Diazepam, Lorazepam, Clonazepam, Alprazolam) que proven un rápido alivio de la ansiedad hasta que el resto de los fármacos pasen su período de latencia.

.Refractariedad

Ante la falta de respuesta a la farmacoterapia luego de 4-8 semanas se debe:

. revisar el diagnóstico
. verificar que cumpla con el tratamiento
. descartar problemas médicos concomitantes
. descartar UISP
. descartar comorbilidad con otros trastornos psiquiátricos
. re-evaluar aspectos psicosociales

Desde el punto de vista farmacológico, considerar agregado de: antidepresivo con distinto mecanismo icon:arrow-right[] Aripiprazol icon:arrow-right[] Litio icon:arrow-right[] T4.

.Alta

Criterios de alta:

. Rectificación de las ideas de muerte
. Desaparición del delirio
. Normalización de las CB
. Mejoría global de la depresión
. Estabilización de los niveles plasmáticos de fármacos

Otorgaremos el alta hospitalaria una vez superado el cuadro actual en el cual es fundamental la rectificación de la conducta suicida. Mantendremos el antidepresivo a dosis plenas por largo plazo. Controlaremos en policlínica quincenalmente en un principio y luego se espaciarán hasta ser mensuales.

===== Psicológico

Realizaremos entrevistas diarias orientadas a:

. Continentar al paciente sin provocar fatiga (para el paciente la entrevista representa un esfuerzo psíquico)
. Generar y consolidar el vínculo terapéutico con el paciente y la familia
. Psicoeducación: generar conciencia de la importancia de la adhesión al tratamiento como determinante del pronóstico a mediano y largo plazo. Se educará acerca de signos y síntomas de recaída.
. Evaluar evolución del tratamiento

Realizaremos apoyo psicológico para reelaboración de pérdidas.

===== Social

Entrevistas con la familia para integrarla al proceso terapéutico. Psicoeducación para familiares. Información sobre el uso de recursos pertinentes para la enfermedad. Biblioterapia.

==== Evolución y pronóstico

Sabemos que la PMD mono/bipolar es una enfermedad crónica que evoluciona por accesos que pueden reiterarse. Pautas previas de recaídas predicen índice futuro. Si bien con el tratamiento profiláctico esperamos los períodos intercríticos y disminuir gravedad de los accesos. Pronóstico alejado depende de adhesión al tratamiento. A mayor edad más episodios, ML, IAE.

PVI PPI: bueno, sujeto a complicaciones: IAE.

PVA: lo que tenga, sujeto al psiquiátrico. Mayor prevalencia de suicidios (en especial depresiones con síntomas psicóticos). La depresión no tratada disminuye la expectativa y la calidad de vida.

PPA:

* orgánico: AF/AP IAE, edad
* psiquiátrico: situaciones adversas, falta de elaboración de pérdidas, sentimiento de abandono.
* social: aislamiento, pérdida de roles, relaciones interpersonales

Hay una tendencia a la pérdida de la reactividad en los episodios con progresiva autonomía de factores desencadenantes.

Se postulan formas evolutivas a la cronicidad (nosografía clásica):
. Melancolía crónica simple (acceso con remisión parcial)
. Delirio crónico melancólico (persiste al desaparecer la depresión), a forma hipocondríaca o a forma de síndrome de Cottard crónico.

===== Factores de mal pronóstico
. Antecedentes de maltrato o abuso en la infancia: factor de riesgo para severidad, precocidad, resistencia y cronificación de cuadros depresivos footnote:[Nelson, Janna, et al. "Childhood maltreatment and characteristics of adult depression: meta-analysis." The British Journal of Psychiatry 210.2 (2017): 96-104.].
. Síndrome metabólico footnote:[Pan, An, et al. "Bidirectional association between depression and metabolic syndrome: a systematic review and meta-analysis of epidemiological studies." Diabetes care 35.5 (2012): 1171-1180.].

==== En suma

Hemos visto un paciente de sexo X, de X años de edad, con un MSEC X, con AF de X, con APM de X, con AP de X, que consulta por X síntomas, en quien diagnosticamos un episodio X en un trastorno X, planteándose DD con X, que hemos estudiado con X, planteando un tratamiento X, cuyo pronóstico es X.
