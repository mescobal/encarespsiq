\chapter*{Trastornos conversivos y disociativos / Neurosis histérica}
\section*{Encare}
\subsection*{Agrupación sindromática}
\subsubsection*{Síndrome conversivo}
Entendiendo por conversión la pérdida o alteración del funcionamiento físico que sugiere trastorno somático/físico pero sin base orgánica demostrable, que se interpreta como vinculado a un conflicto o necesidad psicológica/intrapsíquico dado por:
\paragraph{A nivel motor}
* Crisis de excitación psicomotriz que motivan el ingreso, de X tiempo de evolución, paroxísticas, con movimiento de los 4 miembros, con caída al suelo, rotura de objetos, emisión de gritos, en presencia de público, con o sin pérdida de conocimiento, sin incontinencia esfinteriana ni sopor postcrítico, sin mordedura de lengua, sin cianosis. Relata crisis mostrando indiferencia hacia ésta, frecuencia diaria, duración, desencadenada por conflicto emocional (ejemplo). Por lo que la calificamos de crisis conversiva, expresión atenuada de la gran crisis de Charcot.
* Manifestaciones deficitarias permanentes de X duración:
** Astasia-abasia: incapacidad de andar y mantenerse de pie, quedando la posibilidad de realización de movimientos activos que no sean de deambulación.
** Parálisis y contracción localizadas: sin sistematización anatómica. Incluye afonía por parálisis de músculos fonadores.
** Movimientos anormales.
\paragraph{A nivel sensitivo}
* Anestesia: sin sistematización anatómica, en bota, en guante, hemicuerpo.
* Puntos hiperestésicos
* Algias, con impotencia funcional desproporcionada
\paragraph{A nivel sensorial}
* Visuales: obnubilación, diplopía, ceguera (con las características de: reducción concéntrica del campo visual y diplopía monocular).
* Auditivos
\paragraph{A nivel neurovegetativo}
* Espasmos de músculo liso y esfínteres, faringe, vómitos, respiratorio, vesical, vaginal

\faNotesMedical Crisis de inhibición psicomotriz ("ataques catalépticos", "sueño histérico"), simula un coma. Se recupera por técnicas sugestivas o dolorosas. Sin la tríada característica del sueño normal: miosis, estrabismo divergente por el predominio del tono del gran oblicuo, contracción activa del orbicular de los párpados).
\subsubsection*{Síndrome disociativo histérico}
Definido como la alteración súbita y transitoria de las funciones integradoras de la conciencia, memoria, identidad, percepción, dado por:
\paragraph{Trastornos mnésicos}
Amnesia disociativa psicógena: evocación biográfica imprecisa, amnesia lacunar y selectiva, con dificultad para evocar determinados recuerdos importantes de valor simbólico (Ej.: experiencias dolorosas o vergonzosas), con desencadenante psicológico. Puede ser global y total a partir del desencadenante (generalizada y continua). La evocación de violación/seducción puede corresponder a una ilusión de la memoria, alteración frecuente en estos pacientes \footnote{Políticamente incorrecto, actualmente}. Puede haber identificación imaginaria con otras personas (por ejemplo con otras pacientes de la sala: siente sus síntomas). No hay evidencias de un trastorno mental orgánico.
\paragraph{Fuga disociativa}
Con desencadenante emocional. Amnesia disociativa + desplazamiento intencional (lejos del hogar o lugar de trabajo), en la cual mantiene cuidados básicos de sí mismo, lleva a cabo una interacción simple con extraños y presenta amnesia del episodio, por lo que lo calificamos de fuga psicógena. Puede presentar confusión sobre su identidad.
\paragraph{Estupor disociativo}
Ver encare de estupor. Clínicamente reconocido por una disminución profunda o ausencia de movilidad voluntaria, disminución de reactividad a estímulos exteriores, no está dormida ni inconsciente.
\paragraph{Estados crepusculares}
Debilitamiento del nivel de consciencia, que puede llegar en profundidad desde la obnubilación al estupor. Comporta una experiencia semiconsciente de despersonalización y extrañeza.
\paragraph{Estados segundos}
Caracterizados por un estado de consciencia debilitado dentro del cual ocurre una producción de gran riqueza visual compleja (fenómenos seudoperceptivos). Son de alto valor simbólico afectivo, en las que podemos inferir la expresión de conflictos internos, que recuerda al ensueño, admiten crítica, estereotipadas, parahípnicos (hipnagógicos o hipnapómpicos). Dados por elementos sensorailes excitatorios de diferente complejidad (acufenos, fosfenos). Pueden plantear DD con crisis uncinadas si hay alteraciones olfativas.
Otros nombres: alucinosis histérica, síndrome seudoperceptivo.
\paragraph{Otros}
Sonambulismo, personalidad múltiple.
\subsubsection*{Síndrome de ansiedad-angustia}
Vivencial: estado de alerta y tensión, inquietud permanente sin objeto, desmesurado de las preocupaciones.

Somático: tensión motora, hiperfuncionamiento autónomo, vigilancia y control.
\subsubsection*{Síndrome depresivo}
Humor y afectividad: irritabilidad (disforia histeroide), anhedonia. A/v depresión atípica (irritabilidad, hipersomnia, hiperorexia). Inhibición psicomotriz (presentación, pensamiento: ideas tristes, apatía, astenia, conductas basales y pragmatismos). Dolor moral: ruina, culpa, minusvalía, ideas de muerte o de AE.
\subsubsection]{Síndrome conductual}
IAE.
\subsubsection*{Generalidades del cuadro}
El cuadro tienen las siguientes características:

* Factor desencadenante: relación temporal entre un estímulo estresante y el inicio del síndrome, reactivo.
* Beneficio secundario (3° para el psicoanálisis): le permite evitar un perjuicio u obtener un beneficio del entorno.
* Intencionalidad inconsciente: el síntoma no es voluntario, simboliza un deseo inconsciente.
* Inicio súbito, posterior a una crisis de ansiedad que suele cesar con la instalación del síntoma
* Representa el concepto que la paciente tiene sobre el trastorno somático
* Bella indiferencia hacia el síntoma
* Contexto biográfico y actual que le da sentido al síntoma
* Recurrencia en el tiempo

\subsection*{Personalidad y nivel}
Nivel: pueden haber síntomas conversivos en contexto de nivel marginal / BNI (DD con la puerilidad que puede verse en algunas histerias). Ante la duda: test de nivel.

Personalidad:

Historia de conflictiva infantil.

Rasgos neuróticos globales (yo débil)

* Mal manejo de la agresividad
* Trastornos de la esfera sexual
* Dependiente/inmaduro

Rasgos histéricos/personalidad histérica
\begin{itemize}
	\item Egocentrismo
	\item Histrionismo: hiperexpresividad, dramatismo, fantasía mitomanía
	\item Labilidad emocional
	\item Sugestionabilidad plasticidad: autosugestión, sugestión externa
	\item Dependencia
	\item Erotización de los vínculos
	\item Trastornos sexuales
	\item Superficialidad de vínculos
	\item Poco interés por lo intelectual
	\item Baja tolerancia a las frustraciones
	\item Manipulación del entorno
	\item Acting-out con escasa o nula previsión de sus actos
\end{itemize}

En la entrevista destacar:
\begin{itemize}
	\item bella indiferencia
	\item impresionsimo, teatralidad, sugestionabilidad
	\item intento de manipulación o manejo de la entrevista
	\item puerilidad
	\item erotización o intento de seducción durante la entrevista
\end{itemize}

Siguiendo nosografía propuesta por DSM, puede estar asociados a rasgos histriónicos: patrón de excesiva emotividad y búsqueda de atención con 5 o más de:
\begin{itemize}
	\item incomodidad si no es el centro
	\item erotización de los vínculos (comportamiento sexualmente seductor o provocador)
	\item expresión emocional superficial y cambiante
	\item uso del aspecto físico para llamar la atención
	\item forma de hablar excesivamente subjetiva y carente de matices
	\item teatralidad, dramatización
	\item sugestionabilidad
	\item considera sus relaciones más íntimas de lo que son
\end{itemize}

TIP: El TP Histriónico se asocia a: Trastorno de Somatización, Trastorno Conversivo/Disociativo, otros trastornos del grupo B.

\subsection*{Diagnóstico positivo}
\subsubsection*{Nosografía Clásica}
\faPaste Fragmentos: Neurosis.
\paragraph{Neurosis histérica}
Por síndrome disociativo histérico + síndrome conversivo (críticos o permanentes), en un paciente con rasgos de personalidad histérica, con AP de cuadros similares. Leve/moderada/grave: según grado de repercusión sociofamiliar-laboral, intensidad y duración de los síntomas.

\paragraph{Descompensada}
Por:
\begin{itemize}
	\item Síndrome depresivo
	\item Ansiedad angustia
	\item Exacerbación de síntomas con falla de mecanismos de defensa
	\item Crisis conversiva o estado conversivo
	\item IAE Causa de descompensación: estrés psicosocial situación vital que es incapaz de asumir (matrimonio, hijo, episodio conflictivo intrafamiliar, frustraciones afectivas, situación de abandono o rechazo).
\end{itemize}
\subsubsection*{CIE-10 - DSM IV}
\paragraph{CIE-10}
Las posibilidades diagnósticas (CIE) son:

F44 Trastornos disociativos (de conversión)
\begin{itemize}
	\item F44.0 Amnesia disociativa
	\item F44.1 Fuga disociativa
	\item F44.2 Estupor disociativo
	\item F44.3 Trastornos de trance y de posesión
	\item F44.4 Trastornos disociativos de la motilidad
	\item F44.5 Convulsiones disociativas
	\item F44.6 Anestesias y pérdidas sensoriales disociativas
	\item F44.7 Trastornos disociativos (de conversión) mixtos
	\item F44.8 Otros trastornos disociativos (de conversión)
	\item F44.80 Síndrome de Ganser
	\item F44.81 Trastorno de personalidad múltiple
	\item F44.82 Trastornos disociativos (de conversión) transitorios de la infancia o adolescencia
	\item F44.88 Otros trastornos disociativos (de conversión)
	\item F44.9 Trastorno disociativo (de conversión) sin especificación Recordar que para el DSM pueden o no coexistir un Trastorno de Conversión (eje I), un Trastorno Disociativo (eje I) y un Trastorno Histriónico de la Personalidad (eje II).
\end{itemize}

\faLightbulb: Recordar que en el DSM el eje I y el II son independientes (hasta cierto punto), por lo cual se puede diagnosticar (en teoría) un trastorno conversivo con o sin un trastorno de la personalidad comórbido. Desde el punto de vista de la nosografía clásica no se puede diagnosticar una neurosis histérica y un trastorno de la personalidad histriónico.

\paragraph{DSM}
En general es un diagnóstico con un criterio positivo, varios criterios de exclusión y el requerimiento de qyue haya un malestar "clínicamente significativo".


* Trastorno de conversión
** Inclusión: Síntoma o déficit motor voluntario / sensorial que sugiere enfermedad neurológica + factor psicológico asociado (con desencadenante o conflicto previo)
** Exclusión: Trastorno Facticio, Simulación, enfermedad médica o sustancias
** Especificadores: con síntoma o déficit motor / con crisis y convulsiones / con síntoma o déficit sensorial / de presentación mixta
* Amnesia disociativa
** Inclusión: uno o más episodios con incapacidad para recordar información personal importante (generalmente traumático)
** Exclusión: (no aparece exclusivamente en...) Trastorno de Identidad Disociativo, Fuga Disociativa, TEPT u otros.
* Fuga disociativa:
** Inclusión: amnesia + desplazamiento geográfico + confusión sobre la identidad personal o asunción de una nueva identidad (parcial o completa)
** Exclusión: (no aparece exclusivamente en...) Trastorno de Identidad Disociativo, enfermedad médica, sustancias.
* Trastorno de Identidad Disociativo:
** Inclusión: presencia de 2 o más identidades o estados de personalidad + al menos 2 de estas identidades controlan de forma recurrente el comportamiento del individuo + amnesia disociativa.
** Exclusión: efecto fisiológico directo de una sustancia, enfermedad médica.

\faTrafficLight: el encare de un trastorno somatomorfo puede tener algunos puntos en común con el encare de una neurosis histérica, pero con la nosografía moderna quedan en categorías distintas.
\subsection*{Diagnósticos diferenciales}
. Epilepsia generalizada TC (DD con crisis de EPM conversiva): por las características reseñadas que nos permiten catalogar las crisis como conversivas no pensamos que se trate de una crisis epiléptica. Dada la frecuencia de coexistencia de ambas patologías realizaremos un minucioso estudio paraclínico. Nos aleja de la epilepsia el hecho de que en las crisis no hay pérdida de consciencia, ni mordedura de lengua, ni incontinencia de orina, ni traumatismo al caer, ni sopor postcrítico). Epilepsia de lóbulo temporal (DD con amnesia disociativa, fugas).
. Otros trastornos orgánicos que se manifiestan por plejias, trastornos sensitivos, visuales, como esclerosis múltiple (20-45 años, visión borrosa, diplopía, escotomas centrales, alteraciones sensitivas, debilidad muscular) que evoluciona por empujes. Otros: TEC, tumores, intoxicación, infecciones..
. Trastorno de la personalidad histriónico: rasgos no son inflexibles ni maladaptativos, no existe pauta de egosintonía (pide ayuda), se da en contexto intrapsíquico (no interpersonal), por lo que lo descartamos. NOTA: no es diferencial para la nosografía DSM. La nosografía clásica exige personalidad + síntomas, pero la personalidad es de tipo neurótico (egodistónico, autoplástico) y el TdelaP no (egosintónico, aloplástico).
. Neurosis de angustia / otras neurosis: el cuadro está centrado por la sintomatología disociativa-conversiva y si existe ansiedad-angustia esta aparece descompensando la neurosis estructurada.
. Trastorno afectivo primario (melancolía ansiosa): no existe dolor moral, la depresión es subsidiaria del trastorno neurótico.
. Síndrome amnésico orgánico: es más grave para los hechos recientes que para los remotos, no existe selectividad, no existe relación con desencadenantes emocionales.
. Esquizofrenia (alejado) cuando se presenta con teatralidad exagerada, sobrecargada (pero en la esquizofrenia es por manierismo o catatonía).
. Trastorno psicótico breve: por alteración de conciencia + alteraciones perceptivas.
. Trastorno facticio / simulación (expersa voluntad de engaño)
. Otros: intoxicación alcohólica, enfermedad psicosomática.

En los diferenciales por CIE / DSM: lo orgánico, sustancias.

\subsection*{Diagnóstico etiopatogénico y psicopatológico}

====== Comprensión psicológica
Ey define la histeria como "una neurosis caracterizada por la hiperexpresividad somática de las ideas, imágenes y afectos inconscientes". Para Ey se necesitan 2 elementos para definir la histeria: la fuerza inconsciente de la realización plástica de las imágenes sobre el plano corporal (síntoma) y la estructura inconsciente e imaginaria del personaje histérico (personalidad).

Para el psicoanálisis, comporta una regresión y fijación a la fase edípica del desarrollo psicosexual. La reactivación del conflicto sobrepasa el mecanismo de represión que no basta para contener la angustia en el inconsciente, por lo que se recurre al mecanismo de conversión, con el cual el síntoma somático impide el acceso a la conciencia del conflicto rechazado, siendo el síntoma una expresión simbólica de éste. El conflicto que no puede hacerse consciente se disocia, refugiándose en una nueva realidad y aparece representado en una realidad paralela con lo cual se mitiga la ansiedad. La conversión sería la expresión somática de un conflicto inconsciente. El síntoma somático constituye un compromiso que impide el acceso a la conciencia del conflicto rechazado, al tiempo que implica una realización sustitutiva y disfrazada del deseo prohibido.

Importa destacar que la sintomatología es involuntaria pero cargada de intencionalidad inconsciente. Del diagnóstico psicopatológico jerarquizamos los siguientes aspectos:

* Presenta como beneficio primario la disminución de la angustia o la anulación de ésta manteniéndola fuera del campo de la conciencia.
* Presenta como beneficio secundario el manejo del entorno con lo que se gratifican las necesidades de dependencia de la paciente, condiciona la evolución de la dolencia, ganancia de tipo narcisista. La histeria se modela en función de la respuesta, adaptándose al deseo del otro). Se acompaña de "belle indiference" que es la indiferencia con respecto al síntoma. Este mecanismo implica el uso de mecanismos de defensa como la represión y la conversión.
* Identificación con antecesor u otro enfermo

Con respecto al desarrollo de la personalidad, el Yo histérico no ha logrado una organización estable conforme a una identificación de su propia persona. El papel que toma como rol oculta a su persona. Hay una gran psicoplasticidad (histrionismo) con erotización de la conducta y los vínculos, produciéndose una "falsificación" de la existencia. Se sustituye el principio de realidad por el deseo y la fantasía (pensamiento imaginario). El cuerpo pasa a ser escenario de los conflictos (disposición conversiva).

====== Comprensión biológica

Se postula la existencia de alteraciones en comunicacion interhemisferica, hipometabolismo del hemisferio dominante, hipermetabolismo del no dominante. Alteración de comunicación con la sustancia reticular.

\subsection*{Paraclínica}
Para: apoyar diagnóstico, descartar diferenciales, en vistas al tratamiento, de valoración general. Se realizará desde un triple punto de vista: biológico, psicológico y social.
\subsubsection*{Biológico}
* Consulta con internista con EF completo, con énfasis en lo neurológico (campo visual, pares craneanos, sensibilidad, fuerzas, reflejos), incluyendo Fondo de Ojo. Despistaremos entidades de diagnóstico clínico como Esclerosis Múltiple. Buscaremos signos focales, elementos de síndrome frontal, polineuropatía sensitiva y motora, flapping, rueda dentada, hiperreflexia, hiptertensión endocraneana, síndrome cerebelos, etc. También buscaremos estigmas de UISP ode OH.
* EEG: para despistar foco epiléptico (con registro prolongado, con deprivación de sueño y estimulación con hiperpnea y fotoestimulación).
* Rx cráneo: valorando repercusión de múltiples caídas.
* Valoración general: hemograma, glicemia, azoemia, creatininemia, orina completa, ionograma., funcional y enzimograma hepático.
* Infeccioso: HIV; VDRL, serología para hepatitis.
* Test de beta-HCG descartando embarazo (adolescente con reagudización de sintomatología).
* Función tiroidea.
* Tóxicos en orina.

Interconsultas con especialistas según hallazgos.
\subsubsection*{Psicológico}
Entrevistas que tienen una finalidad diagnóstica y terapéutica.
Superada la agudeza del cuadro evaluaremos características propias del paciente, sus capacidades y motivaciones para la psicoterapia. De ser necesario realizaremos tests:

* Tests de Personalidad: proyectivos (TAT, Rorschach), no proyectivos (MMPI).
* Tests de Nivel (Wechsler).

En función de los hallazgos seleccionaremos el tipo de psicoterapia.

Tendremos la precaución de generar un vínculo dentro de un encuadre adecuado, con adecuada puesta de límites y evitación de la generación de beneficios secundarios, favoreciendo la verbalización como forma de expresión, análisis y resolución de conflictos.
\subsubsection*{Social}
Entrevistas con terceros, valoración de la magnitud de los beneficios secundarios. Valoración de medio familiar, vínculos. Evaluación de red de soporte social, inventario de eventos vitales (en particular eventos traumáticos) y respuesta a los mismos. HC anteriores, tratamientos, respuestas.

Indagar VD e historia de AS.
\subsection*{Tratamiento}
Sintomático y etiológico. Objetivo: compensar el cuadro actual, tratamiento enfermedad de fondo con profilaxis de recidivas y complicaciones. El tratamiento será dinámico, adaptándose a la evolución clínica.
El tratamiento salvo excepciones se realizará de forma ambulatoria: evitar la internación dentro de lo posible. Esta será indicada cuando:

* Hay un IAE o alteraciones comportamentales que impliquen riesgo para sí o terceros.
* Se necesita aislamiento del foco conflictivo para combatir el beneficio secundario que refuerza la sintomatología (complicidad familiar inconsciente)
* Si el medio es poco continente y el cuadro es grave
* Para el tratamiento y control del síndrome depresivo y evitar sus complicaciones

De internarse será breve por alta sugestionabilidad que hace que se alimente de las patologías de otros y la elevada tendencia a realizar un manejo del medio. Solo deberán autorizarse acompañantes más aptos, poco involucrado con los beneficios secundarios

====== Del cuadro actual

.Biológico

Tratamiento sintomático

* Crisis: aislamiento con protección hasta que remita. Eventualmente puede usarse una benzodiacepina IM (Lorazepam 2mg i/m). Se debe psicoeducar a la familia con respecto a las crisis para evitar que sean consideradas producto de una simulación.
* Ansiedad-angustia: Diazepam para disminuir el monto de ansiedad (5-5-10) a regular según evolución (opciones: Clonazepam, Bromazepam, Alprazolam [segunda línea por potencial generación de dependencia]).
* Depresión - ansiedad: ISRS -> sedativos (Paroxetina, Fluvoxamina) o no-sedativos (Sertralina, Citalopram, Fluoxetina) a dosis estándar.
* En caso de Disforia Histeroide, posibilidad de uso de IMAO: Moclobemida 300-600 mg/día (comp 150 mg) 
* Síntomas seudoperceptivos: hay autores que plantean uso de NL a bajas dosis para síntomas seudoperceptivos: Haloperidol 1 mg/día. Otros autores que afirman que los NL están contraindicados, ya que la aparición de efectos secundarios puede agravar el cuadro preexistente. También se postula alto grado de imprevisibilidad en la respuesta al psicofármaco, incluso con reacciones paradojales

.Psicológico

Durante las crisis: habiendo tomado precauciones dejaremos sola a la paciente, dado que las crisis tienen un sentido vincular al cual no responderemos. Luego de las crisis haremos sesiones de apoyo breve. Instruiremos a la familia sobre este punto, evitando denigrar a la paciente, evitando que se la considere una simuladora. 

Durante la internación: psicoterapia de apoyo buscando crear un vínculo terapéutico que asegure el apego al tratamiento a largo plazo y craendo un espacio de abordaje maduro de sus conflictos. Psicoeducación.

.Alta

Se efectuará lo antes posible. Controles quincenales que iremos espaciando. Evitar polifarmacia. Re-evaluar la necesidad de medicación en forma periódica.

====== De la enfermedad de fondo

Una vez superado el cuadro actual será psicológico y social fundamentalmente.

.Psicológico

Psicoterapia de corte psicoanalítico, según: edad, nivel intelectual, duración de la enfermedad, búsqueda de ayuda con deseo de mejoría. Los objetivos serán mejorar los síntomas, con adecuación al medio y lograr cambios perdurables en la estructura de su personalidad con uso de mecanismos de defensa más adaptativos. Será fundamental una comprensión de la sintomatología por parte de la paciente. El psicodrama como terapia grupal puede ser beneficioso.

.Social

Si se encuentra inactiva: puede beneficiarse de laborterapia. Desalentaremos las prácticas religiosas que favorezcan la disociación. Terapia familiar para atacar el beneficio secundario. Psicoeducación.
\subsection*{Evolución y pronóstico}

Es un trastorno crónico que evoluciona con remisiones y reapariciones polimorfas, variadas, con recrudecimiento en relación a conflictos psicosociales. La cronicidad de los síntomas se puede producir si se mantienen constantes los beneficios secundarios cristalización fija. Tiende a disminuir en la madurez. Pueden instalarse trastornos en comorbilidad con las complicaciones consiguientes.
Con psicoterapia pueden prolongarse los tiempos libres de síntomas. 

PVI y PPI: bueno con el tratamiento instituido (excepto por la posibilidad de lesiones por autoagresión o traumatismo durante las crisis).

PVA y PPA: depende de:

* Personalidad premórbida
* Situación ambiental
* Adhesión a psicoterapia
