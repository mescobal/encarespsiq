\documentclass[encares.tex]{subfiles}
\begin{document}
== F84 Trastornos generalizados del desarrollo

- F84.0 Autismo infantil (Psicosis Infantil, Síndrome de Kanner)
- F84.1 Autismo atípico
- F84.2 Síndrome de Rett
- F84.3 Otro trastorno desintegrativo de la infancia
- F84.4 Trastorno hipercinético asociado a RM y movimientos estereotipados
- F84.5 Síndrome de Asperger (Trastorno Esquizoide de la Infancia)

=== Notas Clínicas

Característica fundamental de lo TGD: alteraciones cualitativas de la interacción social y formas de comunicación con un conjunto restringido de intereses y actividades.

Otros nombres: trastorno autístico, autismo infantil, psicosis infantil, síndrome de Kanner.

Síndrome de Asperger: validez diagnóstica dudosa. Autismo SIN déficit o retraso en el lenguaje o en el desarrollo cognoscitivo. Generalmente asociado a torpeza motora. Pueden tener episodios psicóticos ocasionales. Sinónimo: trastorno esquizoide de la infancia.

Tendencia a usar Trastornos del Espectro Autista (TEA) como término paraguas. Prevalencia hasta 1% de la población.

=== Encare

==== Diagnóstico positivo

===== DSM-5

.Trastorno del espectro autista

Diagnóstico bidimensional. Los estudios muestran que tiene más validez que el tridimensional del DSM-IV \footnote{Mandy, W. P., Charman, T., \& Skuse, D. H. (2012). Testing the construct validity of proposed criteria for DSM-5 autism spectrum disorder. Journal of the American Academy of Child \& Adolescent Psychiatry, 51(1), 41-50.}.

. Déficit en la comunicación social e interacción social. Ejemplos:
.. Reciprocidad socioemocional
.. Conductas comunicativas no verbales
.. Desarrollo mantenimiento y comprensión de relaciones (ajuste del comportamiento al contexto).
. Patrones restrictivos y repetitivos del comportamiento, intereses o actividades. > 2:
.. Movimientos (estereotipados)
.. Rutinización
.. Restricción de intereses
.. Hiper o hiporreactividad a estímulos sensoriales
. Síntomas presentes en las primeras fases del desarrollo
. Deterioro clínicamente significativo
. Descartar RM (pueden ser comórbidos)

Especificadores:
* Con o sin déficit intelectual
* Con o sin deterioro del lenguaje
* Asociado o no afección médica / genética /factor ambiental
* Asociado a otro trastorno del neurodesarrollo / mental / del comoportamiento
* Gravedad (según grado de ayuda requerida: 3, 2, 1)

===== CIE-10

.F84.0 Autismo

Diagnóstico basado en 3 dimensiones.

. Desarrollo alterado antes de los 3 años de edad con afectación de 1 de 3:
.. Lenguaje / comunicación
.. Lazos sociales / interacción
.. Juego simbólico y funcional
. 6 síntomas de:
.. Alteraciones en la interacción social > 2 de:
... Falta de contacto visual, expresión facial, postura o gestos que regulen la interacción social.
... Falta de vínculos compartiendo intereses / actividades / emociones
... Falta de reciprocidad socioemocional. Débil integración del comportamiento social, emocional y comunicativo
.. Alteraciones en la comunicación > 1 de:
... Retraso/ausencia de lenguaje hablado sin gestos de compensación
... Fracaso para iniciar / mantener una conversación
... Uso estereotipado o idiosincrático del lenguaje
... Falta de juegos de simulación espontáneos o juego social imitativo
.. Intereses restringidos o estereotipados > 1 de:
... Comportamiento estereotipado
... Adherencia a rutinas carentes de sentido
... Manierismos motores estereotipados / repetitivos
... Preocupación por objetos carentes de funcionalidad
. Descartar otros TGD, trastornos del desarrollo específicos, RM, F20 de inicio en la infancia, S° de Rett.

.F84.5 Síndrome de Asperger

. Ausencia de retraso en el lenguaje o cognitivo
. Alteración cualitativa en la interacción social: similar criterio que el autismo.
. Intereses restringidos, repetitivos y estereotipados.
. Descartar otros TGS, F20, F21, TOC, TP anancástico, trastorno reactivo y desinhibido de la vinculación en la infancia.

No se requieren para el diagnóstico pero suelen estar presentes:

* torpeza motora
* capacidades especiales vinculadas a interés específico.

.Otros

* Autismo atípico: no se cumplen algunos de los criterios.
* Síndrome de Rett: niñas. Inicio normal, luego pérdida de funciones y retraso en crecimiento cefálico que aparece entre los 7m y 2 años.
* Otro trastorno desintegrativo de la infancia: inicio normal → pérdida de funciones.
* Trastorno hipercinético asociado a RM y movimientos estereotipados.


==== Diagnóstico diferencial

En adultos, sobre todo para S de Asperger: DD con Trastorno de la P del grupo A (Esquizoide)

==== Tratamiento

===== Biológico
En metaanálisis no hay diferencias significativas entre el placebo y los siguientes tratamientos footnote:[Yu, Yanjie, et al. "Pharmacotherapy of restricted/repetitive behavior in autism spectrum disorder: a systematic review and meta-analysis." BMC psychiatry 20.1 (2020): 1-11.]: fluvoxamina, risperidona, fluoxetina, citalopram, oxitocina, N-Acetilcisteína, buspirona.

Aripiprazol: potencialmente útil para uso transitorio en el tratamiento de aspectos comportamentales (irritabilidad, hiperactividad, estereotipias). Debe prestarse atención a los efectos secundarios (ganancia de peso, sedación, sialorrea, temblor). Un estudio muestra que a largo plazo no se diferencia del placebo -> se recomienda usar por períodos cortos footnote:[Hirsch, Lauren E., and Tamara Pringsheim. "Aripiprazole for autism spectrum disorders (ASD)." Cochrane Database of Systematic Reviews 6 (2016).].

===== Psicológico
Psicoterapia de apoyo con promoción de conductas sociales e interacción.
Técnicas de resolución de problemas.
Entrenamiento en habilidades sociales.
\end{document}