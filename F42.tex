\documentclass[encares.tex]{subfiles}
\begin{document}
\section*{Trastorno Obsesivo-Compulsivo}
\subsection*{Encare}
\subsubsection*{Agrupación sindromática}
\paragraph*{Síndrome obsesivo-compulsivo}
Dado por:
.Obsesiones
Ideas (imágenes o afectos) parásitas, asediantes, de carácter patológico, incontrolables, incoercibles, invasivos, recurrentes, persistentes, vividas como absurdas, sentidas como egodistónicas, aunque reconocidas como propias, cuya irrupción provoca una lucha ansiosa (síndrome de lucha) en los esfuerzos por ser controladas. Existen obsesiones:

A. Fóbicas: temor específico en ausencia de objeto, evitación imposible. Fobia a padecer enfermedades (Nosofobia), a morir (Tanatofobia), a microbios, Dismorfofobia. Por lo que caracterizamos del punto de vista sindromático como Fobia Obsesiva (temor específico que ocurre en ausencia de un objeto, la evitación es imposible ya que lo temido es la idea).

B. Ideativas: Pensamientos o meditaciones al respecto de idea concreta (palabras, cifras, objetos), idea abstracta (muerte, existencia de Dios), comportamiento (obsesiones ideicas, "locura de la duda") lleva a los rituales de verificación.

C. Impulsivas: Idea y miedo persistente de ser llevado a cometer de forma irresistible un acto absurdo, ridículo, inmoral, perjudicial o criminal. Puede existir acto conjuratorio, no existe evitación posible (obsesión impulsión). Temor ante objetos o situaciones que pueden llevarlo a cometer actos agresivos contra sí mismo o contra los demás: fobia de impulsión suicida u homicida. No pasan al acto.

.Compulsiones
Conductas incompartibles, incoercibles, invasivas, persistentes, estereotipadas, inadaptadas, vinculadas a sus obsesiones, egodistónicas, vividas como absurdas, ridículas y molestas, exageradas, ejecutadas de acuerdo a determinadas reglas, sin finalidad objetiva pero con sentido conjuratorio para el paciente, que disminuye la ansiedad, que toman la forma de:

* Actos: lavarse las manos, tocar objetos.
* Ritos o ceremoniales: actos encadenados (de limpieza o de verificación)
* Rituales mentales conjuratorios: rituales que funcionan en el interior del aparato psíquico. Se lo califica de leve, moderado o grave de acuerdo a la alteración de los pragmatismos, pérdida de tiempo o interferencia con la rutina habitual.

\paragraph{Síndrome de ansiedad-angustia}

Dado por una ansiedad difusa y permanente, que no se puede neutralizar a pesar de las compulsiones y vinculado a las obsesiones. Dado en dos vertientes:

* Vivencial: expectativa aprensiva o sensación de peligro inminente
* Somática: tensión motora, hiperactividad autonómica, vigilancia y control.

\paragraph{Síndrome depresivo}

Ver encare respectivo.

\subsubsection*{Personalidad y nivel}

Nivel: generalmente bueno.

Personalidad

* Conflictiva infantil
* Enuresis
* Onicofagia
* Terrores nocturnos
* Sonambulismo

Rasgos neuróticos globales:

* Mal manejo de la agresividad
* Trastornos en la esfera sexual
* Fatiga por sus conflictos
* Dependiente, inmaduro

Rasgos de carácter obsesivo: inhibidos, prudentes, puntuales, rigurosos en el tratamiento de convenciones sociales, tendencia a la duda, dependientes del jefe, cónyuge , familiares o amigos; laboriosidad, sentimiento de imperfección, inhibición de las emociones, meticulosidad, tendencia al orden, limpieza, detallista, autocontrol, seriedad, constancia, precisión, avaricia, actitudes moralistas acentuadas.

Cuando están presentes en grado suficiente, dan lugar a la personalidad obsesiva. Cuando alguno de ellos o + se hallan exagerados dando lugar a serios trastornos del comportamiento y de la capacidad de adaptación del sujeto, hablamos de personalidad anancástica. Estos rasgos se desarrollan gradualmente como defensa contra la ansiedad, dando lugar a pautas de conducta rígidamente fijadas y preestablecidas. Su utilidad consiste en mantener en el inconsciente los impulsos hostiles, agresivos o sexuales.

Nota Personalidad Anancástica de Kahn (rasgos anales) Personalidad dada por la tríada:

1. Orden: pulcritud, limpieza, puntualidad, meticulosidad, tendencia a la simetría, detallismo.
2. Avaricia: alto sentido de la propiedad, reservado, frugal.
3. Obstinación: tozudez, tenacidad, rigidez y desagrado por los cambios, controlado y controlador, precavido y racional, frío, distante.

Personalidad Psicasténica de Janet Duda, abulia, escrupulosidad, temor a avanzar, evita el enfrentamiento con el futuro, timidez, rigidez, dificultades sexuales, fondo depresivo crónico, sensación de incompletitud, tendencia al autoanálisis. Se evidencia al examen por curso lento buscando términos exactos, que dificulta la comunicación, excesiva racionalización y control de los afectos.

Términos psicoanalíticos

EROTISMO ANAL (retención): dificultad para separarse, terquedad, obstinación, coleccionismo, avaricia, tenacidad, perseverancia, egoísmo (sadicoanal al placer). Su contrario: tendencia a regalar, prodigalidad, generosidad, resignación, sumisión, temeridad (lucha contra el placer.

SADISMO ANAL (suciedad): suciedad, rechazo, resistencia a la autoridad, injurias escatológicas, crueldad. Su contrario: limpieza, educación, bondad, respeto, preocupación por la justicia, meticulosidad, puntualidad, perfeccionismo, sentido del deber, escrupulosidad, orden.

===== Diagnóstico positivo

====== Nosografía clásica

.Neurosis
Por ser un trastorno que afecta el corte longitudinal de su existencia, llevando a una alteración en la estructuración de la personalidad, siendo traducción de un conflicto intrapsíquico, que inhibe las conductas sociales, presentando un fondo permanente de ansiedad-angustia, siendo los síntomas egodistónicos (en conflicto con sus propias pautas), autoplásticos (no generan conductas ni exigencias hacia otros), existiendo conciencia de enfermedad (pide ayuda de forma voluntaria), con buen rapport y vínculo, sin pérdida de juicio de realidad.

.Neurosis obsesiva
Por asentar en un paciente con rasgos de personalidad obsesiva previa, más el predominio del síndrome obsesivo-compulsivo analizado, decimos que se trata de una Neurosis Obsesivo-Compulsiva de grado leve/moderado/grave según limitaciones sociales.

.Descompensada
Decimos que está descompensada por:

* Ansiedad-angustia
* Depresión neurótica
* Exacerbación de síntomas con falla de mecanismos de defensa
* Llevando a una alteración de conductas basales y pragmatismos

Causa de descompensación

Se plantea como causa de descompensación: estrés psicosocial, cambio, pérdida o por evolución natural del cuadro.

====== DSM-IV - CIE-10

Diagnóstico: Trastorno Obsesivo-Compulsivo. Ya que cumple con los criterios especificados:

Obsesiones y/o compulsiones definidas como:

.Obsesiones
1. pensamientos, impulsos o imágenes recurrentes y persistentes que se experimentan en algún momento del trastorno como intrusos e inapropiados y causan ansiedad o malestar significativos.
2. no se reducen a preocupaciones excesivas sobre problemas de la vida real
3. la persona intenta ignorar o suprimir estos pensamientos, impulsos o imágenes o bien intenta neutralizarlos con otros pensamientos o actos.
4. la persona reconoce que esto es el producto de su mente (no impuesto del exterior)

.Compulsiones
1. comportamientos o actos mentales de carácter repetitivo que el individuo se ve obligado a realizar en respuesta a una obsesión o con arreglo a ciertas reglas que debe seguir estrictamente.
2. el objetivo de estos comportamientos es la prevención o reducción del malestar o prevención de algún acontecimiento o situación negativos, sin estar conectados en forma realista con aquello que pretenden neutralizar o bien son claramente excesivos.

.Criterios adicionales:
* En algún momento el individuo lo percibió como excesivo e irracional.
* Malestar clínicamente significativo o pérdida de tiempo (> 1 hora/día) o interferencia con pragmatismos.
* El contenido de las obsesiones no se limita a otro trastorno del eje I.
* Descartar sustancias o enfermedad médica.

Especificadores: "con poca conciencia de enfermedad".

===== Diagnóstico diferencial

Otras formas de ideas/conductas

Hay que diferenciar Idea Obsesiva de Idea Sobrevalorada, Idea Fija o Idea Delirante. No creemos que esto corresponda a Ideas Fijas relacionadas con preocupaciones reales del sujeto. Son intensas y constantes como las ideas obsesivas, pero a diferencia de éstas el individuo no las vive como extrañas (son egosintónicas) ni patológicas.

Hay que diferenciar las Compulsiones de otras actividades "compulsivas" (comer, beber, jugar). Se diferencian porque las últimas producen placer en sí mismas y si el individuo se resiste es por sentirlas peligrosas no por sentirlas absurdas.

Con las impulsiones: no hay lucha previa, hay descarga directa en cortocircuito, sin que medie reflexión, son más típicas de lo orgánico (post-encefalitis, epilepsia, Gilles de la Tourette).

Neurosis fóbica

(si hay fobias límite): serían temores concretos con conductas acordes al temor, circunscriptas, con crisis de angustia, con conductas de evitación eficaces. En cambio las obsesiones fóbicas son temores mágicos con rituales independientes del temor, sin crisis de angustia, con ineficacia de la conducta de evitación. Los mecanismos de defensa implicados son distintos.

Neurosis de angustia

Si bien está de fondo, la ansiedad se ha visto canalizada por la instalación de mecanismos defensivos específicos. La ansiedad-angustia aparece como elemento de descompensación, centrando el cuadro en los mecanismos obsesivo-compulsivos que se ven exacerbados y sobrepasados.

Causa orgánica de la crisis de ansiedad-angustia

Ver neurosis de angustia.

Causas orgánicas del TOC

Tumores cerebrales.

Trastorno de personalidad obsesivo-compulsivo

Queda descartado porque el paciente es consciente de sus síntomas, son egodistónicos y éstos se dan fundamentalmente en el contexto intrapsíquico y no interpersonal.

Depresión mayor con rumiación obsesiva (depresión anancástica)

Si bien está presente la depresión pensamos que ésta es secundaria al trastorno neurótico analizado. En el trastorno afectivo no existen rituales, las ideas obsesivas no se sienten como intrusas ni extrañas y están centradas en temas vinculados a la depresión.

Inicio seudoneurótico de una esquizofrenia

Según limitaciones sociales. En este caso existiría un SDD. En las auténticas obsesiones falta la convicción e identificación morbosa con la idea que caracteriza al delirio. En la esquizofrenia las obsesiones son más extravagantes, menos precisas, con menor carga afectiva, vividas con cierta indiferencia, sin conciencia de enfermedad (las acepta pasivamente).

Algunas epilepsias temporales

Con "pensamiento forzado" (de Panfield): son automatismos, alucinaciones verbales simples, sin simbolismos, a veces con obnubilaciones de la conciencia, no tienen una personalidad obsesiva subyacente, ni se acompañan del cortejo sintomático de este cuadro y sobre todo son de naturaleza paroxística.

Rituales ligados a otras patologías

(Trastornos de la Conducta Alimentaria).

===== Diagnóstico etiopatogénico y psicopatológico

Etiopatogenia

Multifactorial:

Biológico

Genético

Más concordancia en gemelos idénticos.

Neurofisiológico y neuroquímico

Alteraciones neurofisiológicas: alteraciones en el mecanismo de inhibición frontal. Alteraciones neuroquímicas: en los sistemas serotoninérgico y dopaminérgico en los ganglios basales (núcleo caudado y putamen) durante el desarrollo que alteran el funcionamiento de dichos NT, lo que explicaría la acción de los ISRS. Esto está basado en la aparición de sintomatología OC en pacientes sometidos a la acción de m-CPP (agonista serotoninérgico), síntomas que se inhiben con metergolina (antagonista serotoninérgico no selectivo).

Neuroanatómico

Alteraciones neuroanatómicas: basado en la asociación entre sintomatología OC y varios síndromes neuropsiquiátricos o neurológicos y en la eficacia de la neurocirugía (Cingulotomía) para reducir los síntomas (resección de tractos que van desde el Cíngulo al Estriado).. • Estudios de neuroimagen: involucran de forma consistente 3 sitios de hiperactividad:

• Región orbital y medial de la Corteza Frontal.

• Núcleo Caudado (cabeza)

• Corteza del Cíngulo Estas áreas de hiperactividad se normalizan luego de un tratamiento medicamentoso o psicoterapéutico efectivo.

Neuropsicológico

Se plantea la existencia de alteraciones en el proceso de la información, involucrando en el TOC factores tales como:

• Pobreza en las estrategias perceptivas

• Déficit en aprendizaje y memoria visuoespacial

• Déficit en la memoria verbal

• Disociación entre la retroalimentación de la respuesta y la activación emocional.

Psicosocial

Se destacan como factores contribuyentes una educación rígida, moral estricta, culpabilizante que no permite un normal desarrollo del yo.

Psicopatología

Teoría Cognitivo-Comportamental

Modelo bifactorial de Mowrer, plantea una primera etapa donde se condicionan los estímulos neutros a través de un aprendizaje clásico Pavloviano. En un segundo estadio rige el refuerzo negativo según el cual se instauran nuevas respuestas cuando el sujeto aprende que con ellas disminuye la ansiedad que le provocan los estímulos condicionados. A nivel cognitivo, se enfatiza en la existencia de distorsiones cognitivas.

Teoría Psicoanalítica

Para el psicoanálisis comporta una regresión a la fase sádico-anal (relacionado con la retención y el control posesivo del objeto) como consecuencia del conflicto edípico. Esta regresión da lugar a la aparición de modos de funcionamiento primitivos del yo y del super-yo (pensamiento mágico con creencia en la omnipotencia del pensamiento que hace que las ideas agresivas sean terroríficas). Este proceso, junto al empleo de mecanismos de defensa propios de la etapa pregenital como el aislamiento, la anulación retractiva y la formación reactiva, da lugar a la aparición de obsesiones, compulsiones y el carácter obsesivo. Ante el estrés psicosocial, los mecanismos de defensa se ven sobrepasados, no pudiéndose mantener las exigencias pulsionales reprimidas y aparece la angustia. El Yo queda al descubierto y sometido a los ataques de un Superyo sádico y rígido. El Yo se defiende por medio de:

• Anulación: proceso activo que consiste en deshacer psíquicamente lo que acaba de realizarse, de forma mágica y omnipotente (explica ritos).

• Aislamiento: separa la representación de su afecto de manera que puede permanecer en la conciencia y dar lugar a la formación de obsesiones, compulsiones y el carácter obsesivo.

• Formación reactiva: pautas de comportamiento, sentimientos o deseos, diametralmente opuestos a los deseos reprimidos. Contribuye a la formación de rasgos de carácter. La sintomatología puede expresar tanto un deseo como medidas protectoras contra éstos. En un intento por hacer un enfoque más comprensivo de este paciente podemos vincular desde el punto de vista psicológico el surgimiento de sus síntomas (independientemente del modelo teórico que usemos para explicarlos) con:

• Agresividad latente

• Educación rígida, severa, culpabilizante, moral, figura paterna rígida.

• Miedo a la agresividad (como reacción a lo anterior)

• Pensamiento catastrofista

• Temor al descontrol emocional por temor a que sea letal

• Intolerancia a la ambigüedad

• Culpa en relación a todo lo sexual

• Actitud especial ante la autoridad: se inclinan ante la fuerza, pero tratan de desquitarse por medio de algo que anule su sumisión.

• Actividad sexual desprovista de placer e incluso vivida como castigo.

NOTA: los resultados altamente eficaces de la psicoterapia comportamental, relegaron los aportes de la teoría psicoanalítica que se anotan aquí por ser clásicos y porque aún pueden ser relevantes para el Trastorno Obsesivo-Compulsivo de la Personalidad. En lo que concierne al TOC, las evidencias de una alteración orgánica son importantes. Por otro lado, no hay hallazgos que indiquen que determinados rasgos de personalidad sean factores predisponentes para el TOC (hay datos de lo inverso: el TOC puede estar en la base de un TPOC), por lo que difícilmente puede encuadrarse hoy este trastorno dentro de las clásicas "neurosis".

===== Paraclínica

====== Biológico

EF completo y valoración general. Según hallazgos realizaremos consultas con especialistas, buscando descartar causas orgánicas de la ansiedad-angustia. Con vistas a un eventual tratamiento con AD Tricíclicos: examen CV y ECG buscando descartar trastornos de la conducción, extrasístoles.

====== Psicológico

Reiteradas entrevistas para mayor acercamiento a la conflictividad del paciente, evaluando:

* Significado de los síntomas para el paciente.
* Refuerzo de las conductas por parte de personas allegadas
* Hasta qué punto ha organizado su vida alrededor de sus síntomas

Solicitaremos psicodiagnóstico. Tests de personalidad proyectivos y no proyectivos, test de nivel (Weschler), evaluando:

* Fortaleza yoica
* Mecanismos de defensa y manejo de la angustia
* Implementación de psicoterapia Esperamos un perfil con picos en las escalas de psicastenia y quizás depresión y trastornos psicosexuales. En los proyectivos: detalle minucioso en detrimento del conjunto, dudas, intelectualización, críticas a las láminas, fragmentación del relato. Confirmaremos los mecanismos de defensa característicos del Superyo. Pueden encontrarse indicios de relaciones objetales sadomasoquistas, narcisistas y ambivalentes. Puede encontrarse una estructura borderline o psicótica subyacente de la que se defiende con un cuadro obsesivo.

====== Social

Red de soporte social, entrevistas con terceros significativos evaluando la adaptación a sintomatología del paciente. Consultaremos con AS para que visite el hogar observando organización del grupo familiar: distribución de roles, comunicación, estabilidad, continencia, existencia de factores que favorezcan la patología del paciente.

===== Tratamiento

Ambulatorio, se controlará en policlínica con frecuencia para mejor continentación por el monto de angustia, ya que no pasa al acto. Destinado a:

* Yugular el cuadro actual
* Compensar la enfermedad de fondo

====== Biológico

Disminuir la ansiedad: benzodiacepina de vida media intermedia/larga tal como el Clonazepam a dosis iniciales de 1-4 mg repartidos en 2 tomas, pudiendo comenzar con 1 mg cada 12 horas. Además de su acción sobre la ansiedad, existen reportes de una posible acción como antiobsesivo, por lo que la preferimos frente a otras benzodiacepinas. Fármacos con acción sobre la sintomatología obsesivo-compulsiva: Los fármacos de elección son los antidepresivos con acción a nivel del sistema serotoninérgico. Hay datos que indican que la eficacia en el TOC está en relación inversa con la potencia serotoninérgica del fármaco, por lo que en orden de eficacia tenemos: Clorimipramina -> Fluoxetina / Fluvoxamina -> Paroxetina -> Sertralina -> Citalopram. Por tener mayores efectos secundarios con mayores tasas de abandono de la medicación, consideramos la Clorimipramina como un fármaco de segunda línea. En nuestro paciente realizaremos una prueba terapéutica con Fluvoxamina, la que preferimos por ser un antidepresivo también eficaz para los síntomas de ansiedad. Comenzaremos con dosis de 50 mg/día (para evitar efectos secundarios gastrointestinales) en una sola toma que puede ser nocturna (por sus efectos sedativos). Al 4° día aumentaremos a 100 mg/día en una toma. Aumentaremos según la respuesta (que puede aparecer en forma parcial a las 2 semanas), pudiendo llegar a 300 mg/día (en dosis mayores a 150 mg/día repartiremos la dosis en 2 tomas). En caso de falta de respuesta, realizaremos una segunda prueba terapéutica con un ISRS. En este caso usaríamos Fluoxetina dosis iniciales de 20 mg/día en una sola toma, que iremos aumentando según respuesta, sabiendo que en el TOC generalmente se requieren altas dosis, llegando en muchos casos a las dosis máximas (80 mg/día). También sabemos que la respuesta tiene una latencia de al menos 6-8 semanas, siendo 12 semanas el plazo adecuado para cada prueba terapéutica. Luego de 2 ensayos sin respuesta podemos considerar este caso como refractario, teniendo varias alternativas: Si hubo una respuesta parcial con alguno de los ISRS:

• Agregar Pindolol: 2,5 mg cada 8 horas. Esta opción es la preferencial en caso de que se haya obtenido respuesta parcial con un ISRS.

Si no hubo respuesta con los ISRS:

• Agregar / Sustituir por Clorimipramina: comenzando con Clorimipramina 37,5 mg v/o al acostarse (por los efectos sedativos), c/ aumentos de 37,5 mg c/ 2-4 días. Estaremos atentos a los efectos secundarios (sequedad de boca, visión borrosa, constipación, dificultad en la micción, hipotensión postural). Aumentaremos según respuesta clínica hasta llegar a los 150-300 mg/día, sabiendo de la latencia de aprox. 2 meses en su efecto antiobsesivo.

• Agregar Risperidona a dosis de 2 a 4 mg/día repartidos en 2 tomas.

Opciones ante TOC grave refractario:

• Clorimipramina parenteral

• Psicocirugía (cingulotomía anterior, capsulotomia anterior, tractotomía en subcaudado, leucotomía límbica). Luego de la psicocirugía puede que un paciente anteriormente refractario responda por lo que puede realizarse un nuevo ensayo terapéutico.

Casos especiales:

• En un TOC de tipo "Simetría" refractario podría realizarse un ensayo terapéutico con IMAO (con precauciones ante interacciones medicamentosas / dietéticas).

• En un TOC en el contexto de un Trastorno de Tics (Gilles de la Tourette), puede obtenerse una mejor respuesta asociando un ISRS a Pimozide o Haloperidol a bajas dosis.

• TOC + Esquizofrenia: tratar el TOC de forma independiente (considerar el uso de Risperidona).

Otras opciones:

• Opciones sin evidencia suficiente de eficacia: Olanzapina, Triptofano

• Opciones con evidencia contradictoria: Buspirona, hormonas tiroideas, Litio, Clozapina, Trazodona, IMAO (podría ser eficaz para el TOC de tipo "simetría"), Clonazepam, Inositol, terapia con Antiandrógenos.

• Opciones con evidencia de ineficacia: ECT. El tratamiento que muestre eficacia será mantenido por un tiempo prolongado a dosis elevadas ya que la suspensión índice de recaídas.

Psicológico

Entrevistas reiteradas buscando afianzar el vínculo, profundizando en la evaluación del paciente, con sesiones de terapia de apoyo con una actitud de comprensión, escucha y neutralidad, buscando mejor nivel de funcionamiento. En casos leves, podría plantearse el manejo exclusivamente psicoterapéutico, teniendo la terapia Cognitivo-Comportamental índices de eficacia similares a los obtenidos con fármacos. En casos moderado y graves, la psicoterapia sería un coadyuvante de la medicación. En esta modalidad terapéutica se usan técnicas tales como: exposición con prevención de respuesta y detención del pensamiento.

Social

Psicoeducación del paciente y familia, brindando a éstos apoyo emocional y seguridad. Vincularemos al paciente con grupos de autoayuda para pacientes obsesivo-compulsivos. Evaluación de posibilidad de terapia familiar realizada por especialista.

===== Evolución y pronóstico

Evolución: normalmente crónica con variación en la intensidad de los síntomas. Un 5-10% de casos tienen evolución grave crónica con invalidez importante por ritualización de la existencia. Con el tratamiento esperamos alterar el curso natural ya que c/él los índices de curación y mejoría (en un 75%). El curso puede estar marcado por la frecuencia de episodios depresivos. En cuanto al pronóstico en lo inmediato, pensamos yugular el cuadro depresivo y de ansiedad-angustia con el tratamiento instituido. Difícil pasaje al acto. En lo alejado: dependerá de la respuesta a la medicación y a las medidas psicoterapéuticas y de la adhesión al tratamiento por parte del paciente y la familia. Muy difícil manejo. Kaplan: 15% curación, 45% mejoría, 40% igual o empeoran.

Elementos de buen pronóstico:

* poca antigüedad de los síntomas
* desencadenantes ambientales
* buena adaptación social

NOTAS Fobias límite u obsesiones fóbicas Lo temido no es la situación real sino la idea de la situación. Al principio pueden ser más tipo fobias y luego se generalizan. Serie de fenómenos difíciles de adscribir a lo fóbico o a lo obsesivo, puesto que se encuentran en medio del espectro y tienen características de los 2. Se decide la pertenencia según el predominio relativo del resto de la sintomatología. No hay evitación posible porque se trata de una idea. Fobias de impulsión Miedo irracional a ejecutar una forma de acción a la que el paciente se siente impulsado. Esta idea aparece obsesivamente y por lo general es:

• Fobia de impulsión suicida

• Fobia de impulsión homicida (por lo general a un ser querido). Se vincula con la fobia a los cuchillos, armas u objetos cortantes, evitándolos por el miedo excesivo que se siente de pasar al acto. Obsesiones fóbicas Eritrofobia: temor a ruborizarse en público. Por lo general se vincula más a la vergüenza de que una falta (generalmente de tipo sexual) se le lea en la cara. Nosofobia: temor a una enfermedad, a contaminarse por un virus, microbios o suciedad. Temor obsesivo a ser dañado por un agente exterior (genera rituales de limpieza). Es un temor obsesivo hipocondríaco. Tanatofobia: temor a la propia muerte o a la de un ser querido. Dismorfofobia: idea obsesiva de que una parte del cuerpo es deforme o desagradable.

Recomiendo leer: The Journal of Clinical Psychiatry, Vol 63, Supp. 6 (2002), con una revisión sobre TOC, trastornos TOC símiles, TOC en la infancia y TOC refractario. Algunos datos del encare están sacados de revisiones de esa revista.
\end{document}