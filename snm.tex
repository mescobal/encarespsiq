\chapter{Síndrome Neuroléptico Maligno}
\section*{Notas clínicas}
Puede ser causado por TODOS los antipsicóticos (51\% APPG y 45\% APSG: incluyendo aripiprazol, asenapina, clozapina, iloperidona, olanzapina, paliperidona, quetiapina, risperidona y ziprasidona). Incidencia 0.2\%, sobre todo con AP1G incisivos (Haloperidol en el 44\% de todos los casos), menor con AP2G. AP1G sedativos no tienen registro como causa única de SNM. 
Factores de riesgo: patología orgánica preexistente, uso concomitante de litio, infección, retirada de anticolinérgicos o alcohol. \footnote{Schneider, M., Regente, J., Greiner, T., Lensky, S., Bleich, S., Toto, S., ... \& Heinze, M. (2020). Neuroleptic malignant syndrome: evaluation of drug safety data from the AMSP program during 1993–2015. European Archives of Psychiatry and Clinical Neuroscience, 270, 23-33.}. Es más frecuente en hombres en un 50\%, sobre todo jóvenes. \footnote{Gurrera, R. J. (2017). A systematic review of sex and age factors in neuroleptic malignant syndrome diagnosis frequency. Acta Psychiatrica Scandinavica, 135(5), 398-408.}. 
Desarrollo: 16\% dentro de las 24 horas de iniciación del antipsicótico, 66\% a la semana, 90\% ocurren dentro de los primeros 30 días.
\section*{Encare}
\subsection*{Agrupación sindromática}
\subsubsection*{Síndrome de alteración de la conciencia}
Ver Delirium
\subsubsection*{Síndrome de inhibición psicomotriz}
Estupor catatónico.
Rigidez muscular generalizada (en caño de plomo)
Aquinesia.
\subsubsection*{Fenómenos acompañantes}
Fiebre, temblor, sialorrea. Distonía, trismo, mioclono. Disartria, disfagia, rabdomiólisis.
Secuencia habitual(70\%): alteración de conciencia \faArrowRight rigidez muscular \faArrowRight fiebre \faArrowRight disfunción autonómica.
\subsection*{Diagnóstico}
Es un diagnóstico de exclusión (ver diferenciales).
\subsubsection*{Criterios propuestos por DSM-IV}
A. Rigidez muscular intensa + fiebre + asociación temporal con administración de antipsicótico.
B. 2 o + de: diaforesis, disfagia, temblor, incontinencia, disminución del nivel de conciencia (de confusión a coma), mutismo, taquicardia, PA elevada o fluctuante, leucocitosis, aumento de CPK o equivalente.
C. Descartar otras causas (drogas, neurológicas)
D. No se explica mejor por otro cuadro (Síndrome catatónico)
\subsubsection*{Criterios de expertos}
Un consenso de expertos propuso criterios algo más restrictivos\cite{ware2018neuroleptic}:
1. Exposición a un antagonista de DA o retiro de un agonista de DA en las 72 horas anteriores al inicio de los síntomas
2. Hipertermia en 2 ocasiones mayor o iguala 38oC oral.
3. Rigidez
4. Alteración del estado mental con reducción o fluctuación de niveles de conciencia.
5. Elevación de la CPK al menos 4 veces por encima de lo normal.
6. Labilidad del sistema Simpático en al menos 2 de los siguientes parámetros:
a. PA > 25\% sobre lo basal
b. Fluctuaciones de PA > 20 mmHg en la diastólica o 25 mm Hg en la sistólica dentro de 24 horas
c. Diaforesis
d. Incontinencia urinaria.
7. Hipermetabolismo definido como FC \faArrowUp 25\% sobre lo basal y FR \faArrowUp >50\% sobre lo basal.
8. Ausencia de otras etiologías (infecciones, toxinas, causas metabólicas o neurológicas).
\subsection*{Diagnóstico diferencial}
Causas orgánicas:
\begin{itemize}
	\item Causas neurológicas:
	\begin{itemize}
		\item Drogas: estados hipermetabólicos secundarios a drogas (fenilciclidona)
		\item Infecciosas: encefalitis viral aguda, tétanos, infecciones parasitarias, bacterianas, micóticas
		\item Efecto de masa: tumores, abscesos
		\item ACV
		\item Traumático
	\end{itemize}
	\item Catatonía: en un trastorno del humor o en esquizofrenia. La catatonía maligna puede ser indistinguible del SNM.
	\item Endocrinológico: feocromocitoma, tirotoxicosis
	\item Sistémico: LES, enfermedades del tejido conectivo.
	\item Otros: insolación, síndrome serotoninérgico, toxinas, hipertermia maligna luego de exposición a agentes anestésicos inhalatorios, hipertermia parkinsoniana por discontinuación de agonistas DA (levodopa, amantandina), hipertermia por estimulantes (cocaína, anfetaminas) o abuso de alucinógenos (fenciclidina), envenenamiento anticolinérgico, abstinencia de alcohol o sedantes.
\end{itemize}
\subsection*{Paraclínica}
Valoración general
* Hemograma: leucocitosis.
* Ionograma: hiponatremia / hipernatremia.
Gasometría: acidosis metabólica o hipoxia en 75\%
CPK: elevada * 4 (95\%)
Orina: mioglobinuria (67\%)
Función renal: falla renal aguda por necrosis muscular producto de la rigidez, hipertermia e isquemia.
EEG: enlentecimiento difuso en 54\%
Sideremia: Fe disminuido
Estudio de LCR y neurogimagen: no son de primera línea, suelen dar normales.
\subsection*{Etiopatogenia y fisiopatología}
Desencadenado por bloqueo de receptores D2 de centros reguladores del hipotálamo y tronco cerebral que provocan un síndrome hipermetabólico generalizado sistémico. Se postula que la disregulación con hiperactividad del SN Simpático explica muchas manifestaciones del SNM. 
Una disregulación del Simpático previa por estrés emocional podría constituir una vulnerabilidad par el SNM. Otro factor de vulnerabilidad sería la hipoactividad D2 previa.
Factores de riesgo: agitación psicomotriz, administración parenteral, aumento rápido de dosis, dosis total diaria elevada, varón joven, síndrome orgánico cerebral, retardo mental.
Otros factores: extenuación, deshidratación, malnutrición, episodios previos de SNM (17-30\% de incidencia si se someten nuevamente a antipsicóticos).
\subsection*{Tratamiento}
Cesación de agente causante.
Tratamiento de apoyo
\begin{itemize}
	\item monitorización constante
	\item aporte de volumen: agresivo.
	\item corrección de electrolitos
	\item fluidos alcalinizados o carga con bicarbonato puede prevenir falla renal
	\item persistencia de hipertermia: medidas físicas para bajar temperatura
\end{itemize}
Farmacológico:
\begin{itemize}
	\item Benzodiacepinas: Lorazepam i/v 1-2 mg cada 4-6 horas. Reducción de rigidez y fiebre en 24-47 horas, remisión de síntomas catatónicos (mutismo e inmovilidad).
	\item Agentes dopaminérgicos: revierten parkinsonismo, \faArrowDown el tiempo de recuperación y \faArrowDown la mortalidad a la mitad solos o en combinación.
	\begin{itemize}
		\item Amantadina 200-400 mg/día en dosis divididas v/o o por SNG.
		\item Bromocriptina 2.5 mg c/12 o c/8 aumentando hasta un total de 45 mg/día si se requiere. Puede empeorar la psicosis y precipitar hipertensión y vómitos. Debe continuarse 10 días después de la remisión para evitar recurrencia si se discontinúa precozmente.
	\end{itemize}
	\item Dantrolene: relajante de músculo esquelético. Útil en caso asociados con hipertermia extrema y rigidez. Se puede usar junto con BZD o con un agonista DA. No se puede administrar con bloqueadores de Ca++. Dosis: inicio 1-2.5 mg/kg IV, luego 1 mg/Kg cada 6 horas si hay respuesta luego de la primera dosis. Efectos secundarios: insuficiencia respiratoria / hepática. Debe continuarse 10 días luego de la resolución de los síntomas por probabilidad de recurrencia si se retira precozmente.
	\item ECT: puede ser eficaz. Segunda línea, si fallan fármacos o si no se puede descartar una catatonía letal. Se hacen 6-10 sesiones. Aparece respuesta a la 4\textsubscript{a}. Vigilar la aparición de lesión muscular y de hiperkalemia.
\end{itemize}
\subsection*{Evolución y pronóstico}
Luego de ser reconocido y tratado, el SNM es autolimitado a menos que existan complicaciones. El tiempo promedio de recuperación son 7-10 días luego de la discontinuación de la droga. Casi todos los pacientes se recuperan en 30 días o menos. El uso de medicación de depósito puede dar episodios del doble de duración. En algunos pacientes puede haber catatonía y parkinsonismo residual por meses. La ECT suele ser útil para estos síntomas residuales.
Tasa de mortilidad: originalmente era de un 30\%, actualmente hay reportes de 0-15\%.

Buen pronóstico:
\begin{itemize}
	\item diagnóstico precoz
	\item rápida discontinuación del antipsicótico
	\item uso de farmacoterapia específica
\end{itemize}
Mal pronóstico:
\begin{itemize}
	\item mioglobinuria
	\item falla renal
\end{itemize}
La muerte en general se produce por falla cardíaca, respiratoria, neumonia por aspiración, embolia pulmonar, falla renal mioglobinúrica o coagulación intravascular diseminada.

Si se reinicia el antipsicótico de forma inmediata: recurrencia de un 30\%. Se recomienda esperar al menos 2 semanas después de la resolución para retomar cualquier antipsicótico- Se recomienda una dosificación lenta / gradual de neuroléptico sedativo o de APSG. Preferentemente usar antagonistas parciales de DA (aripiprazol, brexpiprazol, cariprazina). Obtener consentimiento informado antes de reiniciar. En general no hay recurrencia si se empieza la misma droga luego de las 4 semanas de la recuperación del episodio inicial.
\printbibliography
