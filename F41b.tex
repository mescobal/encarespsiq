\chapter{Trastorno de ansiedad generalizada}
\section*{Notas clínicas}
\section*{Encare}
\subsection*{Agrupación sindromática}
\subsection*{Personalidad y nivel}
\subsection*{Diagnóstico}
\subsubsection{DSM IV}
Criterios diagnósticos:

A. Ansiedad/preocupación excesivas ante amplia gama de estímulos de más de 6 mese de duración.

B. Dificultad para controlar la ansiedad.

C. 3 o + de 6 (en niños solo 1): inquietud / impaciencia, fatigabilidad, alteraciones en concentración, irritabilidad, tensión muscular, alteración del sueño (IFCITA).

D. Excluye: síntomas debido a otros trastornos del eje I (sobre todo trastornos de ansiedad).

E. Malestar clínicamente significativo.

F. Descartar sustancias, enfermedad médica y otros trastornos de eje I.

\subsection*{Diagnósticos diferenciales}

Comorbilidad más frecuente: depresión.

\subsection*{Diagnóstico etiopatogénico y psicopatológico}
Se plantea que los trastornos de ansiedad en general tienen una propensión heredable de un 30-50\% \footnote{Shimada-Sugimoto M, Otowa T, Hettema JM: Genetics of anxiety disorders: genetic epidemiological and molecular studies in humans. Psychiatry Clin Neurosci 2015; 69:388–401} con una genética compleja que involucra muchos genes,cada uno con un aporte pequeño al riesgo genético total.

Los factores de riesgo no genéticos (no específicos para ansiedad) son: estilo parental \footnote{Otowa T, Gardner CO, Kendler KS, et al.: Parenting and risk for mood, anxiety and substance use disorders: a study in population-based male twins. Soc Psychiatry Psychiatr Epidemiol 2013; 48:1841–1849}, aprendizaje social \footnote{Mineka S, Zinbarg R: A contemporary learning theory perspective on the etiology of anxiety disorders: it’s not what you thought it was. Am Psychol 2006; 61:10–26}, adversidad en la infancia (exposición a estrés, maltrato, bajo nivel socioeconómico) \footnote{Jaffee SR: Child maltreatment and risk for psychopathology in childhood and adulthood. Annu Rev Clin Psychol 2017; 13:525–551} \footnote{Gur RE, Moore TM, Rosen AFG, et al.: Burden of environmental adversity associated with psychopathology, maturation, and brain behavior parameters in youths. JAMA Psychiatry 2019; 76:966–975}. Algunos rasgos de personalidad vinculados a reactividad aumentada al estrés y a afectos negativos (neuroticismo) están asociados a ansiedad y trastornos depresivos \footnote{Barlow DH, Ellard KK, Sauer-Zavala S, et al.: The origins of neuroticism. Perspect Psychol Sci 2014; 9:481–496}.

En la infancia, aparece la inhibición comportamental en respuesta a extraños o situaciones nuevas como un predictor precoz de ansiedad en lavida adulta \footnote{Kagan J, Reznick JS, Snidman N: The physiology and psychology of behavioral inhibition in children. Child Dev 1987; 58:1459–1473}.
\subsection*{Paraclínica}
\subsection*{Tratamiento}
El abordaje farmacológico es de primera línea.

Recomendaciones: duloxetina > pregabalina > venlafaxina > escitalopram. Alternativa: bupropion.\footnote{Slee, A., Nazareth, I., Bondaronek, P., Liu, Y., Cheng, Z., \& Freemantle, N. (2019). Pharmacological treatments for generalised anxiety disorder: a systematic review and network meta-analysis. The Lancet, 393(10173), 768-777.}. Con menor evidencia: mirtazapina, sertralina, fluoxetina, buspirona y agomelatina. Quetiapina: efecto marcado pero con mala tolerabilidad. Paroxetina y benzodiacepinas: mala tolerabilidad.

Duloxetina: comenzar con 30 mg/día por 1 semana y luego 60 mg/día. Si hay respuesta parcial: aumentar hasta 120 mg día.

Pregabalina: comenzar con 150 mg/dia en 2 o 3 tomas, aumentos semanales hasdta dosis máxima de 600 mg/día.

Venlafaxina: comenzar con 75 mg/día por 4 días, luego 150 mg/día. Máximo 225 mg/día en 2 o 3 tomas (preparados de liberación sostenida: 1 toma diaria)
\subsection*{Evolución y pronóstico}
\printbibliography
