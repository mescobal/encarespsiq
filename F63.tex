\chapter{Cleptomanía}
\section*{Encare}
\subsection*{Agrupación sindromática}
\subsubsection{Síndrome de alteración del control de impulsos}
Dado por fracaso reiterado en los intentos de resistirse al impulso de robar objetos que no se utilizan para uso personal o por fines lucrativos (los objetos suelen deshecharse, regalarse o esconderse). El acto está precedido por sensación de tensión emocional y seguido de gratificación durante e inmediatamente después de realizar el acto. El acto se realiza de forma solitaria sin grandes esfuerzos para evitar ser descubierto. B. Otros síndromes síndrome de ansiedad angustia, síndrome depresivo (vinculado a culpa). Esto no impide la repetición del acto.
\subsection*{Personalidad y nivel}
\subsection*{Diagnóstico positivo}
Cleptomanía
\subsection*{Diagnóstico diferencial}

• Hurto recurrente sin trastorno mental manifiesto: son actos planificados y existe un motivo de ganancia personal.
• Deterioro cognitivo (olvido de pagar objetos, alteración de juicio con desinhibición que lleva a conductas de robo)
• Psicosis: hurto en contexto discordante, delirante o por comando alucinatorio.
• Trastorno de la personalidad borderline o antisocial: en este caso no hay una conducta maladaptativa sino un patrón que afecta todas las áreas de la vida de la persona
\subsection*{Etiopatogenia y psicopatología}
\subsection*{Paraclínica}
Biológico: ninguna. Psicológico: tests de personalidad.
\subsection*{Tratamiento}
===== Biológico
Antidepresivos + tratamiento de cuadros comórbidos.
===== Psicologico
• exposición in vivo + prevención de respuesta
• reestructuracion cognitiva
• sensibilización encubierta
• desensibilización sistematica
==== Evolución y pronostico
=== Fuentes
Comité de consenso de Catalunya. Recomendaciones
