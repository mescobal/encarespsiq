== F90 Trastornos hipercinéticos

=== Notas clínicas

=== Encare

==== Paraclínica

* Consulta con cardiólogo previo al uso
* Controles de PA regulares
* Periódicamente: hemograma (raro: leucopenia / anemia) en caso de tratamiento prolongado.

==== Tratamiento

===== Biológico
Adolescentes: 
- primera línea: Mefilfenidato, Dexanfetamina, lisdexanfetamina
- segunda línea: atomoxetina (si están CI los estimulantes, si no los tolera, si no respondió a 2 estimulantes distintos a dosis adecuadas)
Adultos:
- Primera línea: Metilfenidato, dexanfetamina, lisdexanfetamina
- Segunda línea: Atomoxetina o guanfacina (si están CI los estimulantes, si no los tolera, si no respondió a 2 estimulantes distintos a dosis adecuadas)
- Tecera línea: Bupropion, Clonidina, Modafinilo, Reboxetina, Venlafaxina
.Metilfenidato
Estimulante del SNC.
Aprobado por la FDA para el tratamiento del TDAH en niños y adultos.
Usos no aprobados por FDA: narcolepsia, tratamiento de la depresión resistente.

Mecanismo: ↑ NA y DA x bloqueo de recaptación. ↑ acción en corteza dorsolateral prefrontal → mejora atención, concentración, funciones ejecutivas y vigilia. El aumento de DA en otras regiones (ganglios basales) puede mejorar la hiperactividad. Por acción en la corteza prefrontal medial y el hipotélamo: mejoría en depresión, fatiga y somnolencia.
Inicio de acción: 30'. Máximo beneficio terapéutico: varias semanas.
Si es efectivo: tratamiento prolongado a la vida adulta.
Si no es efectivo: ajuste de dosis, cambio de presentación o cambio de fármaco.
Siempre intentar con monoterapia antes de plantearse terapias combinadas.
Combinaciones:

* Liberación inmediata + liberación prolongada
* Con Modafinil o Atomoxetina
* Con antipsicóticos atípicos en casos de TB y/o TDAH con mucha resistencia al tratamiento
* Con antidepresivos en caso de depresión resitente

Presentaciones:

* Liberación inmediata: duración de acción 4-6 horas (2-3 tomas al día)
** Ritalina (Novartis) 10 mg x 30 comprimidos
** Rubifen (Servimedic) 10 mg x 30 comprimidos.
* Liberación bifásica: duración de efecto 8 horas (una toma diaria)
** Ritalina LA (Novartis) 20  mg x 30 comp.
** Ritalina LA (Novartis) 30 mg x 30 comp
* Liberación bifásica  prolongada: 12 horas (una toma diaria)
** Concerta (Janssen) 18 mg x 30 comprimidos
** Concerta (Janssen) 36 mg x 30 comp.

Efectos secundarios:

Por ↑ NA periférica: efectos autonómicos. Temblor, taquicardia, HTA, arritmias.
Por ↑ NA y DA central: insomnio, agitación, psicosis, abuso.
Notables: insomnio, cefaleas, ↑ de tics, nerviosismo, irritabilidad, sobreestimulación, temblor, mareos. Anorexia, náuseas, dolor abdominal, disminución de peso. Discutido: retraso del crecimiento en niños.
Peligrosos: episodios psicóticos (especialmente si hay abuso parental), priapismo (raro), convulsiones, palpitaciones, taquicardia, HTA, SNM (raro), activación de hipo/manía o ideación suicida (discutido). MS en pacientes con anomalías CV preexistentes.
Manejo: β-bloqueantes para efectos autonómicos periféricos. En general no sirve agregar fármacos. Mejor cambiar.

Dosis:
Rango: 2,5 - 10 x 2 en intervalos de 4 horas. En liberación extendida. Similar pero dosis única (máximo 30 mg).
Tip: presentación racémica es mitad de dosis de la no racémica. Liberación extendida: tiene la 1/2 como inmediata y la 1/2 retardada → liberación en 2 pulsos. Comida: retrasa el pico en 2-3 horas. Vida media de eliminación 2.2 horas. No inhibie CYP450.

Uso prolongado: puede aparecer dependencia/abuso, puede aparecer tolerancia. Se discute si se asocia o no a supresión del crecimiento. Para discontinuarla: gradual. En el uso abusivo: vigilar si hay depresión después de discontinuar.

Interacciones: puede inhibir el metabolismo de ISRS, anticonvulsivantes (fenobarbital, fenitoina, primirdona), anticoagulantes cumarínicos → bajar las dosis de estas drogas. No combinar con clonidina (potenciales efectos advesos serios). Las acciones del MF podrían potenciarse con bloqueadores de NA: ADT, desipramina, venlafaxina, duloxetina, atomoxetina, milnacipram, reboxetina. En teoría los antipsióticos inhibirían el efecto estimulatorio del MF. Y el MF inhibiría el efecto antipsótico y estabilizador del humor de los AP atípicos.
Para expertos: combinación de MD con APA, anticonvulsivantes o litio.
No dar con antiácidos: pueden alterar la liberación de la formulación de liberación extendida.

Precauciones:
* en HTA, hipertirodismo o historia de abuso de sustancias
* en niños con detención del crecimiento
* peoría de tics
* en paciente psicóticos puede empeorar la desorganización del pensamiento o del comportamiento
* potencial de abuso
* monitorización en el uso prolongado
* asociado a muerte súbita en niños con problemas cardíacos.
* puede bajar el umbral convulsivo.
* puede dar viraje ("inducción de estado bipolar") requiriendo discontinuación o agregado de estabilizador del humor.
* NO USAR en apcientes agitados
* precaución si hay tics o Tourette
* no usar con IMAOs o dentro de los 14 días de suspender un IMAO.
* evitar en glaucoma, anomalías cardíacas estructurales, angioedema, anafilaxis, alergia MF

Situaciones especiales:
* IR o IH: no lleva ajuste de dosis
* Cardiopatía: precaución o no usar.
* Añosos: menores dosis son mejor toleradas
* Embarazo: categoría C. Se prefiere discontinuar.
* Lactancia: discontinuar.

.No farmacológico
Actúan sobre todo sobre la variable "inhibición" y menos sobre la memoria de trabajo.
- Ejercicio físico \footnote{Lambez, B., Harwood-Gross, A., Golumbic, E. Z., \& Rassovsky, Y. (2020). Non-pharmacological interventions for cognitive difficulties in ADHD: A systematic review and meta-analysis. Journal of psychiatric research, 120, 40-55.}. Es la intervención con mayor tamaño de efecto. Se recomienda ejercicio aeróbico e integrar deportes complejos (deportes con pelota, artes marciales) para mejorar la flexibilidad e inhibir la conducta impulsiva.
- Neurofeedback: tamaño de efecto moderado.

===== Psicológico
Al igual que los tratamientos no farmacológicos la acción es sobre todo sobre la inhibición más que sobre la memoria de trabajo.
- Psicoterapia CC: tamaño de efecto moderado.
- Entrenamiento cognitivo: menor tamaño de efecto.