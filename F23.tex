== Episodio Psicótico Agudo

=== Notas clínicas

Un EPA no es un diagnóstico sino un cuadro clínico que determina un encare "de planteo". En caso de estar frente a un EPA polimorfo, se puede hacer el encare de PDA.

WARNING: EPA no es un diagnóstico nosológico.

Múltiples nombres (no siempre son sinónimos): Psicosis Delirante Aguda, Trastorno Psicótico Breve, Psicosis Reactiva Breve, Boufée Delirante, Psicosis Alucinatoria o Imaginativa Aguda, deliro d'emblée (de inicio súbito). En la CIE: Trastornos Psicóticos Agudos y Transitorios (F23).

=== Encare

==== En suma

Paciente con AFP de X, APM de X, APP de X que es traido por un cuadro clínico de inicio agudo, polimorfo, centrado en una alteración del pensamiento, en un contexto de (pupererio, bajo nivel intelectual, retardo mental, trastorno de la personalidad, UISP, etc.)

==== Agrupación sindromática

===== Síndrome delirante

De inicio brusco, de X evolución, dado por la presencia en el contenido del pensamiento de ideas mórbidas incompartibles, irreductibles a la lógica, carentes de juicio de realidad, que le generan conductas, por lo que las catalogamos como ideas delirantes. Este delirio es:

* Politemático: persecutorio, megalomaníaco, transformación corporal (sexual, envenenamiento, influencia, riqueza, potencia), elementos del Síndrome de influencia, automatismo mental, despersonalización. Destacar peligrosidad si corresponde.
* A mecanismo múltiple: intuitivo, interpretativo, alucinatorio.
** Intuitivo: inmediatez irreflexiva, se le presenta como verdad revelada ("de pronto me di cuenta que..."). Aceptado sin crítica ni razonamiento.
** Alucinatorio: AAV, visuales, cenestésicas, táticles, motoras (puede estar dentro de un SAM, siendo un elemento de mal pronóstico). Si hay alucinaciones olfativas plantear en DD causa orgánica (epilepsia temporal con crisis uncinadas).
** Imaginativo: fantástico, basado en ficciones.
** Interpretativo: por inferencias y deducciones erróneas a partir de un hecho real.
* Mal sistematizado: sus componentes no guardan una lógica. Presente movilidad, carácter cambiante, mínima organización. Carente de hilo argumental: expresado sin orden, coherencia y claridad. Sin marcado poder de convicción.
* Estructura: polimorfa.

Conductas generadas por el delirio.

TIP: Características del delirio: PAFaInVif

Características del delirio:

* Polimorfo: politemático, a mecanismo múltiple.
* Adhesión: el paciente no puede desprenderse del contenido del delirio.
* Fascinación: paciente fascinado por la experiencia.
* Inefabilidad: hay incapacidad de organizar el discurso delirante, no da cuenta del origen de la experiencia.
* Vivido: el delirio es más vivido y actuado que relatado (sensible y actual).
* Fluctuante, cambiante.

El delirio está constituido desde el inicio, sin trabajo elaborativo delirante (d'emblée), con intensificación parahípnica.

===== Síndrome de alteración del humor y la afectividad

Oscilante en forma solidaria con el delirio. Ansiedad MIDI, manifestada s/t a nivel de la psicomotricidad. Puede haber aceleración del pensamiento con taquipsiquia y verborea. La mímica y gestualidad pueden estar aumentadas, con fascies cambiante.

===== Síndrome conductual

Cuadro actual Auto/heteroagresividad, EPM, CB y pragmatismos, psicomotricidad (subsidiario al síndrome delirante ya analizado).

Curso de vida Uso/abuso de sustancias psicoactivas. Sº de consumo de sustancias (pauta que desconocemos).

===== Síndrome de alteración de la conciencia

Toda la persona está alterada en función del delirio. La conciencia se encuentra polarizada con déficit en la atención espontánea y voluntaria. Dificultad para ordenar la experiencia consciente en el presente. No presentifica la entrevista, carece de conciencia mórbida.

Memoria alterada con paramnesias (falsificaciones del acto mnésico, mezcla del pasado y presente, de lo real con lo imaginado, con falsos reconocimientos, ilusión de sosías), sin DOTE.

Hay una atmósfera de estado crepuscular de la conciencia (fascinación, "ser consciente en penumbras"). Es capaz de estar en el mundo compartible con OTE, pero con una disgregación y estrechamiento del campo de la conciencia, como hipnotizado, inmerso en el contenido patológico de la experiencia, sin poder salir de la misma. Actitud meditativa y de escucha que permite inferir la fascinación por la experiencia.

==== Personalidad y Nivel

icon:paste[] Nivel en diferido.

.Personalidad
Destacar todo lo que haya en la HC porque importa en el DD y en el pronóstico.

==== Diagnóstico positivo

===== Nosografía clásica

icon:paste[] Psicosis.

icon:paste[] Psicosis aguda

.Psicosis delirante aguda

Por: inicio brusco, sin prodromos, de un cuadro con predominio de lo delirante (sobre la alteración del humor y de la consciencia), con curso fluctuante y agravación parahípnica, el carácter intensamente vivenciado con fascinación e inefabilidad, polimorfismo dado por un delirio politemático, a mecanismo múltiple, cambiante.

===== DSM-IV

Trastorno psicótico breve (provisorio).

Trastorno esquizofreniforme (provisorio), ya que cumple los criterios de corte transversal para fase activa de esquizofrenia y en cuanto al corte longitudinal lleva menos de 6 meses de evolución, quedando sujeto el diagnóstico final a este criterio de duración, plazo en el cual deberá volver al nivel previo de funcionamiento.

===== CIE-10

F23: Trastornos psicóticos agudos y transitorios
F23.0: TPA polimorfo sin síntomas de esquizofrenia.
F23.1: TPA polimorfo con síntomas de esquizofrenia.
F23.2: TPA de tipo esquizofrénico.
F23.3: TPA con predominio de ideas delirantes.
F23.8: Otros TPA y transitorios.
F23.9: TPA y transitorio sin especificación.

==== Diagnósticos diferenciales

Según edad: AP consumo de drogas - Sintomatología acompañante cambiar orden.

En primer lugar, con otras psicosis de instalación aguda que se pueden presentar con delirio:

* Manía delirante: AP y AF afectivos. Comparten el debut temprano, la existencia de una desestructuración de la conciencia, pudiendo haber inquietud motora, verborrea e ideación megalomaníaca en ambas. Pero en la PDA predomina el trastorno delirante sobre la afectación del humor, siendo la afectividad cambiante, oscilante, congruente con la temática delirante. No existe actitud lúdica ni verdadera con fuga de ideas.
* Melacolía delirante (en caso de tener ideas con contenido depresivo). Lo descartamos por la ausencia de un síndrome depresivo. En la depresión suele haber un inicio más progresivo, centrado en el humor en menos, con IPM, DM y el delirio es TOMOPOADIR. En la PDA predomina el delirio pos sobre la alteración del humor.
* Causa orgánica o medicamentosa: descartaremos por la clínica y paraclínica, no existiendo datos en la historia (tiene más peso en un > 40 años, sin AF ni AP psiquiátricos).
** Tóxica: intoxicación, uso o abstinencia de estimulantes del SNC, alcohol, cocaína, anfetaminas, fenciclidina, alucinógenos, antidepresivos, corticoides, clonidina, otros medicamentos (isoniazida, AINEs, digitálicos, anticolinérgicos, L-Dopa, suspensión brusca de IMAO). Absinencia de OH, BZD.
** Endocrinológica: hipertiroidismo, Cushing.
** Metabólica: porfiria aguda, encefalopatía hepática, hipo / hipercalcemia. Enfermedad de Wilson.
** Nutricionales: pelagra, déficit de tiamina, déficit B12.
** Neurológica: tumores, TEC, hematoma subdural, epilepsia (crisis parciales complejas), esclerosis múltiple, corea (Huntington), vascular. Demencias (Alzheimer, Pick). Ictus.
** Infecciosa: meningitis, encefalopatía por HIV, encefalitis virales. Neurosífilis.
** Autoinmune: LES
* Confusión mental o Delirium. Comparten la dificultad para ordenar la experiencia consciente actual, los falsos reconocimientos. Alejado por la falta de estructura onírica en el delirio, ausencia de perplejidad y ausencia de causa orgánica clara. En la PDA predomina el delirio por sobre la alteración de conciencia.

TIP: Orientadores de organicidad: primer episodio con debut tardío, atipicidad, alucinaciones olfativas o visuales prominentes, evolución atípica.

Exacerbación de esquizofrenia paranoide: si corresponde a un 2º episodio de PDA, pese a reiterar episodios delirantes, no pensamos que se trate de una psicosis crónica por el período intercrítico libre de sintomatología y sin deterioro pragmático. Eventual DD con Trastorno Esquizoafectivo.

En caso de muchos elementos de mal pronóstico puede plantearse DD con inicio de Esquizofrenia.

Psicosis histérica: neurosis histérica descompensada con síntomas disociativos. Lo descarta la ausencia de una personalida histérica, falta de antecedentes de síntomas conversivos o disociativos, falta de desencadenante emocional, beneficio secundario, bella indiferencia y por la ausencia de conflicto insconsciente en juego. Alternativamente: trastorno de la personalidad con síntomas disociativos.

==== Diagnóstico etiopatogénico y psicopatológico

===== Etiopatogenia

Multifactorial: biológicos y psicosociales.

Importa destacar el factor terreno (s/t si hay AF AP de cuadros similares) que evoca un predisposición del sujeto, una fragilidad yoica con bajo umbral ara delirar sobre la cual inciden factores desencadenantes BPS.

En lo biológico: consumo de sustancias, en especial el consumo de marihuana es un factor de riesgo para el desarrollo de episodios psicóticos \footnote{Moore, T. H., Zammit, S., Lingford-Hughes, A., Barnes, T. R., Jones, P. B., Burke, M., \& Lewis, G. (2007). Cannabis use and risk of psychotic or affective mental health outcomes: a systematic review. The Lancet, 370(9584), 319-328.} , abandono de medicación.

En lo psicosocial: medio familiar, pérdidas o estresantes.

===== Psicopatología

Psicoanálisis: los sucesos estresantes provocan gran angustia que es proyectada como un mecanismo de defensa en el delirio, siendo el mecanismo de defensa una negación psicótica de la realidad.

Para Jaspers, esta experiencia delirante primaria se constituye a medida que el campo de la conciencia se desorganiza, llegando en profundidad a medio camino del ensueño, viviendo la experiencia delirante y alucinatoria como la proyección del inconsciente hacia el mundo exterior.

Según la TOD de Ey, corresponde a una desestructuración de conciencia de 2° grado o conciencia oniroide, con ósmosis de los espacios vitales/vivenciales (realidad externa e interna), en la cual la conciencia se hace suficientemente imaginativa como para que instale secundariamente la experiencia delirante y alucinatoria como una proyección del inconsciente.

La vivencia delirante se constituye a medida que el campo de la conciencia se desorganiza.

==== Paraclínica

El diagnóstico es clínico. Realizaremos exámenes para: descartar diagnósticos diferenciales (s/t lo orgánico), con vistas al tratamiento, de valoración general). Se solicitarán estudios desde un triple punto de vista: biológico, psicológico y social.

===== Biológico

Examen físico completo, con énfasis en lo neurológico. Consulta con internista. Buscaremos elementos para descartar causas orgánicas reversibles del cuadro (HTEC, estigmas de UISP, síntomas neurológicos focales y de irritación meníngea.

Rutinas: hemograma, glicemia, función renal, orina, ionograma, funcional y enzimograma hepático (ecefalopatía hepática y por uso de fármacos de metabolización hepática).

En mujer en edad genital activa: test de embarazo.

Si hay elementos clínicos que lo ameriten: TAC / RNM. Sabiendo que no se recomienda la realización de TAC o RMN de rutina en un primer episodio de psicosis, excepto que exista algún otro elemento de sospecha \footnote{Albon, E., Tsourapas, A., Frew, E., Davenport, C., Oyebode, F., Bayliss, S., ... \& Meads, C. (2008). Structural neuroimaging in psychosis: a systematic review and economic evaluation.} \footnote{Khandanpour, N., Hoggard, N., \& Connolly, D. J. A. (2013). The role of MRI and CT of the brain in first episodes of psychosis. Clinical radiology, 68(3), 245-250.}.

Para descartar diagnósticos diferenciales:

. monitorización de fármacos y drogas en sangre y orina.
. HIV (encefalopatía por HIV), VDRL (neurosífilis).  Si la situaciuón lo amerita: HVB, HVC.
. Función tiroidea.
. Según la clínica: EEG con deprivación de sueño y registro prolongado.

Para descartar contraindicaciones ante eventual tratamiento con ECT: consulta con cardiólogo, ECG, RxTx, examen neurológico y Fondo de ojo.

===== Psicológico

Luego de superado el cuadro actual. Tests de personalidad proyectivos y no proyectivos, tests de nivel (Bender, Weschler). Procurando conocer la conflictividad del paciente así como sus aspectos más sanos, mecanismos de defensa, integridad de la organización del pensamiento y manejo de la agresividad y angustia, para un abordaje terapéutico eventual.

===== Social

Entrevistas con terceros para ampliar información, inventario de eventos vitales, analizar incidencia el medio en la patología, valoración de la red de soporte social, Interesa investigar el nivel de funcionamiento previo y la eventual existencia de un corte existencial. Explicaremos las medidas terapéuticas a realizar, riesgos y beneficios de la ECT, obteniendo el consentimiento informado por escrito por parte de familiares. Datos de internaciones anteriores, tratamiento instituido y respuesta al mismo.

==== Tratamiento

El tratamiento será dinámico, adaptado a la evolución clínica, realizado por equipo multidisciplinario.

Internaremos en Hospital Psiquiátrico, dado el intenso estado delirante, alucinatorio, del paciente y la inestabilidad psíquica que esto implica, que puede llevar a conductas auto o heteroagresivas con consecuencias médico-legales.

Lo ideal es una sala individual, sin elementos de riesgo (ventanas, espejos), con asistencia de enfermería especializada las 24 horas y acompañante continentador a permanencia.

De esta forma lograremos:

. continentar al paciente calmando su sufrimiento psíquico
. tratar el delirio de forma rápida y eficaz
. acortar la duración del episodio actual, mejorando el pronóstico
. ajustar la medicación de forma rápida según la evolución del cuadro
. proteger al paciente y terceros de las posibles complicaciones médico-legales
. vigilar fuga y conductas de riesgo / autoeliminación
. realizar la paraclínica necesaria para descartar diferenciales

Realizaremos estrictos controles clínicos y monitoreo del tratamiento.

===== Biológico

Haloperidol, NL incisivo, con efecto antidelirante, del grupo de las Butirofenonas, 5 mg i/m H8 y H20, que regularemos según respuesta clínica y tolerancia (pudiendo agregar otros 5 mg H14 i/m de ser necesario). Controlaremos la aparición de efectos secundarios tipo extrapiramidal (temblor, rigidez, rueda dentada, bradipsiquia). Si aparecen concentraremos las dosis en la noche (ya que éstos no aparecen durante el sueño).
Actúa bloqueando los receptores dopaminérgicos D2 cortico-meso-límbicos.

TIP: Deben vigilarse efectos extrapiramidales (en especial distonías agudas) en pacientes varones, jóvenes. En caso se puede plantear asociar antiparkinsonianos de entrada. Similares consideraciones en caso de AF de enfermedad de Parkinson o de AP de reacciones extrapiramidales.

* Si no lo controlamos de éste modo, agregaremos un antiparkinsoniano de síntesis como el Biperideno 2 mg v/o H8 H14. Si hay distonías agudas: Biperideno 2 mg i/m c/8 hs que en 2-3 días se pasa a v/o.

Pasaremos la medicación a v/o si a los 5-7 día obtenemos mejoría.

Si no hay mejoría, agregaremos otros 5 mg i/m H14 de Haloperidol.

.Falta de respuesta
Si a los 10-14 días no hubo mejoría clínica/sintomática significativa en la actividad delirante y/o alucinatoria y persiste la dificultad en el contacto con la realidad indicaremos ECT a realizar por anestesista, con paciente en ayunas, 1 sesión cada día por medio con oxigenoterapia, monitoreo ECG y EEG, con anestesia a determinar por anestesista y curarizantes como la succinilcolina, con colocación de electrodos bitempora. Controlaremos la duración de la convulsión. Regularemos la cantidad de sesiones según respuesta, planteando inicialmente entre 8-12 sesiones para lograr el efecto deseado. Vigilaremos al paciente luego de cada sesión sabiendo que pueden existir cefaleas y trastornos mnésicos de breve duración. Debemos contar previamente con consentimiento informado firmado por familiar responsable.

La ECT puede ser de primera elección en caso de riesgo vital (rechazo de alimentos, mal estado general, contraindicaciones de antipsicóticos).

.Ansiedad
Lorazepam 1 amp i/m cada 6-8 horas, pasando luego a vía oral. Segunda línea: Levomepromazina (NL sedativo) 25 mg i/m cada 8 horas. En este caso estaremos atentos a los efectos secundarios: sedación, hipotensión postural, efectos anticolinérgicos).

.Insomnio
Flunitrazepam 2m 1c v/o noche o, de requerir IM, Midazolam 1 amp im.


===== Psicológico

Entrevistas diarias para:

* promover alianza terapéutica
* configurando un marco continentador y de apoyo
* evaluando si hay la crítica del delirio.
* investigando y reforzando aspectos sanos
* evaluando facto desencadenante y estresores ambientales

Valorar la posibilidad de psicoterapia una vez superado el cuadro actual, supeditado a paraclínica.

===== Social

Visitas a discreción, personas más aptas

Información a familia de la enfermedad y del pronóstico, jerarquizando cumplimiento de la medicación. Buscar alianza terapéutica entre la familia y el equipo tratante.

Medidas psicoterapéuticas para disminuir el estrés familiar que propicia recaídas.

.Alta

Se efectuará una vez logrado:

* Remisión total o considerable de la sintomatología delirante
* Aparición de crítica
* Normalización de las CB, la afectividad y el autocuidado
* Ausencia de ideación suicida


Una vez lograda la remisión otorgaremos el alta hospitalaria con Haloperidol 5 mg v/o H8 y H20 (con la dosis con que se obtuvo mejoría) (retorno al hogar como factor de estrés). Biperideno según lo mencionado antes. Eventualmente medicación sedativa para lo que preferimos una benzodiacepina de vida media larga.

Realizaremos controles en policlínica seriados, que iremos espaciando hasta llegar a un control mensual. Mantendremos las dosis de Haloperidol que según la evolución iremos disminuyendo lentamente mes a mes (según historia) luego de 6 meses-1 año, hasta lograr la dosis mínima eficaz.

En caso de perfil de bajo cumplimiento indicaremos un NL de depósito tal como Decanoato de Haloperidol i/m cada 21 días, sabiendo que 100 mg i/m de NLD corresponden a 5 mg v/o (10 v/o = 150 mg HD; 15 v/o = 200 mg HD). Segunda línea: Palmitato de Pipotiazina cada 4 semanas).

==== Evolución y pronóstico

Pensamos obtener la remisión del cuadro actual con el tratamiento instituido. El pronóstico dependerá de la adhesión al tratamiento y controles pautados.

* PVI: sujeto a riesgos vitales que impliquen sus conductas delirantes. Posibilidad de instalación de depresión postpsicótica.
* PPI y PPA: puede ser variable.

Evolución:

50% evolucionan favorablemente 50% restante:

* intermitente con repetición de episodios similares
* evolución a cuadros afectivos
* evolución a psicosis crónica tipo esquizofrenia

Este paciente presenta elementos de buen/mal pronóstico:

Buen pronóstico:
\begin{itemize}
\item profunda alteración de la conciencia. Gran desestructuración (cuanto más confuso mejor pronóstico)
\item brusquedad del inicio delirio
\item breve duración de las crisis
\item polimorfismo
\item buena respuesta al tratamiento
\item trastornos del humor
\item intensamente vivenciado
\item AP de cuadro similares breves con buena respuesta
\item reactividad del cuadro
\item AP de RAP grupo B, sobre todo histriónicos (dramatización, teatralidad).
\item riqueza imaginativa
\item alteración de CB
\end{itemize}
Mal pronóstico:

* automatismo mental importante
* presencia de elementos de SDD
* sistematización del delirio
* duración de las crisis
* elementos centrados en la corporeidad / hipocondríacos
* resistencia a la terapéutica o abandonos de tratamientos
* AP de RAP grupo A (s/t esquizoide)
* aplanamiento afectivo
* AF de psicosis crónica
* persistencia de estresores ambientales / mala continentación socio-familiar

===== Notas psicosis puerperal

En caso de psicosis puerperal:

* riesgo inicial 1/500 primíparas
* en lo subsiguientes partos: 1/3

Depresión puerperal no psicótica = 10-15% de primíparas. Recurrencia de 50% en mujeres sin AP y de 100% en mujeres con AP.

Etiología:

* hormonal
* factores psicosociales: estrés, cambios vitales por emabrazo (matrimonio, roles). Psicoanálisis: pérdida narcisita del yo independiente.

Predisponentes:En las primíparas y pacientes con AP o AF de trastornos del estado de ánimo o episodios previos de depresión o psicosis postparto, se incrementa el riesgo.
Recurrencia elevada: psicosis 1/3, depresión 1/2.

==== En suma
Hemos visto un paciente de sexo X, de X años, procedente de MSEC X, con AF de X, APM de X, APP de X, que consulta por X, en quien diagnosticamos X, reconociendo como desencadenantes X, planteando diagnósticos diferenciales con X, que hemos estudiado con X, realizado un tratamiento con X, planteando un pronóstico X.

==== Bibliografía
