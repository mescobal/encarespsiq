== F98 Otros trastornos de inicio en la niñez y adolescencia

=== F98.5 Tartamudez

Definición: alteración en la fluencia y/o patrón temporal del lenguaje que es inapropiado para la edad del individuo y persistente en el tiempo. Puede estar acompañado de fenómenos motores.
Otros nombres: trastorno de la fluencia de inicio en la infancia.
Edad de inicio: 2 a 7 años. A los 16: 65-85% se recuperan.

==== Etiología
Desconocida. Conceptualización actual: dominancia incompleta de los centros primarios del lenguaje de etiología multifactorial.
Genética: da cuenta del 50-80% de los casos. Correlación entre hermanos: 19%. Riesgo x 3 en familiares de primer grado. Asociación con genes dopaminérgicos SLC6A3 y DRD2,
Comparte características con S de la Tourette: inicio en la infancia, hombres 4:1, curso fluctuante, empeoran con ansiedad, están asociados a tics, tienen anomalías cerebrales en los ganglios basales, empeoran pora agonistas dopaminérgicos, mejoran con antagonistas.
Infecciosa: algunos casos asociados a PANDAS (trastornos autoinmunes pediátricos asociados a infecciones estreptocóccicas), que están vinculados a Tourette y TOC.
Casos adquiridos en adultos: asociados a medicación y TEC.
