== Parafrenia

=== Notas clínicas

=== Encares

==== Agrupación sindromática

===== Síndrome delirante

* Temática: fantástica, megalomaníaca, persecutoria.
* Mecanismo: imaginativo, alucinatorio, intuitivo.
* Sistematización pobre (no existe tema central, fantasía a temas múltiples renovados).

Edad de comienzo: 30-45 años, con conservación de facultades fuera del delirio. Bipolaridad: vive en la realidad, a la que permanece adaptado, el delirio no sustituye a la realidad sino que se añade a ella. Yuxtaposición de la realidad con integridad de funciones intelectuales y de la afectividad (fuera del tema delirante) conservando la actividad delirante y el comportamiento social (concepto de Ey). Evolución sin déficit. A destacar: amplitud "cósmica" de los temas, pensamiento mágico, mitología, aspecto estético-poético. Falsos recuerdos, fantasía barroca, mística, autorreferencial. El mecanismo imaginativo ahoga lo alucinatorio.

===== Síndrome de automatismo mental

Más pensado que vivido.

===== Síndrome de influencia

===== Síndrome de alteración del humor

==== Diagnóstico positivo

===== Nosografía clásica

.Psicosis.
.Psicosis crónica.
.Parafrenia
Por las características del síndrome delirante analizado, que es fantástico, a mecanismo alucinatorio predominantemente, con bipolaridad característica.
.Forma clínica:
A. Sistemática (Psicosis Alucinatoria Crónica) Inicio 30-40 años, sobre todo en hombres Delirio a temática persecutoria o de influencia, mecanismo alucinatorio (cenestésico, auditivo), SAM. Alucinaciones: son constantes y necesarias para el diagnóstico, ricas, múltiples, a predominio auditivo y cenestésico. Auditivas: "malignas" o amenazantes (burlas, críticas), sirven a veces de canal al síndrome de Influencia. Cenestésicas: desagradables y dolorosas, pueden dar un aspecto hipocondríaco. Olfativas y gustativas: malos olores, malos gustos, integradas al delirio, se imponen del exterior con una finalidad persecutoria. Delirio PAC: alucinatorio, tema de persecución, influencia, posesión, pobre sistematización, vivencia pasiva sin desorden grave del comportamiento. Puede reaccionar con denuncias o agresiones. Evolución lenta, momentos fecundos, "enquistamiento".
B. Parafrenia fantástica Inicio: antes de los 30 años. Delirio: a temática megalomaníaca, desbocamiento hacia lo irreal, monstruoso y absurdo, ideas de "persecución colosal". Mecanismo: alucinatorio plurisensorial, imaginativo e intuitivo. Mal sistematizado. Evolución a bajo ruido los primeros años, luego aminora y se hace estereotipado. Ocasional evolución al deterioro.
C. Parafrenia expansiva Inicio: 30-50 años. Casi exclusivamente en mujeres. Delirio: temática de grandeza (religioso, profético, erótico), mecanismo alucinatorio (s/t visual), poco sistematizado. Es la forma de mayor repercusión en pragmatismos. Trastornos del humor: exaltación, episodios delirantes fecundos. Evolución: esquizofrenia, hipomanía crónica, se fija y esclerosa el delirio.
D. Parafrenia confabulante (rara) Inicio: 20-40 años. Delirio: temática de tipo "novela megalomaníaca" con fabulación repentina. Mecanismo imaginativo (no alucinatorio). Vaga coherencia, conserva la personalidad social. Evolución: palidece el delirio, no va a la disociación.

.Estado
Descompensada: por aumento del monto delirante (momento fecundo).

==== Diagnóstico diferencial

. Con otras psicosis crónicas:
.. Paranoia: no predomina lo alucinatorio, delirio sistematizado, rasgos de personalidad paranoicos, calor afectivo.
.. Esquizofrenia: tiene SDD, autismo, ambivalencia, aplanamiento afectivo
. Psicosis agudas: PDA, EPA, manía, confusión, melancolía delirante.

==== Diagnóstico etiopatogénico-psicopatológico

Etiopatogenia

Biológico

neurotransmisores (basado en eficacia de antipsicóticos).

Psicosocial

Psicopatología

Proceso con estructura positiva y negativa. A partir de la experiencia delirante primordial se edifica la estructura delirante. Estructura negativa: proceso psíquico que implica modificación de la personalidad con desorganización estructural que la vuelve mágica e impermeable a la experiencia (condición necesaria para la instalación de los elementos positivos). Estructura positiva: lo fantástico, el delirio y alucinaciones. Psicodinámico: conflicto entre las exigencias pulsionales y el Yo. Solo es capaz de generar psicosis instalándose sobre elementos negativos. Psicoanálisis: dificultad para separar el Yo del no-Yo. Yo frágil: falla del mecanismo de represión predominando el mecanismo de negación de la realidad, proyección, identificación proyectiva, idealización. Exuberancia del inconsciente que lleva a disgregación del Yo. Manifiesta a la vez la fuerza de la pulsión inconsciente y el control del Yo que pone en juego mecanismos de defensa psicóticos. El delirio expresaría simbólicamente las exigencias pulsionales inconscientes.

Paraclínica

Igual que en esquizofrenia.

==== Tratamiento

Enfatizar pragmatismos conservados:

* rescatar núcleos más sanos
* no se pretende yugular completamente el delirio
* minimizar la interferencia del delirio con la vida cotidiana

==== Evolución y pronóstico

Destacar la posibilidad de evolución a otras psicosis crónicas. Frecuentemente evolucionan a la disociación -> evolución a la esquizofrenia.

Nota: Parafrenia es un diagnóstico discutido por muchos clínicos. Existencia fundamentada sobre todo por Ey y nosografías europeas, NO está contemplado en sistemas clasificatorios actuales. Muchos clínicos opinan que corresponden a esquizofrenias con desarrollo tardío de síntomas negativos. Al ser un concepto más antiguo, no hay investigaciones actuales al respecto.
