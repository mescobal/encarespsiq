\chapter{Trastorno facticio}
Producción intencionada o fingimiento de síntomas o incapacidades somáticas o psicológicas

Sinónimos: Síndrome de Münchausen, paciente peregrinante, visitador de hospitales, "hospitalismo".

Excluye: simulación, Münchausen por poderes.
\section*{Encare}

\subsection*{Diagnóstico positivo}
.CIE-10
- Patrón persistente de producción intencionada o simulación de síntomas y/o daño autoinfligido para producirse los síntomas.
- Sin evidencia de motivación externa (económica, evitación de daño). Excluye simulación.
- Excluye otro trastorno que pueda explicar mejor los síntomas.

.DSM-IV
- Fingimiento o producción intencionada de signos o síntomas físicos o psicológicos
- El sujeto busca asumir el rol de enfermo
- Ausencia de incentivos externos (económico, legal)

Especificadores:

- Con predominio de síntomas psicológicos
- Con predominio de síntomas físicos
- Con ambos

\subsection*{Diagnóstico diferencial}

- Enfermedad física verdadera
- Simulación
- Trastorno facticio por poderes
- Trastornos somatomorfos
