\chapter{Manía}
\section*{Notas clínicas}

.Afecto
Expresiones de vida media corta reflejas de contingencias emocionales momentáneas.

.Humor
Emociones sostenidas sentidas interiormente

.Humor patológico
Desproporcionado al estresor o a la situación concurrente, no responde a continentación, mantenido por semanas / meses / años, juicio alterado por severa influencia del humor. Endoreactivo: demasiado intenso, disregulación permanente dada por (1) la facilidad con que se alcanza un intenso estado de humor ante un estrés y (2) persiste autónomamente cuando el estresor cesa.

\subsection*{Tratamiento}
\subsubsection*{Conductas a evitar}
\begin{itemize}
	\item En depresión: Monoterapia con antidepresivos, litio, valproato, aripiprazol, ziprasidona, donepecilo, paroxetina (excepto por ansiedad comórbida), gabapentina. No usar combinaciones: lamotrigina + acido fólico, litio + paliperidona, valproato + paliperidona / ziprasidona / gabapentina / topiramato. No usar adjunto de ziprasidona, armodafinil. Datos insuficientes: gabapentina, adjuntar AAS / Celecoxib / Levetiracetam.
	\item En manía: evitar tratamiento en monoterapia con donepecilo, gabapentina, lamotrigina, nifedipina, paliperidona, risperidona, topiramato, verapamil, ziprasidona, alopurinol, omega3. Datos insuficientes: oxcabamazepina, olanzapina, risperidona + carbamazepina.
	\item En mantenimiento: evitar aripiprazol + lamotrigina, memantina, pramipexol, verapamil, ADT. Evitar combinación con ADT, estabilizador + memantina. Datos insuficientes: lamotrigina, asenapina, gabapentina, topiramato.
	\item Otros:
	\begin{itemize}
  		\item Divalproato en mujeres en edad genital activa.
		\item Carbamazepina durante embarazo
	\end{itemize}
\end{itemize}
\section*{Encare}
Motivos de consulta: traído por terceros (trastorno de conductas basales o complejas y pragmatismos), complicación medicolegal, abandono de medicación.
\subsection*{En suma}
Paciente de sexo X, con AF de trastornos afectivos o sus equivalentes (alcoholismo), con AP de episodios de humor en más (y/o en menos), con tratamientos que (mantuvo / abandonó), con (buena/mala) adaptación pragmática interepisódica, que es traído (por terceros, en general no por cuenta propia) a la consulta por alteraciones de conducta / alteración de CB / pragmatismos con un cuadro centrado en el humor que pasamos a analizar.
\subsection*{Agrupación sindromática}
\subsubsection*{Síndrome de exaltación del humor}
Centra el cuadro clínico, de X tiempo de evolución (en general desarrollo rápido en 1-2 semanas), que se presenta luego de icon:paperclip[] (60\% tienen como antecedente algún estresor psicosocial), en el que se aprecia una aceleración de los procesos psíquicos, evidenciado por los elemento que se detallan.
\paragraph{Humor}
Central en el cuadro. Humor predominantemente expansivo y exaltado, que pareciera optimista, pero de tonalidad inestable y oscilante y que puede llegar a colérico, eufórico e irritable, con una tendencia a dominar el encuentro y ridiculizar al entrevistador, con elementos eróticos, con una afectividad que se presenta lábil, con oscilaciones, llantos y risas. Puede presentar un síndrome de ansiedad que reconocemos como parte integrante de este cuadro del humor (puede describirse aparte si la magnitud lo permite).
\paragraph{Presentación}
Un acortamiento del espacio interpersonal, confianza excesiva, el paciente es invasivo, nos tutea, nos toca, nos interrumpe, usa un tono de voz alta, tiende a dominar la entrevista, está inquieto, alegre, canta, baila, grita, desinhibido, se desnuda, porte desprolijo o desaliñado, facies animado, alegre o furioso, habla sin cesar, está en continuo movimiento y parece no cansarse.
\paragraph{Mímica y gestualidad}
Aumentadas, congruentes con las oscilaciones del humor.
\paragraph{Actitud}
Inadecuada, lúdica, expansiva.
\paragraph{Cognición}
Se distrae fácilmente por estímulos periféricos, tiene dificultad para fijar la atención voluntaria y espontánea en algo en particular (hipoprosexia), con alteraciones de la memoria reciente, con dificultades para reconstruir el pasado reciente, pero con una exaltación de la memoria lejana. En general está BOTE (puede no estar por falta de atención). Carencia de insight. Sin conciencia de morbidez.
\paragraph{Pensamiento}
Forma: lenguaje con logorrea, un flujo del habla incontrolable, excesivo, que invade y domina la conversación. Un curso acelerado, con verborrea, configurando una taquipsiquia, acompañada de una fuga de ideas, pues el paciente comienza un relato que no concluye para pasar de inmediato a otro diferente mediante la asociación por asonancia, es decir una asociación superficial y efímera de ideas que se presenta por lo tanto como un pensamiento no finalista. La taquipsiquia y la fuga de ideas (o fuga de temas) dan cuenta de la exaltación del ser psíquico en su totalidad.

Contenido: reconocemos ideas sobrevaloradas, exageradas, condicionadas afectivamente, comprensibles en la vida del paciente, pero no compartibles, con calor afectivo, de las cuales puede no tener una verdadera convicción, son fugaces, admiten críticas, enmarcadas en un entorno lúdico, parecen delirantes pero tienen menor intensidad. Centradas en grandiosidad, omnipotencia e hiperoptimismo. 

\faStickyNote Si las ideas exceden lo sobrevalorado \faArrowCircleRight Síndrome delirante.
\paragraph{Psicomotricidad}
Se presenta exaltada, no puede estar quieto, con necesidad imperiosa de actividad, volcado al ambiente, camina, cambia las cosas de lugar, canta, baila, se sienta, se levanta, con un actitud lúdica, pero que es improductiva, estéril y sin cansancio físico proporcional, con menor necesidad de dormir. Llega a realizar estos actos sin finalidad, desordenados. Toma elementos del ambiente.
\paragraph{Conductual}
En las conductas basales destacamos adelgazamiento, disminución de la necesidad de dormir, anorexia o hiperorexia con polidipsia y polifagia (pueden ser prodromos). En los pragmatismos destacamos la esterilidad e la hiperactividad con deterioro laboral, social y familiar (pobre juicio social), hipersexualidad con riesgo de ETS, llamadas excesivas inadecuadas. En las conductas complejas: agresividad, derroche, malos negocios, prodigalidad, compromiso excesivo en actividades placenteras, consumo de sustancias/OH (dentro y fuera del episodio actual), juegos de azar, casamiento impulsivo, actos con repercusión medicolegal.
\subsubsection*{Síndrome delirante / síndrome de alteración del pensamiento}
En ocasiones las ideas dejan de ser sobrevaloradas para ser delirantes. En este caso son ideas mórbidas, permanentes, incompartibles, irreductibles a la lógica, con defensa con convicción y con generación de conductas.
\paragraph{Formal}
Curso acelerado, asociación de ideas rápida y superficial, asociaciones frágiles, flujo continuo de frases rápidas, mal hilvanadas. Salta de un tema a otro (idea a otra) constituyendo una taquipsiquia con fuga de ideas que traduce la exaltación del ser psíquico en su totalidad. Esta hiperactividad de los procesos psíquicos leva a una incapacidad de fijarse en una idea concreta con fijación a estímulos irrelevantes. Esta "volatilidad" es responsable de trastornos en atención (voluntaria y espontánea), memoria y orientación.
\paragraph{Contenido}
* Temática: megalomaníaca (capacidades especiales, identidad grandiosa, riqueza, misión especial), mística, filiatoria, erótica, persecutoria, de envidia.
* Mecanismo: principalmente intuitivo, pero puede ser imaginativo, interpretativo, alucinatorio.
* Sistematización: mala sistematización: con escaso orden, coherencia y claridad.
* Conductas: le genera conductas (compras, gastos, sexo, violencia, robos, consumo de sustancias, alcohol). Se configura un delirio móvil, cambiante, desorganizado.
* Congruencia: puede ser congruente o incongruente con el estado de ánimo expansivo.
\subsubsection*{Síndrome de ansiedad-angustia}
Ansiedad masiva, invasiva, desestructurante e incompartible, por lo que la catalogamos como ansiedad psicótica.
\subsubsection*{Síndrome de alteración de la conciencia}
Evidenciado por la imposibilidad del paciente de adaptar el campo fenomenológico de la conciencia al momento presente, lo que configura para la Teoría Organodinámica de Ey una desestructuración de conciencia de primer grado o nivel ético-temporal.

Carece de conciencia mórbida. No presentifica el encuentro con el médico psiquiatra. Dificultad para reconstruir el pasado inmediato con alteración de la atención espontánea y voluntaria, dejándose llevar por estímulos ambientales.
\subsubsection*{Síndrome somático}
Evidenciado por las repercusiones de los cambios de apetito y sueño, así como de las conductas (consumo de sustancias, exposición a clima adverso). Destacar elementos metabólicos (adelgazamiento, deshidratación, hipertermina), neurológicos y endocrinológicos relevantes.
\subsubsection*{Síndrome depresivo}
Puede darse en retrospectiva (ver encare correspondiente) o bien con elementos depresivos dentro del cuadro actual (en ese caso, puede que haya que plantear episodio mixto).
\subsection*{Personalidad y nivel}
\subsubsection*{Nivel}
\faPaste: Nivel en diferido.
\subsubsection*{Personalidad}
Lo evaluaremos luego de remitido el cuadro actual.
\subsection*{Diagnóstico positivo}
\subsubsection*{Nosografía clásica}
\faPaste: Psicosis.
\faPaste: Psicosis aguda.
\paragraph{Crisis de manía}
Por presentar una exaltación del humor como elemento central del cuadro clínico, del que se destaca la fuga de ideas, la exaltación psicomotriz con actitud lúdica e hiperactividad desordenada (y en caso de presentarlas destacar las ideas deliroides o delirantes), que ha repercutido negativamente en los desempeños y funcionamiento vital.
\paragraph{Forma clínica}
Las diferentes formas clínicas son un continuo dinámico, según fluctuaciones de superficialización o pofundización de la alteración de conciencia.
\begin{itemize}
	\item Manía franca (simple o aguda): es el cuadro clásico. Carece de delirio y alucinaciones. Pero como la imaginación está exaltada puede darse, en las formas con más desestructuración de la conciencia, ideas de convicción subdelirantes o fabulatorias, como un delirio en estado naciente que no se consolida dada la gran hiperactividad y aceleración de los procesos mentales.
	\item Manía delirante: al desestructurarse la conciencia más profundamente se pasa a esta forma clínica. En ella se reconoce un delirio, una "experiencia delirante", aunque no de las características de la "experiencia delirante primaria" de la PDA. Este es cambiante, móvil, mínimamente sistematizado. Es un delirio verbalizado más que vivido (como en las PDA).
	\item Furor maníaco: es el grado máximo de exaltación psicomotriz. Se presenta como una exaltación de la expresión principalmente motora, con oscurecimiento de la conciencia. Puede haber rechazo del alimento y signos orgánicos graves de agotamiento, deshidratación e hipertermia.
	\item Estados mixtos: en todo episodio coexisten elementos maníacos y depresivos, pero en ocasiones esta mezcla es un rasgo principal del cuadro.
	\item Hipomanía: caracterizada por la fuga de temas (la idea llega a desarrollarse o formar un tema), un humor jovial, eufórico, hiperactivo, con múltiples iniciativas y proyectos que no llegan a finalizar, prodigalidad, hipersexualidad. Menor deterioro funcional.
	\item Manía confusa: desorientación TE, alteraciones mnésicas, trastornos del pensamiento.
\end{itemize}
\paragraph{Diagnóstico nosológico}
Este acceso maníaco se inscribe en una enfermedad crónica, de manifestación episódica: Psicosis Maníaco-Depresiva. Esta se define por la presencia de uno o más episodios de manía, generalmente acompañado por uno o más episodios depresivos, en el contexto de antecedentes personales y familiares destacados.
\subsubsection*{Según sistemas de clasificación (DSM IV)}
\paragraph{Diagnóstico del episodio}
Episodio maníaco

A. Período diferenciado de estado de ánimo anormalmente y persistentemente elevado, expansivo o irritable, de al menos 1 semana de duración (o cualquier duración si se hospitaliza).
B. Al menos 3 de estos síntomas: megalomanía, insomnio, verborrea, fuga de ideas, distraibilidad, hiperactividad, humor lúdico.
C. Malestar o deterioro clínicamente significativo
D. No cumple criterios para episodio mixto
E. Se excluye cuadro provocado por sustancias, enfermedad médica, tratamiento farmacológico, trastorno por déficit atencional con hiperactividad, EDM con irritabilidad.

Especificadores del episodio

* Gravedad: leve, moderado, grave, con/sin síntomas psicóticos.
* Curso: en curso, remisión parcial / total.
* Síntomas psicóticos: congruentes / no congruentes con el estado de ánimo.
* Síntomas catatónicos.
* Inicio: inicio en postparto.

Episodio Hipomaníaco Requiere: A y B: igual que manía, pero duración de al menos 4 días +

A. Igual que manía
B. Igual que manía
C. Cambio con respecto a humor habitual +
D. Cambio observable por terceros +
E. Sin alteración importante de pragmatismos +
F. Descartar sustancias, medicamentos, enfermedad médica.

Episodio Mixto Requiere:

A. Se cumplen criterios para episodio maníaco y para episodio depresivo mayor casi cada día x 1 período mayor a 1 semana +
B. Alteración de pragmatismos +
C. Descartar sustancias, enfermedad médica.

Episodio Depresivo Ver encares de depresión

\paragraph{Diagnóstico nosológico}

Trastorno Bipolar I

• Requiere: al menos 1 episodio maníaco o mixto (previo o actual).
• Especificar: último episodio + especificadores del último episodio.
• Especificadores de curso longitudinal: recuperación interepisódica (si/no), patrón estacional (si/no), ciclos rápidos (si/no).

Tipos:

• TB I episodio maníaco único
• TB I episodio más reciente X (hipomaníaco, maníaco, mixto, depresivo)

Trastorno Bipolar II

• Requiere: al menos 1 episodio hipomaníaco + historia de uno o más episodios depresivos (SIN historia de episodios maníacos o mixtos).
• Especificar: último episodio + especificadores de curso longitudinal.
• Trastorno ciclotímico
• Requiere:

A. historia de 2 años de varios episodios hipomaníacos + episodios depresivos que no cumplen criterios de EDM +
B. Nunca asintomático x más de 2 meses +
C. Dos primeros años sin EDM, episodio maníaco o mixto (si aparecen luego de los 2 años, codifican los 2 trastornos) +
D. Descartar esquizoafectivo, esquizofrenia, esquizofreniforme, trastorno delirante +
E. Descartar sustancias, enfermedad médica +
F. Alteración de pragmatismos.

Especificadores del trastorno

Curso:
. ciclos rápidos (al menos 4 episodios en 12 meses, 15-20% de los bipolares)
. con/sin patrón estacional
. con/sin recuperación interepisódica total.

Promotores del ciclado:

* Hipotiroidismo subclínico
* Sustancias/alcohol
* Alteraciones del ciclo sueño/vigilia
* Fármacos: antidepresivos, corticoides
* Lesiones cerebrales

\subsection*{Diagnóstico diferencial}
\subsubsection*{Del episodio}
Con otras psicosis agudas

. Manía secundaria a causa orgánica: si el cuadro se presenta a edad tardía, con trastorno de conciencia, desorientación, UISP, primer episodio, elementos atípicos:
.. Causa: tóxica: anfetaminas, cocaína, alcohol, intoxicación o abstinencia.
.. Fármacos: antidepresivos, corticoides.
.. Endocrinológicas: hipertiroidismo, Cushing, encefalopatía hepática.
.. Neurológica: epilepsia parcial compleja, esclerosis múltiple, corea, tumores, TEC.
.. Infeccionsa: neurosífilis, HIV
.. Metabólica
.. Neoplasias: páncreas, pulmón
.. Autoinmune.
. PDA / Trastorno psicótico breve: tienen cosas en común (episodio agudo, desestructuración de conciencia y afectos, delirio, experiencia sensible y actual), pero con diferencias (predominio del humor exaltado y de la fuga de ideas, con actitud lúdica, ideas delirantes secundarias al trastorno del ánimo, AF, AP), le falta elementos (delirio polimorfo, alteración de conciencia más profunda, de tipo oniroide).
. Confusión mental: tiene cosas en común (desestructuración de conciencia). En contra: menor profundidad de la desestructuración, delirio onírico en la confusión, falta de perplejidad, conservación de la orientación temporoespacial.

No pensamos que este cuadro sea icon:paperclip[], por los AF, los AP de episodios maníacos y melancólicos y por el abandono de medicación determinando cuadros similares. Por la paraclínica descartaremos algunas de estas causas.
\subsubsection*{Del trastorno}
.Con psicosis crónicas

En el joven se pueden ver debut clínico de Esquizofrenia o de un Trastorno Esquizoafectivo con un episodio maníaco. A factor: deterioro en el curso evolutivo, el hipopragmatismo o el corte existencial, el delirio incongruente con el estado de ánimo. En contra: prima la alteración del humor por sobre la del pensamiento, no elementos del Síndrome Disociativo-Discordante, por los AF y los AP.

.Demencias

En el paciente añoso se puede ver el debut clínico de un cuadro Demencial, alejándonos los AF y AP, la ausencia de causa orgánica y la falta de un deterioro global cognitivo.

.Trastorno de la personalidad

Tanto DD como comorbilidad.
\subsection*{Diagnóstico etiopatogénico y psicopatológico}
Destacar elementos del cuadro clínico del paciente en particular, agregando observaciones teóricas SOBRE el cuadro clínico.
\subsubsection*{Diagnóstico etiopatogénico}
Se plantea una causa multifactorial. Existen múltiples niveles complementarios e integrativos de comprensión e intento de explicación de esta enfermedad. La vía final es la interacción estrés-diátesis.
\paragraph{Comprensión biológica}
. Genética: hay una carga genética predisponente, dada la frecuencia de AF de trastornos afectivos, alcoholismo, IAE, comprobándose asociaciones con algunos cromosomas específicos. Pero la concordancia entre gemelos no es del 100%, por lo cual no es un factor determinante.
. Constitucional: desde las descripciones clásicas se plantea la asociación con el biotipo pícnico (Kretschmer), lo cual se ve reforzado por la constatación de una mayor prevalencia de alteraciones metabólicas.
. Hipótesis catecolaminérgica: involucra los neurotransmisores dopamina y noradrenalina, planteada en 1965 por Bunney y Davis.
. Hipótesis serotinérgica: planteada por Coppen y Lappin en 1969. Cambios primarios en los sistemas monoaminérgicos y cambios en la modulación realizada por el sistema serotoninérgico. Existiría una disregulación en estas vías.
. Existiría una alteración de la carga alostérica al estrés, es decir la capacidad de conservar la estabilidad. Sería una enfermedad de la respuesta, del retorno a la normalidad.
. Se postula la existencia de un fenómeno de kindling límbico-amigdalino-prefrontal: en los sucesivos episodios, el desencadenante exógeno es menor y finalmente el fenómeno adquiere autonomía de las causas externas.
. Se detectan también cambios neuroendócrinos en: CRH, RCRH, VSP, ACTH, cortisol.

Una causa frecuente de descompensación es el abandono de medicación.
\paragraph{Comprensión psicológica}
Puede encontrarse dificultad para superar pérdidas y para adaptarse a situaciones nuevas. Sobre un terreno de vulnerabilidad actúan factores psicosociales: pérdidas, dificultades interpersonales.

Hay etapas vitales con mayor riesgo de síntomas afectivos: adolescencia, embarazo, puerperio, climaterio, menopausia, envejecimineto, duelo.
\paragraph{Comprensión social}
Estresores sociales como factor exterior sobre la vulnerabilidad de base. Pérdida de roles laborales, pérdida de posición social.
\subsubsection*{Diagnóstico psicopatológico}
Para Binswanger se trata de una modalidad regresiva global con modificación de la estructura temporal de la vida psíquica, con desencadenamiento de los impulsos.

Para la Teoría Organodinámica de Ey de la desestructuración de conciencia (el Ser Consciente), el maníaco presenta una desestructuración del orden del cuerpo mental en su nivel ético-temporal (de 1° grado). Etico por la incapacidad de postergar la realización de los deseos y temporal por la estrechez del presente en un punto virtual siempre renovado y sin trascendencia, con distensión, laxitud, relajamiento de la continuidad histórica del individuo. Determina una pérdida de la capacidad de adaptación a las exigencias del aquí y ahora. Comporta un aspecto negativo (regresivo o deficitario) y aspecto positivo, de liberación de instancias inferiores.

Para los psicoanalistas se trata de una regresión a las etapas infantiles del desarrollo psicosexual, anteriores a toda frustración exterior. Las pulsiones se liberan, especialmente las pregenitales. En este sentido, sería lo contrario del melancólico, pues el maníaco se precipita a la satisfacción inmediata de las pulsiones como una forma de escapar de la angustia. Sus mecanismos de defensa son la negación de la pérdida de objeto (mal manejo de una pérdida) y la omnipotencia ante la melancolía (en todo maníaco hay un fondo nuclear melancólico).
\subsection*{Paraclínica}
El diagnóstico es clínico. La paraclínica está destinada a realizar una valoración general del paciente, descartar diagnósticos diferenciales y con miras a los diferentes recursos terapéuticos de los que disponemos. Lo solicitaremos desde un punto de vista integral: biológico, psicológico y social.

Solicitaremos la historia clínica previa o su resumen para objetivar los antecedentes clínicos y de recursos terapéuticos. En caso que sea necesario se pedirá información al juez o a la policía.
\subsubsection*{Biológico}
.Valoración general

Realizaremos una anamnesis médica general al paciente y terceros. Un examen físico completo con énfasis en el aspecto neurológico (con el paciente sedado, si corresponde), buscando elementos de organicidad que nos pongan en la pista de una patología reversible determinante de la expresión clínica actual. En particular buscaremos elementos de hipertensión endocraneana (fondo de ojo), estigmas de UISP, focos infecciosos.

Solicitaremos exámenes de valoración general:
\begin{itemize}
	\item Metabólica: glicemia, perfil lipídico (para establecer línea de base ante el eventual uso de fármacos con repercusión metabólica).
	\item Hematológica: hemograma, crasis
	\item Renal: función renal, ionograma (con calcio)
	\item Hepática: perfil hepático
	\item Infecciosa: HIV, VDRL y si la situación clínica lo determina: HVB, HVC
	\item Tóxica: screening de sustancias psicoactivas en orina
	\item Endócrina: función tiroidea
	\item Cardiovascular: ECG
\end{itemize}
Si es pertinente: test de embarazo. Si hay amenorrea: prolactinemia.

Si es clínicamente necesario: TAC, consulta con neurólogo, enzimograma cardíaco (cocaína).

Se solicitarán consultas con especialistas según hallazgos.

.Con miras a posibles tratamientos

Litio: examen de orina, función renal (contraindicado en insuficiencia renal), función tiroidea (por comorbilidad, por factor causal y como línea de base por efecto secundario del litio), test de embarazo (el litio es teratogénico), ionograma (hiponatremia aumenta probabilidades de intoxicación por litio), hemograma (litio da leucocitosis), ECG (por efectos sobre la conducción cardíaca). Descartar estados que lleven a balance negativo de Na (dieta hiposódica, diuréticos) ya que en su eliminación, el LI se intercambia por Na a nivel renal y un déficit de este ion puede llevar a un aumento de la litemia con el consiguiente riesgo de intoxicación.

ECT: ECG y consulta con cardiólogo para descartar IAM reciente o arritmias ventriculares graves que contraindicarían su realización). Rx Tx (para descartar aneurisma de aorta). Fondo de ojo/TAC: para descartar HTEC. En algunos casos puede plantearse la realización de EEG. En pacientes añosos y según el caso clínico puede solicitarse una evaluación del estado cognitivo basal.

TIP: Contraindicaciones de ECT: IAM reciente, arritmias inestables, aneurisma de aorta, PEIC con HTEC.

Carbamazepina: hemograma (por ser depresor de la médula ósea, contraindicado en caso de citopenia), funcional y enzimograma hepático (por determinar movilización enzimática y potencial toxicidad hepática).

Acido valproico: funcional y enzimograma hepático, hemograma.
\subsubsection*{Psicológico}
Será diferido hasta superada la agudeza del cuadro actual, salvo la existencia de dudas diagnósticas. Realizaremos entrevistas para evaluar las características propias del paciente y sus capacidades para en un futuro integrarse a grupos de psicoterapia.

Realizaremos tests de personalidad proyectivos (Rorscharch y TAT) y no proyectivos (Minessota), que nos informarán sobre los mecanismos de defensa, integridad yoica, manejo de la agresividad y rasgos de personalidad.

Realizaremos test de nivel, si hay dudas. La realización de tests no es imprescindible y no retrasará el inicio del tratamiento.
\subsubsection*{Social}
Realizaremos entrevistas con familiares a los efectos de valorar: red de soporte y vínculos, características de los tratamientos previos y sus resultados, funcionamiento premórbido e intercrítico, antededentes de corte existencial, inventario de eventos vitales, valorar medio socio-económico-cultural.

Informaremos a la familia sobre los diagnósticos positivos y diferenciales, las dudas, los tratamientos disponibles, sus riesgos y beneficios y nuestra opinión sobre lo mejor para este paciente en este momento. La información será transmitida siempre con un objetivo de psicoeducación. Pediremos consentimiento informado por la posibilidad de ECT.
\subsection*{Tratamiento}
El tratamiento será dinámico, adaptado constantemente a la evolución clínica y a la aparición de complicaciones, integrado por recursos farmacológicos, psicológicos y sociales\cite{yatham2018canadian}.

. Objetivos inmediatos: remitir rápidamente el cuadro actual, descartar causa orgánica, prevenir complicaciones.
. Objetivos mediatos: compensar la enfermedad de fondo, prevenir futuras recaídas, prolongar los períodos de remisión, reinsertar al paciente en su mejor nivel de funcionamiento.

Lo internaremos en sala de patología aguda de hospital psiquiátrico por: gran exaltación, presencia de un delirio, agresividad, ansiedad, riesgo suicida, alteración de las conductas basales, carencia de continencia familiar.

Lo ideal es internarlo en sala individual, en un entorno con poca estimulación, sin elementos de riesgo (ventanas, espejos), con asistencia de enfermería especializada las 24 horas y acompañante continentador a permanencia. Límites claros y firmes. Evitar interacciones provocativas.

La internación será en sala de hospital general si reconocemos una causa determinante orgánica tratable y reversible que necesite de medios asistenciales más complejos. 

Será dentro de lo posible con su consentimiento, pero debemos hacerla aún de forma compulsiva, evaluando riesgo/beneficio. La internación es una medida de protección del paciente y de terceros.

De esta forma lograremos: continentar al paciente calmando su sufrimiento psíquico, tratar su excitación / ansiedad / delirio, acortar la duración de la crisis actual, mejorando el pronóstico; ajustar la medicación; proteger al paciente y terceros de las posibles complicaciones medicolegales, vigilar fugas e IAEs, descartar causa orgánica.

Se llevará adelante por un equipo multidisciplinario. Indicaremos controles de enfermería especializada. Permitiremos visitas de figuras continentadoras. Realizaremos adecuado aporte nutricional. Se verificará la toma de medicación.
\subsubsection*{Farmacológico}
\paragraph{Tratamiento del episodio}
Tratamiento de la fase aguda, busca la remisión de síntomas específicos.

Por niveles de evidencia:
\begin{itemize}
	\item Primera línea en monoterapia: litio, quetiapina, divalproato, asenapina, paliperidona, risperidona, cariprazina.
	\item Primera línea en combinación: quetiapina + litio/divalproato, risperidona + litio/divalproato, aripiprazol + litio/divalproato, asenapina + litio/divalproato.
	\item Segunda línea: olanzapina, carbamazepina, olanzapina + litio/divalproato, litio + divalproato, ziprasidona, haloperidol, ECT.
\end{itemize}

Por situación clínica:
\begin{itemize}
	\item Manía típica (eufórica) sin síntomas psicóticos: Litio (o Divalproato) \faArrowCircleRight + benzodiacepina \faArrowCircleRight + antipsicótico atípico \faArrowCircleRight DVP + Litio \faArrowCircleRight Cambiar de antipsicótico \faArrowCircleRight DVP + Li + CBZ \faArrowCircleRight ECT
	\item Manía mixta (disfórica): Divalproato \faArrowCircleRight mismo esquema que manía típica.
	\item Hipomanía: mismo esquema que manía eufórica (con menos énfasis en el uso de antipsicóticos).
	\item Manía con síntomas psicóticos: Divalproato (o Litio) + AAP (o CAP) \faArrowCircleRight cambiar AAP o + BZD \faArrowCircleRight DVP + LI \faArrowCircleRight cambiar AAP o AAP + CAP \faArrowCircleRight ECT \faArrowCircleRight DVP + LI + CBZ (o agregar Clozapina)
	\item Manía en paciente con ciclado rápido: DVP \faArrowCircleRight DVP + (LI o CBZ) \faArrowCircleRight + AAP \faArrowCircleRight DVP + LI + CBZ \faArrowCircleRight Clozapina \faArrowCircleRight Lamotrigina \faArrowCircleRight Gabapentina \faArrowCircleRight ECT
	\item Depresión en bipolar (no psicótica - no ciclos rápidos) sin medicación previa moderado: Li \faArrowCircleRight +AD
	\item Depresión en bipolar (no psicótica - no ciclos rápidos) sin medicación previa severo: LI (o DVP) + AD \faArrowCircleRight LI + DVP.
	\item Si estaba con estabilizador: maximizar estabilizador como primer paso \faArrowCircleRight Li + DVP \faArrowCircleRight + AD o Lamotrigina \faArrowCircleRight + AD si no tenía (o cambiarlo).
	\item Si hay refractariedad en la depresión: ECT \faArrowCircleRight T3 \faArrowCircleRight Otros estabilizadores \faArrowCircleRight Clozapina o estimulante o fototerapia.
	\item Episodio depresivo psicótico: igual pauta, con más énfasis en antipsicóticos atípicos (ECT a cualquier altura del algoritmo).
	\item Depresión en paciente con ciclado rápido: DVP \faArrowCircleRight + (Li o CBZ o Lamotrigina) \faArrowCircleRight + AD \faArrowCircleRight cambio de AD \faArrowCircleRight T3/T4 o AAP \faArrowCircleRight Gabapentina o Clozapina o Fototerapia -> ECT.
\end{itemize}

TIP: Regla general: LIT en manía típica, DVL en el resto (por se de más fácil manejo).

\faMedkit Ansiedad: inicialmente usaremos benzodiacepinas, como el lorazepam (del cual contamos con presentación parenteral de ser necesario). Iniciamos con dosis de 2 mg v/o c/6-8 horas. Una alternativa es el uso de clonazepam a dosis de 2 a 4 mg c/8-12 horas, pudiendo llegar a 12 mg/día (con efecto sobre la disforia y la impulsividad). Ambos fármacos actúan sobre receptores GABA.

\faMedkit Excitación psicomotriz
\begin{itemize}
	\item Primera línea: lorazepam IM (2 mg IM a repetir). En Uruguay no contamos con: loxapina inhalada, aripiprazol IM, , olanzapina IM.
	\item Segunda línea: haloperidol IM (5 mg/dosis, máximo: 15 mg/día), haloperidol + midazolam  (7.5 mg/dosis, máximo: 15 mg/día). En Uruguay no contamos con: asenapina SL, prometazina IM, risperidona TDO, ziprasidona IM.
	\item Tercera línea: haloperidol VO (5 mg/dosis, máximo: 15 mng/día), quetiapina VO (dosis variable), risperidona VO (2 mg/dosis). En Uruguay no contamos con loxapina IM.
\end{itemize}
De nos ser suficiente con la benzodiacepina, utlizaremos antipsicóticos sedativos, sustityéndola o como complemento. Indicaremos Levomepromazina 25 mg i/m c/8 horas con un posible refuerzo de dosis nocturno (50 mg H 20) evaluando el pasaje a v/o, atentos a los efectos anticolinérgicos e hipotensión postural.

\faMedkit Delirio

Escenario 1: vía IM.
Indicaremos neurolépticos incisivos antidelirantes del grupo de las butirofenonas, como el Haloperidol, que actúa bloqueando los receptores dopaminérgicos D2 córtico-mesolímbicos, comenzando con dosis de 5 mg i/m horas 8 y 20 a fin de lograr la seguridad en la toma de medicación y niveles terapéuticos adecuados en los sitios de acción. Destacamos además el efecto antimaníaco de esta medicación además de la acción sobre la excitación y los síntomas psicóticos. La dosis y la vía se ajustarán según respuesta clínica. El Haloperidol puede elevarse a dosis de 15-20 mg/día si la evaluación clínica lo indica. Pasaremos la totalidad de la dosis a la noche, en lo posible.

Estaremos atentos a la aparición de efectos secundarios de los neurolépticos. En caso de un paciente de riesgo (varón, menor de 35 años, AF de Enfermedad de Parkinson), indicaremos Biperideno de forma preventiva a dosis de 2 mg H 8 y H14 por v/o por vía i/m. Las formas de liberación prolongada se pueden dar solamente en la mañana.

De aparecer distonía aguda, acatisia, síntomas extrapiramidales (rigidez, rueda dentada, bradiquinesia, temblor) comenzaremos con Biperideno, evaluando la posibilidad de disminuir las dosis del antispsicótico (y/o concentrar la dosis en la noche) y discontinuándolo en un plazo de 3 meses si la evolución lo permite.

Mantendremos el Haloperidol i/m de 3 a 5 días y pasaremos luego a v/o según disminuya la exaltación y el delirio. Debemos retirarlo completamente lo antes posible por riesgo de viraje hacia la depresión, con aumento de frecuencia de crisis y reducción de períodos intercríticos. Debemos considerar además que los paciente con trastornos afectivos tienen también mayor riesgo de presentar disquinesias tardías.

Escenario 2: VO

Consideramos de elección el uso de antipsicóticos atípicos por la menor incidencia de efectos secundarios. Solo en caso de que se requiera medicación intramuscular, usaremos Haloperidol i/m que pasaremos luego a vía oral.

Olanzapina (primera línea, con o sin síntomas psicóticos): iniciando con 5 mg/día en toma única, aumentando a 10 si hay buena tolerancia, pudiendo aumentar hasta 20 mg/día. Propiedades como antipsicótico y como estabilizador del humor.

Risperidona: comenzamos con 2 mg/día v/o en 2 tomas, aumentando hasta 4.5 mg/día en 2 tomas. Luego 1 semana puede administrarse en una única toma nocturna. Máximo: 6 mg/día (dosis más altas aumentan el riesgo de efectos secundarios).

\faMedkit Insomnio

De persistir el insomnio a pesar de los ansiolíticos, indicaremos Midazolam i/m, o si la situación lo permite, Flunitrazepam 2 mg v/o a la noche. La restauración de un ciclo sueño-vigilia normal es fundamental para la recuperación clínica.

\faBolt ECT

Si en 10-15 días no obtenemos mejoría (disminución de exaltación, disminución de entrega a la experiencia maníaca) evaluaremos las posibles causas y consideraremos el aumento de la dosis de los fármacos y evaluaremos la realización de ECT, para lo cual solicitaremos consentimiento informado a familiar.

La ECT se considera de primera línea en caso de afectación severa de conductas basales (rechazo de alimentos), repercusión general, mal estado general y cuando los fármacos están contraindicados por algún motivo. El mecanismo de acción de la ECT es desconocido.

Indicaremos una serie inicial de 8 a 10 sesiones, una día por medio, realizadas con asistencia de anestesista, psiquiatra y enfermería especializada, bajo monitoreo ECG y EEG. Descartaremos previamente elementos que la contraindiquen, como se especificó en el apartado Paraclínica.

Puede ser necesario la suspensión de benzodiacepinas en las horas previas dado que éstas aumentan el umbral convulsivo (se puede sustituir por Levomepromazina). La dosis de litio de la mañana se postergará por mayor riesgo de confusión mental y amnesia post ECT.

.Tratamiento de la enfermedad de fondo

El tratamiento de la fase aguda será seguido de un tratamiento de continuación (4-12 meses) donde se busca mantener el control del episodio actual y se comienza la fase de prevenir o atenuar futuros episodios.

\faPills Litio

De primera elección en manías típicas (sin estados mixtos, sin ciclado rápido, sin abuso de sustancias). Indicaremos desde el inicio del tratamiento. Pese a su latencia de 8-10 días proporciona un efecto antimaníaco más específico, además de ser estabilizador del humor y profiláctico de recidivas. Comenzaremos con 300 mg v/o c/8 horas, con las comidas, probando tolerancia, ya que al inicio son frecuentes los trastornos digestivos leves que, al igual que la sintomatología neurológica inespecífica (letargia, fatiga, debilidad muscular y temblor fino distal), polidipsia y poliuria, son todos fenómenos reversibles y transitorios. Indicaremos abundantes líquidos v/o. Estaremos atentos a la aparición de estos síntomas. Controlaremos la aparición de signos incipientes de toxicidad: ataxia, temblor grueso, disartria, fasciculaciones.

\faRadiation Atentos a los signos de intoxicación por litio (ATeGDiF): ataxia, temblor grueso, disartria, fasciculaciones
\begin{itemize}
	\item Intoxicación leve: apatía, letargia, debilidad, temblor fino, síntomas gastrointestinales (náuseas, vómitos, diarreas).
	\item Intoxicación moderada: temblor grueso, ataxia, lenguaje lento, confusión, hiperreflexia, clonus, cambios ECG inespecíficos.
	\item Intoxicación grave: convulsiones, coma, shock, fasciculaciones generalizadas, alteraciones del ECG (todo tipo), arritmias, muerte \footnote{Osés, I., Burillo-Putze, G., Munné, P., Nogué, S., \& Pinillos, M.A.. (2003). Intoxicaciones medicamentosas (I): Psicofármacos y antiarrítmicos. Anales del Sistema Sanitario de Navarra, 26(Supl. 1), 49-63}.
\end{itemize}
Si el paciente es añoso, o con problemas renales, o sensible a efectos secundarios: comenzar con 150 mg v/o c/8.

Probablemente lleguemos a un rango de dosis de 900 a 1800 mg). La posología en 1 o 2 tomas diarias no modifica la eficacia y puede minimizar algunos efectos adversos, además de favorecer el cumplimiento con el tratamiento.

A los 5-7 días (tiempo en que se tarda en llegar al estado de meseta) realizaremos la primera litemia (12 horas luego de la última toma, por la variación pico-valle) y según ella iremos ajustando la dosis hasta llegar al rango terapéutico establecido de 0.8-1.2 mEq/l (según el paciente aprox 900-1800 mg/día). La litemia se repetirá semanalmente el primer mes y luego mensual durante el primer semestre. La dosis se ajustará según concentraciones séricas y cuadro clínico. El nivel plasmático depende de muchos factores, entre ellos: masa corporal, filtrado glomerular e idiosincrasia farmacológica individual. Para la crisis de manía se postula un rango terapéutico de 1.0 a 1.2 mEq/l y para la profilaxis 0.8 a 1.0 mEq/l.

El litio no actúa en el espacio sináptico sino intracelularmente, en los sistemas de proteína G y segundos mensajeros. Por eso la latencia de hasta 3 semanas para el inicio de los efectos terapéuticos.

Las litemias se realizarán cada 3 meses o más seguido si hay efectos tóxicos o incumplimiento del tratamiento. Se realizará una función renal evaluando creatininemia cada 6 meses y ante cambios de dosis, de respuesta terapéutica o ante sospecha de falla renal. Se realizará ECG cuando sea necesario, pero solo una arritmia grave determina la suspensión del Litio. Se realizará TSH cada 6 mees si hay clínica de disfunción tiroidea.

En paciente con ciclado rápido, se postula que el litio tiene menor eficacia, presentando mejor respuesta a Ácido Valproico o Carbamazepina. La disfunción tiroidea puede ser un factor predisponente para el ciclado rápido


\faTasks: Poner predictores de buena respuesta al litio (ej. AF afectivos).

\faPills Ácido Valproico

Actúa sobre la neurotransmisión GABA. Se plantea una dosis inicial de 250 mg c/12 horas (probando tolerancia, sobre todo por efecto gastrointestinales), que se aumentará hasta dosis máxima de 20 mg/kg/día (en 2 o 3 tomas) o concentraciones plasmáticas de 50 a 125 mcg/ml. Se logra una meseta plasmática al cabo de 2 semanas. El efecto puede tener una latencia de 3 semanas. Precaución en pacientes con AP de disfunción hepática. No dar en embarazo o lactancia. Alta unión a proteínas. Ajustar dosis en insuficiencia renal y hepática, en ancianos, coagulopatías, dislipemias severas, desnutrición. Aumenta los niveles de AAS, fenitoina, carbamazepina, warfarina, diazepam, lorazepam, amitriptilina. Efectos secundarios: intolerancia digestiva, sedación, astenia, rash cutáneo, leucopenia y plaquetopenia benignas, alopecia, temblor. Puede haber como efecto idiosincrático: insuficiencia hepática y agranulocitosis.
Es más eficaz en los episodios mixtos que en la manía clásica.

\faExclamationTriangle : Las presentaciones de divalproato de sodio tienen mejor tolerancia gástrica. Tienen una cobertura entérica por lo que no es recomendable partir el comprimido.

Precaución: trombocitopenia, insuficiencia hepática. Realizaremos la valoración paraclínica descrita previo a su uso.

\faPills Carbamazepina

Generalmente como coadyuvante de otro estabilizador del humor. Dosis iniciales de 200 mg v/o c/12 horas que se aumentará hasta 1200-1400 mg/día. con control de función hepática y hematológica. Se plantea lograr una concentración terapéutica de 4-12 mcg/ml de plasma.

\faLightbulb Predictores de respuesta a antivonvulsivantes: ciclado rápido, episodio mixto, pobre respuesta a litio, manía secundaria, comorbilidad con abuso de sustancias.
\paragraph{Mantenimiento}
Tratamiento de mantenimiento:
\begin{itemize}
	\item Primera línea. Nivel 1: Quetiapina, Litio, Lamotrigina. Nivel 3: Lurasidona + Litio/Divalproato. Nivel 4: Lurasidona, Lamotrigina adjunto.
	\item Segunda línea. Nivel 1: Divalproato. Nivel 4: ECT.
\end{itemize}
\subsubsection*{Psicológico}
Haremos entrevistas diarias con el paciente, con el objetivo de: crear un vínculo terapéutico cálido y continentador, evaluar la evolución y las oscilaciones diarias, investigar y reforzar los aspectos sanos, evaluar el factor desencadenante si lo hubiera, evaluar factores de vulnerabilidad para próximos episodios, darle referencias de realidad sin confrontarlo. Se fomentará la alianza terapéutica.

En la fase de mantenimiento, es de primera línea la psicoeducación. De segunda línea: psicoterapia cognitivo-comportamental, terapia enfocada en la familia, terapia de ritmo social e interpersonal y soporte entre pares\cite{yatham2018canadian}.

\subsubsection*{Social}
Entrevistas reiteradas con familiares con fines de psicoeducación sobre el diagnóstico y tratamiento instituido, los pronósticos y su relación con la adherencia al tratamiento, la importancia de los controles y afianzar el vínculo como aliado terapéutico. Evaluaremos el impacto de la patología en la autoestima del paciente. Facilitaremos el acceso a biblioterapia. Contactaremos a la familiar con grupos de psicoeducación de familiares de pacientes bipolares (en especial grupos que sigan el modelo propuesto por Colom y Vieta).

El familiar es un aliado en la evitación del abandono del tratamiento y en la detección de signos precoces de descompensación, que llevan a la consulta precoz.

Se evaluarán las condiciones laborales evitando turnos rotativos, favoreciendo la estabilidad en el ciclo sueño-vigilia.

Mujer en edad genital activa: derivar a planificación familiar (potencial teratogénico de algunos fármacos, aumento de posibilidades de descompensaciones vinculadas a ciclos reproductivos).
\subsubsection*{Alta hospitalaria}
Dependerá de la respuesta al tratamiento. Se dará al haber: remisión de sintomatología psicótica, aparición de crítica, normalización de las conductas basales y el autocuidado, adquisición de conceptos básicos de psicoeducación, compromiso con el paciente y la familia en el control evolutivo en policlínica. Se retirarán, en la medida de lo posible los fármacos que no sean necesarios.

A largo plazo lo ideal es la monoterapia con estabilizadores del humor. Sabemos que esto no siempre es posible que en general se recurre a una combinación de fármacos a las mismas dosis con las que se obtuvo la mejoría (ver esquema previo con secuencia de uso de fármacos de 1a, 2a y 3a linea).

Es fundamental el seguimiento para el control evolutivo, el cumplimiento con el tratamiento, la dosificación de fármacos en sangre (si corresponde).

\faTasks Poner control de fármacos según CANMAT

\subsection*{Evolución y pronóstico}
\subsubsection*{Evolución}
Estamos ante una enfermedad crónica de manifestación episódica, estando el pronóstico supeditado al subtipo clínico, la respuesta y adhesión al tratamiento, el funcionamiento psicosocial y la presencia de estresores.

Sin tratamiento evoluciona hacia el aumento de la frecuencia de las crisis, con períodos libres de síntomas más cortos, con crisis más intensas y prolongadas y con refractariedad a la terapéutica profiláctica. Espontáneamente una crisis de manía remite al cabo de 3 a 6 meses y una de melancolía al cabo de 8 a 12 meses.

Con tratamiento adecuado y adherencia al mismo, se logra en un alto porcentaje de pacientes la remisión de las crisis, prolongación de períodos intercríticos, disminución de la frecuencia de las crisis, las crisis que ocurren son de menor duración y de menor intensidad, con menor necesidad de internaciones y de medicación, con menor repercusión psicológica individual, de pareja y familiar, menor compromiso laboral y en los estudios.
\subsubsection*{Pronóstico}
Pronóstico psiquiátrico inmediato: bueno con el tratamiento instituido.

Pronóstico vital inmediato: supeditado a la exclusión de patologías orgánicos, al riesgo de IAE, autolesiones, conductas de riesgo y heteroagresividad.

Pronóstico psiquiátrico alejado: sujeto a la adhesión al tratamiento.

Pronóstico vital alejado: sujeto a descompensaciones con conductas de riesgo; agresividad UISP, alcoholismo, sexualidad (HIV, VDRL, HVB, HVC). Comorbilidad médica (insuficiencia renal, enfermedades cardíacas).

Se considera refractario a un tratamiento si no ha habido respuesta significativa luego de 12 semanas de niveles terapéuticos en sangre.

A mayor edad, tienden a disminuir los períodos intercríticos, con mayor frecuencia y duración de las crisis (kindling).

De tratarse de una mujer en edad genital activa: control de natalidad y anticoncepción con ginecólogo.

Suicidio: en el trastorno bipolar bipolar hay 30 veces más riesgo que en la población general. Se registra un 15\% de suicidio consumado.

Elementos de mal pronóstico
\begin{itemize}
	\item Presencia de comorbilidad (deterioro cognitivo, consumo de sustancias)
	\item Alta frecuencia de episodios
	\item Estresores ambientales / psicosociales
\end{itemize}
\subsection*{En suma}
Hemos visto un paciente de sexo \faQuestionCircle, de \faQuestionCircle años de edad, con un MSEC \faQuestionCircle, con AF de \faQuestionCircle, con AP de \faQuestionCircle, que consulta por \faQuestionCircle síntomas, en quien diagnosticamos un Trastorno Bipolar de tipo I / II, de características \faQuestionCircle. con un episodio actual \faQuestionCircle, con características \faQuestionCircle, en comorbilidad con \faQuestionCircle. Hemos planteado diagnósticos diferenciales con \faQuestionCircle, hemos estudiado con \faQuestionCircle, y hemos tratado con \faQuestionCircle. Planteamos una evolución buena con el tratamiento indicado, dependiendo el pronóstico a largo plazo de la adherencia al tratamiento.
