\documentclass[encares.tex]{subfiles}
\begin{document}
=== F48 Otros trastornos neuróticos

==== F48.1 Trastorno de despersonalización-desrealización

===== Clínica

El individuo se queja espontáneamente de la vivencia de que su propia actividad mental, su cuerpo, su entorno o todos ellos, están cualitativamente transformados, de manera que se han vuelto irreales, lejanos o mecánicos (faltos de espontaneidad). El enfermo puede sentir que ya no es él el que rige su propia actividad de pensar, imaginar o recordar, de que sus movimientos y comportamiento le son de alguna manera ajenos, que su cuerpo le parece desvitalizado, desvinculado de sí mismo o extraño, que su entorno le parece falto de colorido y de vida, como si fuera artificial o como si fuera un escenario sobre el que las personas actúan con papeles predeterminados. En algunos casos, el enfermo puede sentir que se está observando a sí mismo desde cierta distancia o como si estuviera muerto. La queja de pérdida de los sentimientos es la más frecuente entre estos diversos fenómenos. El número de enfermos que sufre este trastorno de forma pura o aislado es pequeño. Por lo general los fenómenos de desrealización-despersonalización aparecen en el contexto de enfermedades depresivas, trastornos fóbicos y obsesivo-compulsivos. 

===== Pautas para el diagnóstico

. síntomas de despersonalización tales como que el enfermo siente que sus propias sensaciones o vivencias se han desvinculado de sí mismo, son distantes o ajenas, se han perdido, etc.
. síntomas de desrealización tales como que los objetos, las personas o el entorno parecen irreales, distantes, artificiales, desvaídos, desvitalizados, etc.
. el reconocimiento de que se trata de un cambio espontáneo y subjetivo y no ha sido impuesto por fuerzas externas o por otras personas (persiste una adecuada conciencia de enfermedad) y
. claridad del sensorio y evidencia de que no se trata de un estado tóxico confusional o de una epilepsia.

Incluye:

- síndrome de despersonalización-desrealización

===== Diagnóstico diferencial

* Individuo sano que experimenta la despersonalización-desrealización como fenómeno aislado: estados fatiga, privación sensorial, fenómenos hipnagógicos/hipnopómpicos, situaciones de peligro extremo.
* Secundaria a consumo de sustancias
* Causa orgánica: hipoglicemia, estados pre o posticales en epilepsia del lóbulo temporal.

DD con otros trastornos en los que se vivencia un "cambio de personalidad":

- Esquizofrenia: ideas delirantes de transformación o imposición y de vivencias de ser controlado
- Trastornos disociativos: donde no existe conciencia de que se ha producido un cambio
- Algunos casos de demencia incipiente. 


Jerarquía: si el síndrome de despersonalización-desrealización aparece como parte de un trastorno depresivo, fóbico, obsesivo-compulsivo o esquizofrénico que satisfacen las pautas diagnósticas respectivas, este último diagnóstico tiene preferencia como diagnóstico principal.
\end{document}