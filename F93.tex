\chapter{Síndrome de fatiga crónica}
\section*{Notas clínicas}
Nombre alternativo: encefalomielitis miálgica (F93.32).
Enfermedad crónica compleja que afecta mùltiples sitemas corporales cuya fisiopatología está en investigación. Afecta a cada individuo de forma diferente, es fluctuante con cambios impredecibles en el día, en la semana o en períodos más largos. Puede afectar diferentes aspectos de la vida diaria, familiar, social, emocional, laboral, académico.
Frecuentemente objeto de prejuicio o descreimineto, estigmatización.
Severidad:
\begin{itemize}
	\item Leve: pueden hacer tareas domésticas livianas, pueden tener dificultades con la movilidad. Pueden mantener una actividad laboral o académica pero en detrimento de otras actividades. Muchas veces optan por trabajar menos.
	\item Moderada: restricción en todas las actividades de la vida diaria, con oscilaciones. En general no trabajan/estudian. Necesidad de descansos en la tarde de 1-2 horas. Sueño nocturno de mala calidad, fragmentado.
	\item Severa: incapacidad para realizar tareas diarias mínimas. Severos trastornos cognitivos. Pueden requerir de silla de ruedas para moverse. Incapacidad para salir de la casa, o lo hacen con consecuencias prolongadas. La mayor parte del tiempo en cama. Sensibilidad extrema a la luz o el sonido.
	\item Muy severa: en cama todo el día, dependiente de cuidador, necesitan ayuda con higiene personal y para comer. Muy sensibles a estímulos externos. En ocasiones pueden no deglutir, necesitando de alimentación por sonda.
\end{itemize}
En la atención importa la individualización, atender el estigma, involucrar a familiarew / cuidadoares, sensibilidad al contexto socioeconómico, cultural, étnico.
\subsection*{Diagnóstico}
Sospecha:
\begin{itemize}
	\item Síntomas por más de 4 semanas (niños y adolescentes) o 6 semanas (adultos):
	\begin{itemize}
		\item Fatiga debilitante que empeora con la actividad, no causada por esfuerzo físico o emocional. No alivia de forma significativa con el descanso.
	\end{itemize}
	\item Afectación de pragmatismos
	\item Exclusión de otras patologías que expliquen los síntomas
\end{itemize}