\chapter{Trastornos de la conducta alimentaria}
\section*{Notas clínicas}
Los trastornos de la conducta alimentaria (TCA) son un conjunto de trastornos que se agrupan dentro de los «trastornos del comportamiento asociados a disfunciones fisiológicas y a factores somáticos» (F5x) en la CIE-10. El DSM-IV reconoce 2 categorías: anorexia nerviosa (AN) y bulimia nerviosa (BN), especificando una propuesta de criterios de investigación para el trastorno por atracón (TA). La CIE-10 es más abarcativa, incluyendo la AN (F50.0), la AN atípica (F50.1), BN (F50.2), BN atípica (F50.3), hiperfagia asociada a otros trastornos psicológicos (F50.4), vómitos asociados a otros trastornos psicológicos (F50.5) y categorías residuales.

Todos estos trastornos están caracterizados, en un grado variable, por alteraciones persistentes en el comportamiento asociado a la alimentación o al control del peso y a preocupaciones excesivas por el peso o figura footnote:[Hilbert A, Hoek HW, Schmidt R. Evidence-based clinical guidelines for eating disorders: international comparison. Curr Opin Psychiatry. 2017;30(6):423-437.].

La característica central de la AN es el bajo peso. El TA está caracterizado por episodios de atracones (hiperfagia con pérdida de control). En la BN existe, además, una conducta compensatoria (purgativa o no purgativa). Todos estos trastornos producen un impacto significativo en el funcionamiento y calidad de vida.

El inicio es más frecuente en la adolescencia o adultez temprana. Tanto la AN como la BN evolucionan en un 50% de los casos a la remisión a largo plazo. Se conoce menos de la evolución del TA. La AN puede tener una prevalencia de aproximadamente 4% de mujeres jóvenes. La BN y el TA tienen una prevalencia de vida de 1.0 y 1.9 % respectivamente.

\section*{Encare}
\subsection*{Agrupación sindromática}
\subsubsection*{Síndrome de alteración de la conducta alimentaria}
\begin{itemize}
	\item Atracones: ingesta de gran cantidad de alimentos acompañada de sensación de pérdida de control, seguido de culpa o autodesprecio. El atracón puede tener cantidades objetiva o subjetivamente grandes de comida. Generalmente egodistónico en la bulimia y egosintónico en el TPA.
	\item Conductas compensatorias: conductas destinadas a compensar el aumento de peso que provocaría el atracón:
** Purgativas: vómitos autoinducidos, laxantes, diuréticos
** No purgativas: ejercicio excesivo o períodos de restricción dietética.
	\item Cognitivo: rumiación sobre comida (tipo, composición, secuencia en el día, etc.)
\end{itemize}

\subsubsection*{Síndrome de distorsión del esquema corporal}
* Cognitivo: autoevaluación influida excesivamente por el peso o figura.
* Conductual: tiempo excesivo frente al espejo, uso de ropas que disimulen la figura, conductas que traducen desprecio por el propio cuerpo. Aislamiento social secundario a percepción alterada de la imagen. Acciones tendientes a la ocultación de su patología (mentiars, manipulación, etc.).
\subsubsection*{Síndromes accesorios}
* Síndrome de alteración del estado de ánimo: fluctuaciones de humor que pueden ser secundarias al atracón-vómito.
* Síndrome depresivo: incluye depresiones atípicas. Diferenciar de alteraciones del humor post-atracón.
* Síndrome de ansiedad-angustia.
* Síndrome de alteración del control de los impulsos, consumo de sustancias, sexualidad autodestructiva, cleptomanía.
* Síndrome obsesivo-compulsivo
\subsection*{Diagnóstico diferencial}

* Otros trastornos alimentarios
* Alteración alimentaria secundaria a trastorno de la personalidad.
* Trastorno dismórfico corporal.
* Trastornos neurológicos: epilepsia, tumores SNC, S° de Klüver-Bucy (agnosia visual, mordeduras, hiperfagia, hipersexualidad: muy raro).
* Otras causas médicas de vómitos excesivos.

\subsection*{Diagnóstico etiopatogénico}
* B: serotonina, NA, endorfinas, más AF de depresión.
* P: dificultad de separación, hiperaglutinación familiar, conflictiva familiar, criticismo parental.
* S: imagen, rol y género. Expectativas sociales.

.Psicopatología
CC: Distorsiones cognitivas, procesamiento de la información.
PD: ausencia de objetos transicionales (cuerpo), ambivalencia, oralidad (bueno/malo)

\subsection*{Paraclínica}
* General: peso, IMC base, crecimiento, desarrollo, hidratación, elementos de acidosis/alcalosis metabólica. Al EF: signo de Russell, aumento del tamaño de las glándulas salivales, lanigo.
* Ionograma completo (Ca, Mg [hipomagnesemia], fósforo), función renal, hemograma
* Función hepática: amilasemia [hiperamilasemia, indicador de persistencia de vómitos]
* ECG: bradicardia, hipotensión, arritmias x disionía.
* Sangre en heces por abuso de laxantes.
* Función tiroidea
* Hormonas:
** Estradiol (en mujeres) o testosterona (hombres) si hay desnutrición > 6 meses.
** LH y FSH si hay amenorrea.
* Densitometría ósea
* Nutricionista, ginecólogo, odontólogo, gastroenterólogo, endocrinólogo.
* Eventual TAC en AN
* βHCG, HIV, VDRL, perfil lipídico.

.Psicológico
Buscar abusos exual.

\subsection*{Tratamiento}
En equipo, prestando atención a las maniobras tendientes a la escición.
Tratamiento higiénico-dietético: estructuración de la alimentación (orden alimentario).
Tratamiento nutricional según lo indicado por nutricionista.
B: ISIS a altas dosis en BN o en TPA
P: TCC: estilos de afrontamiento, técnicas de manejo de estrés, reestructuración cognitiva. Entornos estructurados (permiten la observación de todas las etapas del ciclo alimentario). Terapia familiar: límites, contratransferencia.

\subsection*{Evolución y pronóstico}
Complicaciones:
\begin{itemize}
	\item * Metabólico, hidroelectrolítico, CV
	\item * Osteoporosis
	\item * Gastrointestinal: rotura esofágica, esofagitis, etc.
	\item * Odontológico: esmalte dental
	\item * Heridas en dedos.
	\item * Evolución a otros TCA
\end{itemize}

